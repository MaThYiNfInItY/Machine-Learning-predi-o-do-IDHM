% Options for packages loaded elsewhere
\PassOptionsToPackage{unicode}{hyperref}
\PassOptionsToPackage{hyphens}{url}
%
\documentclass[
]{article}
\usepackage{amsmath,amssymb}
\usepackage{iftex}
\ifPDFTeX
  \usepackage[T1]{fontenc}
  \usepackage[utf8]{inputenc}
  \usepackage{textcomp} % provide euro and other symbols
\else % if luatex or xetex
  \usepackage{unicode-math} % this also loads fontspec
  \defaultfontfeatures{Scale=MatchLowercase}
  \defaultfontfeatures[\rmfamily]{Ligatures=TeX,Scale=1}
\fi
\usepackage{lmodern}
\ifPDFTeX\else
  % xetex/luatex font selection
\fi
% Use upquote if available, for straight quotes in verbatim environments
\IfFileExists{upquote.sty}{\usepackage{upquote}}{}
\IfFileExists{microtype.sty}{% use microtype if available
  \usepackage[]{microtype}
  \UseMicrotypeSet[protrusion]{basicmath} % disable protrusion for tt fonts
}{}
\makeatletter
\@ifundefined{KOMAClassName}{% if non-KOMA class
  \IfFileExists{parskip.sty}{%
    \usepackage{parskip}
  }{% else
    \setlength{\parindent}{0pt}
    \setlength{\parskip}{6pt plus 2pt minus 1pt}}
}{% if KOMA class
  \KOMAoptions{parskip=half}}
\makeatother
\usepackage{xcolor}
\usepackage[margin=1in]{geometry}
\usepackage{color}
\usepackage{fancyvrb}
\newcommand{\VerbBar}{|}
\newcommand{\VERB}{\Verb[commandchars=\\\{\}]}
\DefineVerbatimEnvironment{Highlighting}{Verbatim}{commandchars=\\\{\}}
% Add ',fontsize=\small' for more characters per line
\usepackage{framed}
\definecolor{shadecolor}{RGB}{248,248,248}
\newenvironment{Shaded}{\begin{snugshade}}{\end{snugshade}}
\newcommand{\AlertTok}[1]{\textcolor[rgb]{0.94,0.16,0.16}{#1}}
\newcommand{\AnnotationTok}[1]{\textcolor[rgb]{0.56,0.35,0.01}{\textbf{\textit{#1}}}}
\newcommand{\AttributeTok}[1]{\textcolor[rgb]{0.13,0.29,0.53}{#1}}
\newcommand{\BaseNTok}[1]{\textcolor[rgb]{0.00,0.00,0.81}{#1}}
\newcommand{\BuiltInTok}[1]{#1}
\newcommand{\CharTok}[1]{\textcolor[rgb]{0.31,0.60,0.02}{#1}}
\newcommand{\CommentTok}[1]{\textcolor[rgb]{0.56,0.35,0.01}{\textit{#1}}}
\newcommand{\CommentVarTok}[1]{\textcolor[rgb]{0.56,0.35,0.01}{\textbf{\textit{#1}}}}
\newcommand{\ConstantTok}[1]{\textcolor[rgb]{0.56,0.35,0.01}{#1}}
\newcommand{\ControlFlowTok}[1]{\textcolor[rgb]{0.13,0.29,0.53}{\textbf{#1}}}
\newcommand{\DataTypeTok}[1]{\textcolor[rgb]{0.13,0.29,0.53}{#1}}
\newcommand{\DecValTok}[1]{\textcolor[rgb]{0.00,0.00,0.81}{#1}}
\newcommand{\DocumentationTok}[1]{\textcolor[rgb]{0.56,0.35,0.01}{\textbf{\textit{#1}}}}
\newcommand{\ErrorTok}[1]{\textcolor[rgb]{0.64,0.00,0.00}{\textbf{#1}}}
\newcommand{\ExtensionTok}[1]{#1}
\newcommand{\FloatTok}[1]{\textcolor[rgb]{0.00,0.00,0.81}{#1}}
\newcommand{\FunctionTok}[1]{\textcolor[rgb]{0.13,0.29,0.53}{\textbf{#1}}}
\newcommand{\ImportTok}[1]{#1}
\newcommand{\InformationTok}[1]{\textcolor[rgb]{0.56,0.35,0.01}{\textbf{\textit{#1}}}}
\newcommand{\KeywordTok}[1]{\textcolor[rgb]{0.13,0.29,0.53}{\textbf{#1}}}
\newcommand{\NormalTok}[1]{#1}
\newcommand{\OperatorTok}[1]{\textcolor[rgb]{0.81,0.36,0.00}{\textbf{#1}}}
\newcommand{\OtherTok}[1]{\textcolor[rgb]{0.56,0.35,0.01}{#1}}
\newcommand{\PreprocessorTok}[1]{\textcolor[rgb]{0.56,0.35,0.01}{\textit{#1}}}
\newcommand{\RegionMarkerTok}[1]{#1}
\newcommand{\SpecialCharTok}[1]{\textcolor[rgb]{0.81,0.36,0.00}{\textbf{#1}}}
\newcommand{\SpecialStringTok}[1]{\textcolor[rgb]{0.31,0.60,0.02}{#1}}
\newcommand{\StringTok}[1]{\textcolor[rgb]{0.31,0.60,0.02}{#1}}
\newcommand{\VariableTok}[1]{\textcolor[rgb]{0.00,0.00,0.00}{#1}}
\newcommand{\VerbatimStringTok}[1]{\textcolor[rgb]{0.31,0.60,0.02}{#1}}
\newcommand{\WarningTok}[1]{\textcolor[rgb]{0.56,0.35,0.01}{\textbf{\textit{#1}}}}
\usepackage{graphicx}
\makeatletter
\def\maxwidth{\ifdim\Gin@nat@width>\linewidth\linewidth\else\Gin@nat@width\fi}
\def\maxheight{\ifdim\Gin@nat@height>\textheight\textheight\else\Gin@nat@height\fi}
\makeatother
% Scale images if necessary, so that they will not overflow the page
% margins by default, and it is still possible to overwrite the defaults
% using explicit options in \includegraphics[width, height, ...]{}
\setkeys{Gin}{width=\maxwidth,height=\maxheight,keepaspectratio}
% Set default figure placement to htbp
\makeatletter
\def\fps@figure{htbp}
\makeatother
\setlength{\emergencystretch}{3em} % prevent overfull lines
\providecommand{\tightlist}{%
  \setlength{\itemsep}{0pt}\setlength{\parskip}{0pt}}
\setcounter{secnumdepth}{-\maxdimen} % remove section numbering
\ifLuaTeX
  \usepackage{selnolig}  % disable illegal ligatures
\fi
\IfFileExists{bookmark.sty}{\usepackage{bookmark}}{\usepackage{hyperref}}
\IfFileExists{xurl.sty}{\usepackage{xurl}}{} % add URL line breaks if available
\urlstyle{same}
\hypersetup{
  pdftitle={APLICAÇÃO DO MACHINE LEARNING NA PREVISÃO DE ÍNDICES ECONÔMICOS: O IDHM COM O MODELO RANDOM FOREST DE 2012 À 2021. DOI: 10.13140/RG.2.2.36759.55204},
  pdfauthor={Matheus Assis de Oliveira},
  hidelinks,
  pdfcreator={LaTeX via pandoc}}

\title{APLICAÇÃO DO MACHINE LEARNING NA PREVISÃO DE ÍNDICES ECONÔMICOS:
O IDHM COM O MODELO RANDOM FOREST DE 2012 À 2021. DOI:
10.13140/RG.2.2.36759.55204}
\author{Matheus Assis de Oliveira}
\date{2023-12-03}

\begin{document}
\maketitle

\begin{center}\rule{0.5\linewidth}{0.5pt}\end{center}

Tese disponível em:
\url{https://www.researchgate.net/publication/375992831_APLICACAO_DO_MACHINE_LEARNING_NA_PREVISAO_DE_INDICES_ECONOMICOS_O_IDHM_COM_O_MODELO_RANDOM_FOREST_DE_2012_A_2021}

\begin{enumerate}
\def\labelenumi{\arabic{enumi}.}
\tightlist
\item
  \textbf{INTRODUÇÃO}
\end{enumerate}

O ritmo acelerado do desenvolvimento tecnológico na era digital tem
permeado todas as esferas da sociedade, transformando setores inteiros
da economia, alterando padrões de comportamento social e abrindo novas
possibilidades para a pesquisa e o avanço do conhecimento.

Uma das áreas de destaque é a Inteligência Artificial (IA),
especialmente o ramo do Aprendizado de Máquina (Machine Learning - ML).
As técnicas de ML estão sendo cada vez mais utilizadas em uma variedade
de campos de estudo para encontrar padrões em dados complexos e fazer
previsões precisas.

No campo da economia, o ML vem ganhando relevância, visto que a
quantidade de artigos científicos e trabalhos acadêmicos acerca do
Aprendizado de Máquina na economia ainda é limitado, mas em crescimento.
Com sua aplicação variando desde a previsão de preços de ações, ou
indicadores e demais variáveis econômicas até a avaliação de políticas
econômicas.

Uma das principais economistas que lideraram essa área da econometria é
Susan Athey, cujos trabalhos sobre os impactos do ML na economia têm
atraído o interesse ao uso dessas técnicas em estudos econômicos. Seus
trabalhos, analisando o impacto do ML na economia (Athey, 2019) e os
métodos de aplicação do Machine Learning que auxiliariam economistas
(Athey, Imbens, 2019), servem como referências fundamentais,
proporcionando uma visão geral das técnicas de ML que são mais
relevantes para os economistas, e as vantagens de suas aplicações. No
artigo sobre o impacto do ML na economia Athey debate a dificuldade
inicial em se definir Machine Learning, segundo ela:

``O termo pode ser (e tem sido) usado de forma ampla ou restrita; pode
referir-se a uma coleção de subcampos da ciência da computação, mas
também a um conjunto de tópicos que são desenvolvidos e usados na
ciência da computação, na engenharia, na estatística e, cada vez mais,
nas ciências sociais''. Susan Athey (2019).

Os índices econômicos, ferramentas para a compreensão e avaliação de
economias em diferentes níveis, fornecem uma medida quantitativa de
vários aspectos econômicos, como crescimento, inflação, desemprego,
desenvolvimento humano, entre outros.

Dois índices usados no Brasil são o Índice de Desenvolvimento Humano
(IDH) e o Índice Firjan de Desenvolvimento Municipal (IFDM). O IDH,
desenvolvido pelo Programa das Nações Unidas para o Desenvolvimento
(PNUD), com publicação anual do Human Development Report (CONCEIÇÃO,
2022), (o mais recente de 2021/2022 dirigido por Pedro Conceição), mede
a média de três dimensões básicas do desenvolvimento humano: vida longa
e saudável, acesso ao conhecimento e um padrão de vida decente.

Já o IFDM, desenvolvido pela Firjan até o ano de 2018, é um indicador
brasileiro que mede a qualidade de vida em todos os municípios
brasileiros em três áreas: emprego e renda, educação e saúde conforme em
sua Metodologia (2018) disponível no site da Firjan.

A pesquisa em econometria aplicada tem explorado o uso de técnicas de ML
para aprimorar e expandir as possibilidades de análise dos índices
econômicos. Por exemplo, Luke Sherman (SHERMAN et al., 2023) usou
imagens de satélite e ML para estimar o IDH em todo o mundo com alta
resolução. Da mesma forma, Noor Ell Goldameir (GOLDAMEIR et al., 2021)
aplicaram o método Bootstrap Aggregating para classificar o IDH na
Indonésia. Tobaigy (Tobaigy et al.,2019) usaram técnicas de ML para
determinar e classificar os fatores significativos que afetam o IDH.

O artigo de Sendhil Mullainathan e Jann Spiess (MULLAINATHAN; SPIESS,
2017) enfatizam a integração do aprendizado de máquina com a
econometria, demonstrando como as técnicas de aprendizado de máquina
podem ser aplicadas para resolver problemas complexos em diversos
campos, incluindo economia, saúde e meio ambiente. A abordagem dos
autores explora a aplicação de algoritmos de aprendizado de máquina em
dados econômicos e políticos, um campo tradicionalmente dominado por
métodos estatísticos mais convencionais.

Um artigo publicado no Journal of Economic Perspectives oferece uma
visão sobre o uso de imagens de satélite para o estudo da economia
(DONALDSON; STOREYGARD, 2016) , a integração de dados de sensoriamento
remoto com a pesquisa econômica.

Outro artigo a ser considerado destaca a relevância do Machine Learning
na predição volatilidade do mercado de criptomoedas apresenta um desafio
intrigante para analistas e investidores (BHATTAD et al., 2023).

O artigo de Shubham Bhattad, Stefin Sunnymon, Dallas Vaz e Prof.~Chhaya
Dhavale, analisa de diversos modelos de aprendizado de máquina, com o
objetivo de identificar as técnicas mais eficazes para prever os preços
das criptomoedas. Esta abordagem é particularmente pertinente,
considerando a crescente popularidade e a natureza imprevisível das
criptomoedas destacado pelos autores.

A integração do Machine Learning no estudo econométrico representa a
introdução de uma nova técnica para buscar a compreensão e análise dos
fenômenos econômicos. Por meio de algoritmos de aprendizado de máquina,
com a possibilidade de explorar e capturar padrões complexos em grandes
conjuntos de dados econômicos, obtendo insights mais precisos e
robustos.

A modelagem de séries temporais econômicas, a seleção de variáveis
relevantes, detecção de outliers e a previsão de demanda são exemplos de
áreas em que o Machine Learning pode contribuir. No entanto, a aplicação
bem-sucedida do Machine Learning na econometria requer que haja,
combinação de conhecimentos em econometria e ciência de dados, bem como
uma interpretação consistente dos resultados à luz das teorias
econômicas subjacentes.

A uma análise das técnicas de aprendizado de máquina aplicadas na
previsão de preços de ações (SONI; TEWARI; KRISHNAN, 2022). O trabalho
destaca uma abordagem sistemática que explora um campo complexo e em
constante evolução. O artigo estabelece a importância da previsão do
mercado de ações, não apenas para investidores, mas também para
analistas de dados e profissionais nos campos de Machine Learning (ML) e
Inteligência Artificial (AI).

A previsão de preços de ações utilizando algoritmos de aprendizado
profundo ``Deep Learning'' (NIKOU; MANSOURFAR; BAGHERZADEH, 2019) e sua
comparação com outros métodos de aprendizado de máquina. O estudo de
Mahla Nikou, Gholamreza Mansourfar e Jamshid Bagherzadeh é relevante ao
estudo atual, em que demonstra a capacidade do aprendizado de máquina
prever tendências de mercado com precisão pode significar uma grande
vantagem competitiva.

Uma abordagem do ML à predição de preços de ações (SONI; TEWARI;
KRISHNAN, 2022) advindo da Universidade de Manitoba, Canadá, demonstra a
aplicação de técnicas de aprendizado de máquina no campo da inteligência
empresarial, com um foco especial na previsão de preços de ações. O
artigo propõe uma abordagem utilizando máquinas de vetores de suporte
estruturais (SSVMs) para prever movimentos de preços de ações.

Outros trabalhos internacionais também descrevem a eficiência do Machine
Learning em previsões, como a movimentos de preços de ações no setor de
energia alemão utilizando aprendizado de máquina (Vogt, 2021). A
pesquisa de Vogt utiliza modelos específicos de aprendizado de máquina,
incluindo Gradient Boosted Regression Tree, Multilayer Perceptron e Long
Short-Term Memory, para prever o preço de fechamento das ações no dia
seguinte. Esses modelos foram escolhidos por sua capacidade de capturar
complexidades nos dados e potencial para lidar com a natureza não linear
dos mercados financeiros.

Outro estudo concentrou na aplicação de técnicas de Machine Learning
para analisar o papel do capital humano nas startups. O estudo de Brigo
começa com a criação de um dataset robusto, selecionando aleatoriamente
8731 indivíduos e suas 4753 startups de um banco de dados inicial de
18.000 nomes e 10.203 empresas ( Brigo, 2019).

Uma análise sobre a previsão de preços de ações utilizando redes neurais
Long Short-Term Memory (LSTM) e Gated Recurrent Unit (GRU), concentrando
no índice S\&P 500, um dos mais importantes e representativos do mercado
de ações (PESCI, 2021). Contribuindo para o campo da análise financeira,
demonstrando o potencial das redes neurais LSTM e GRU em aplicações
práticas no mercado de ações.

Um estudo de Semyannikov Gleb Valeryevich (СЕМЯННИКОВ Г.В.) analisando o
desempenho de algoritmos de aprendizado de máquina em comparação com o
índice S\&P 500, em que resultados mostram que o algoritmo supera o
desempenho do índice S\&P 500 (СЕМЯННИКОВ, 2020). O processo de
implementação desses modelos é descrito, a partir organização de
conjuntos de dados, seguido pelo treinamento e ajuste dos algoritmos, e
finalmente testando-os contra conjuntos de dados reais. O artigo destaca
a eficácia desses modelos ao comparar seus resultados com o desempenho
real de ativos financeiros.

Outro estudo russo ressaltou a eficácia do aprendizado de máquina, e em
particular do modelo Facebook Prophet, na previsão de preços de ações
(Агнон Х.О.. 2021). A alta precisão alcançada pelo modelo demonstra seu
potencial como uma ferramenta valiosa para investidores e analistas de
mercado. Além disso, a criação da ferramenta ``Stockers'', que incorpora
o algoritmo FB Prophet, destaca a aplicabilidade prática dos resultados
do estudo no mundo real.

O objetivo deste estudo é demonstrar como utilizar um modelo simples de
predição com ML, aplicado no IDHM, um indicador conhecido,
principalmente no campo da economia regional, para demonstrar a prática
da aplicação de uma técnica de predição de ML no estudo econométrico.

Especificamente o objetivo é demonstrar como preparar, montar e aplicar
um modelo introdutório de Machine Learning Random Forest com regressões
ao IDHM e seus componentes na Linguagem de programação R, para fazer
predições do índice através de uma base de dados dívida em dados para
treino e teste do modelo, com a disposição dos códigos, para auxiliar
estudantes de economia a iniciando seus estudos de ML tenham um primeiro
de exemplo simples para aplicar.

A aplicação de ML à indicadores econômicos não se limita à predição, mas
também há possibilidade de obter insights valiosos sobre as
características dos índices e a maneira como eles interagem entre si.

Além disso, o uso de ML também pode ajudar a identificar fatores que são
particularmente importantes para o desenvolvimento econômico e que podem
ser alvo de políticas públicas para melhorar a qualidade de vida e o
bem-estar econômico.

O método proposto neste estudo consiste de maneira prática apenas em
aplicar o algoritmo de aprendizado de máquina Random Forest com
regressão, aos componentes do IDHM. O Random Forest é um algoritmo de
aprendizado supervisionado que é particularmente adequado para lidar com
conjuntos de dados complexos e multidimensionais.

Ele é capaz de capturar interações complexas entre variáveis e obter uma
medida de importância variável, evitando torná-lo muito complexo, ou
treiná-lo excessivamente até o ponto dele se ajustar aos dados
absorvendo até mesmo os ruídos. O modelo Random Forest é capaz de evitar
o Overfitting, o que o torna útil para entender quais componentes do
IDHM são os mais importantes para o desenvolvimento econômico.

No estudo, utilizou-se o modelo de florestas de árvores com entradas
aleatórias desenvolvido por Andy Liaw e Matthew Wiener (LIAW; WIENER,
2002), baseado no modelo de Breiman (BREIMAN, 2001), e importado para o
R.

Ao longo do trabalho, será examinada uma literatura relevante para o
contexto da pesquisa. Serão explorados trabalhos de economistas que
aplicaram técnicas de ML à economia, como Susan Athey, Guido Imbens e
Hal Varian. Também serão examinadas pesquisas que aplicaram ML à
indicadores de desenvolvimento, mais especificamente o IDH, e aplicações
do ML em previsões de preços do setor de energia alemão, criptomoedas e
preços de ações de empresas.

Espera-se com este trabalho destacar o potencial e a aplicabilidade do
aprendizado de máquina na economia, especificamente na criação e análise
de índices econômicos, contribuindo para o corpo de literatura que
destaca o uso do ML na economia e fornecendo um exemplo introdutório de
uma técnica para auxílio da análise e compreensão da compreensão acerca
do campo econômico.

A predição de um índice baseado em ML apresenta uma ferramenta para os
formuladores de políticas públicas. O campo de aplicação do ML na
economia está a amadurecer, apresentando novas técnicas e abordagens
desenvolvidas continuamente.

Portanto, embora este estudo utilize o algoritmo Random Forest, futuras
pesquisas podem explorar o uso de outros algoritmos e técnicas de ML
para analisar indicadores econômicos, e demais variáveis econômicas, bem
como contribuir com um novo ferramental para a econometria.

Por último, a pesquisa realizada neste estudo também visa contribuir
para a disseminação do conhecimento sobre o uso do ML na economia.
Através da apresentação clara e acessível da metodologia com os códigos
utilizados, estes sendo uma adaptação do desenvolvido por Julie Anne
Hockensmith (ANNE, 2020), no Github, buscando tornar a aplicação de ML
na economia mais compreensível e acessível.

Anne utilizou o pacote WDI (AREL-BUNDOCK; BACHER, 2022), com a versão
atual 2.7.8 de setembro de 2022 para obter dados do Banco Mundial, sobre
o IDH e demais indicadores, no desenvolvimento do modelo é feito uma
divisão de 90\%/10\% entre os dados para treino e teste, diferindo do
particionamento normalmente usados de 80\% para treino e 20\% para
teste.

Em resumo, este estudo se propõe a explorar o uso do aprendizado de
máquina na predição de um índice econômico, com o objetivo de servir
como um manual introdutório ao uso do Machine Learning, aplicando um
parâmetro conhecido no campo do Machine Learning, em um indicador
conhecido na área da economia e outros campos das ciências humanas como
o caso da geografia.

No decorrer deste trabalho, buscar-se-á contribuir para o campo da
economia, adicionando à literatura sobre a aplicação do ML na economia
um ferramental prático e valioso para os formuladores de políticas
públicas e profissionais da área econômica e afins como geografia e
estatística.

\begin{enumerate}
\def\labelenumi{\arabic{enumi}.}
\setcounter{enumi}{1}
\tightlist
\item
  \textbf{REVISÃO DE LITERATURA}
\end{enumerate}

A aplicação de técnicas de aprendizado de máquina no campo econômico tem
sido uma área atraente para acadêmicos e profissionais. Os métodos de
aprendizado de máquina, como Random Forest, são amplamente usados para
lidar com a complexidade e a multicolinearidade de conjuntos de dados de
alta dimensão, tornando-os úteis para a análise de indicadores
econômicos e de desenvolvimento.

A revisão da literatura concentra trabalhos anteriores que exploraram a
aplicação do aprendizado de máquina no estudo econômico, especialmente a
área da econometria, além dos estudos na predição de indicadores, nem
sempre econômicos, mas com sua possível sua utilização na área econômica
com sua metodologia. Através desta revisão, espera-se entender melhor o
estado atual do campo e identificar lacunas na literatura existente que
o trabalho poderá preencher.

Como notável exemplo do crescente uso do aprendizado de máquina na
economia, Hal Ronald Varian, economista-chefe da Google Inc., e
professor emérito de economia da University of California, Berkeley,
destacou a importância do Big Data e das novas técnicas econométricas em
seu artigo em que destaca o Big Data como um novo truque para
economistas (VARIAN, 2014).

Embora este trabalho não esteja diretamente relacionado à predição de
indicadores, ele ilustra a importância do aprendizado de máquina na
economia e a necessidade de os economistas se familiarizar com essas
técnicas de aprendizado de máquina.

Variam descreve na página 6 do artigo o funcionamento do Machine
Learning em que as variáveis x chamadas de ``preditoras'', como
``características'' em alguns casos, e o fim do ML é encontrar a função
que forneça uma boa previsão de ``y'' como função de ``x''. Ele define a
boa previsão como sendo a que minimiza o erro em uma função de perda.

Varian debate também o uso de ferramentas de banco de dados como o SQL,
``linguagem de consulta estruturada'', como ferramenta para manipulação
de Big Data, ferramentas para análise como o Google File System e o
Bigquery. Entrando no escopo do ML Hal Varian descreve o modelo de
Classificação e Random Forest, um exemplo na economia utilizando dados
de Xavier Sala-i-Martín (1997).

A literatura sobre a aplicação de técnicas de aprendizado de máquina na
economia está em seu nascituro nos últimos anos, contando com a
influência de Susan Athey, professora de Economia da Tecnologia na
Escola de Humanidades e Ciências da Stanford Graduate School of
Business, anterior a seu ingresso em Stanford lecionou na Harvard
University e no Massachusetts Institute of Technology (MIT), uma das
principais pesquisadoras e especialista na integração do ML no estudo da
ciência econômica.

Athey em seu artigo sobre os impactos do ML na economia publicado em
2018, discute os desafios e as oportunidades do uso do aprendizado de
máquina em pesquisas econômicas. Athey destaca que, embora a
aprendizagem de máquinas possa auxiliar na previsão e na identificação
de padrões complexos em grandes conjuntos de dados, também apresenta
desafios, como a tendência ao sobre ajuste e a dificuldade de
interpretação. Athey sugere que uma abordagem que combine métodos
econômicos e de aprendizado de máquina pode oferecer as melhores
oportunidades para avançar a pesquisa econômica.

Para Athey, ``o aprendizado de máquina é um campo que desenvolve
algoritmos projetados para serem aplicados a conjuntos de dados, com as
principais áreas de foco sendo previsão (regressão), classificação e
tarefas de agrupamento ou agrupamento.'' Athey destaca a divisão nos
dois ramos principais do ML, o supervisionado e o não supervisionado.

O ML não supervisionado envolvendo a procura por encontrar grupos de
observações semelhantes em suas covariáveis e a possibilidade de serem
interpretados como redução de dimensionalidade. Dentre as técnicas
destacam-se o agrupamento k-means, modelagem de tópicos, métodos de
detecção de comunidade para redes e entre outras técnicas.

O modelo não supervisionado detém de uma saída de modelo típico de ML
própria, conforme Athey descreve: ``\ldots é uma partição do conjunto de
observações, onde as observações dentro de cada elemento da partição são
semelhantes de acordo com alguma métrica; ou um vetor de probabilidades
ou pesos que descreve uma mistura de tópicos ou grupos aos quais uma
observação pode pertencer''. Susan Athey (2019).

A revisão de literatura irá explorar mais profundamente a aplicação do
aprendizado de máquina na análise de indicadores, revisando as
metodologias usadas e os resultados obtidos. Ao fazer isso, espera-se
compreender melhor o estado atual do campo das ferramentas econômicas e
identificar lacunas na literatura existente para o trabalho poder
preencher.

Guido Imbens, economista neerlandês - americano e professor de economia
da escola de Negócios da Universidade de Stanford e laureado do Prémio
de Ciências Económicas em Memória de Alfred Nobel em 2021, discutiu os
impactos e potencialidades do aprendizado de máquina na economia.

Athey e Imbens (2019), escrevem um artigo destacando métodos de
aplicação do ML em que economistas deveriam saber, também fornecem uma
visão abrangente dos métodos de aprendizado de máquina relevantes para
os economistas. Athey e Imbens exploram o modo como essas técnicas podem
ser usadas para abordar questões econômicas, incluindo aquelas
relacionadas ao desenvolvimento humano.

Este trabalho destaca o quão importante é entender e aplicar
corretamente as técnicas de aprendizado de máquina no estudo econômico,
e o fornecimento de contextos valiosos para o uso dos métodos na
previsão de variáveis econômicas. Outro trabalho, uma contribuição
notável vem de Maninder Kaur (KAUR et al., 2019). Em seu estudo, os
autores aplicam técnicas de aprendizado de máquina supervisionado para
prever o Índice de Qualidade de Vida de várias nações.

Os autores demonstram que os modelos de aprendizado de máquina, como a
regressão logística e a floresta aleatória, além dos modelos adicionais
para comparação como o modelo cubist, elastic net, redes neurais e
máquina de vetores de suporte. Os modelos foram comparados analisando o
R2, a raiz quadrática média dos erros RMSE e a precisão que os modelos
atingiram na análise, fornecendo assim ferramentas poderosas para
análise de políticas e planejamento de desenvolvimento.

No artigo de Kaur os modelos árvores de decisão, redes neurais, floresta
aleatória e máquinas de vetores de suporte apresentaram maior acurácia
dentre os modelos.

Outro estudo importante no campo é o de Ozden e Guleryuz (2021). No
artigo os autores aplicaram algoritmos de aprendizado de máquina
otimizados para analisar a relação entre o desenvolvimento econômico e o
capital humano. Este estudo destaca a habilidade dos algoritmos de
aprendizado de máquina em capturar relações complexas em conjuntos de
dados econômicos, fornecendo insights valiosos para o desenvolvimento de
políticas.

Sherman et al.~(2023) também contribuíram significativamente para este
campo de pesquisa em seu estudo empregam uma abordagem em certo grau
inovador combinando imagens de satélite e técnicas de aprendizado de
máquina para gerar estimativas de alta resolução do Índice de
Desenvolvimento Humano (IDH) das Nações Unidas. Esse estudo demonstra o
poder das técnicas de aprendizado de máquina quando combinadas com dados
espaciais complexos e grandes conjuntos de dados, abrindo novas
possibilidades para a análise e monitoramento do desenvolvimento humano
em nível global.

Tobaigy, Alamoudi e Bafail (2023) contribuíram com uma análise útil dos
principais fatores que determinam e classificam o Índice de
Desenvolvimento Humano (IDH) em seu artigo. Os autores realizaram uma
análise estatística aprofundada para identificar os principais
determinantes do IDH e classificá-los em termos de significância. Este
estudo fornece insights valiosos sobre as variáveis que podem ser mais
influentes na previsão do IDH e, portanto, podem ser úteis na construção
de modelos de aprendizado de máquina precisos para a previsão do IDH.
Este estudo fornece um ferramental poderoso para classificar os fatores
determinantes do indicador a ser construído com o IDH.

O artigo de Arumnisaa e Wijayanto (2023) contribui para a literatura do
ML na economia, ao comparar diferentes métodos de aprendizado de
máquina, especificamente Random Forest, Support Vector Machine (SVM) e
AdaBoost, na classificação do Índice de Desenvolvimento Humano (IDH).
Esta abordagem possui certa originalidade, uma vez que o IDH é um
indicador amplamente usado para medir o desenvolvimento socioeconômico
dos países e pode ter uma complexa relação multivariada que métodos de
aprendizado de máquina são bem equipados para explorar.

Os autores constatam que o uso de técnicas de aprendizado de máquina
para modelar e prever o IDH tem várias vantagens potenciais, incluindo a
capacidade de capturar relações não-lineares e complexas, além da
flexibilidade para lidar com grandes conjuntos de dados.

Essa descoberta é consistente com as pesquisas anteriores de Athey e
Imbens (2019), que destacam a utilidade do aprendizado de máquina na
economia. O artigo acrescenta a esta literatura, demonstrando a
aplicabilidade dessas técnicas ao IDH, um indicador econômico chave.

As pesquisas revisadas convergem na aplicação de métodos de aprendizado
de máquina (ML) para melhor entender e modelar índices econômicos e
sociais, comumente o IDH. O uso de técnicas de ML, particularmente o
Random Forest, Support Vector Machine (SVM) e AdaBoost, na modelagem de
indicadores de desenvolvimento tem provado ser vantajoso em fornecer
insights não lineares e complexos que não são tão facilmente capturados
por métodos estatísticos tradicionais.

Athey e Imbens (2019) forneceram uma revisão ampla sobre os métodos de
ML que os economistas deveriam conhecer, enquanto Sherman et al.~(2023)
e Arumnisaa e Wijayanto (2023) aplicaram essas técnicas para melhor
entender e estimar o IDH. Ozden e Guleryuz (2021) e Tobaigy et
al.~(2023) também contribuíram com pesquisas na exploração da relação
entre o desenvolvimento econômico e o capital humano, assim como a
determinação e classificação dos fatores significativos do IDH,
respectivamente.

Outro campo em que o ML tem sido explorado são as criptomoedas e o
mercado de ações, em que o ML treinado com modelos de series temporais
para previsão de preços tem gerado estudos acadêmicos na área. O artigo,
conduzido por Bhattad et al, mergulha na análise de diversos modelos de
aprendizado de máquina, com o objetivo de identificar as técnicas mais
eficazes para prever os preços das criptomoedas (BHATTAD et al., 2023).
Esta abordagem é pertinente, considerando a natureza das criptomoedas.

A metodologia adotada pelos autores desenvolve uma análise comparativa
de vários estudos, onde diferentes modelos de aprendizado de máquina
foram aplicados na previsão de preços de criptomoedas. Entre os modelos
analisados estão o Prophet, ARIMA, LSTM, XGBoost, SVM, Regressão
Logística e Naive Bayes. Esta variedade de modelos ofereceu uma visão
abrangente das capacidades de cada técnica no contexto do mercado de
criptomoedas.

Um aspecto crucial da predição de preços de ações com a abordagem do
Machine Learning (SONI; TEWARI; KRISHNAN, 2022) é a discussão sobre os
desafios enfrentados nas metodologias existentes. Este não apenas
identifica as limitações, mas também aponta direções para pesquisas
futuras, sugerindo a integração de diversas fontes de dados e a melhoria
da precisão e confiabilidade dos modelos preditivos.

Os autores não se limitaram a discutir técnicas tradicionais de
aprendizado de máquina, como Regressão Linear e SVM apenas, mas também
abordam métodos mais avançados, como Redes Neurais, AI e Aprendizado
Profundo. Esta seção fornece uma comparação entre diferentes abordagens,
destacando suas forças e limitações. O artigo conclui enfatizando a
necessidade contínua de abordagens destas técnicas na previsão de preços
de ações.

A importância dos índices de segurança como ferramentas principais para
avaliar o estado dos mercados financeiros, dada a natureza não linear e
não estacionária das séries financeiras, a previsão se torna um desafio
complexo (NIKOU; MANSOURFAR; BAGHERZADEH, 2019). O estudo de Nikou et
al, utiliza dados diários do preço de fechamento do fundo negociado em
bolsa iShares MSCI United Kingdom, de janeiro de 2015 a junho de 2018,
para testar a eficácia de quatro modelos de algoritmos de aprendizado de
máquina na previsão de preços de ações.

Os resultados indicam que o método de aprendizado profundo supera os
outros métodos em termos de precisão na previsão, com a regressão
vetorial de suporte (support vector regression - SVR) classificando-se
em segundo lugar, seguida pela rede neural e pelo método de floresta
aleatória, ambos com menor erro. Esses achados são significativos, pois
demonstram a superioridade do aprendizado profundo sobre métodos
tradicionais em um campo tão volátil e imprevisível quanto o mercado de
ações.

O estudo utiliza uma abordagem rigorosa para testar e comparar os
diferentes modelos. A escolha dos dados, abrangendo um período
significativo e focando em um fundo específico, fornece uma base sólida
para a análise. Além disso, a utilização de critérios de avaliação de
erro, como o erro médio absoluto (MAE), o erro quadrático médio (MSE) e
o erro quadrático médio da raiz (RMSE), adiciona credibilidade aos
resultados.

Outro estudo propõem uma abordagem utilizando máquinas de vetores de
suporte estruturais (SSVMs) para prever movimentos de preços de ações
(SONI; TEWARI; KRISHNAN, 2022). A abordagem tem a capacidade de
classificar entradas complexas, como nós em uma estrutura de gráfico,
que conectam empresas no setor de tecnologia da informação.

Os resultados experimentais apresentados no artigo são particularmente
impressionantes. Os autores testaram seu modelo em um conjunto de dados
do setor de Tecnologia da Informação do Índice S\&P 500, utilizando
dados financeiros e movimentos de preços de ações. A precisão do SSVM em
amostras de treinamento foi superior a 78\%, o que indica uma
aprendizagem bem-sucedida do modelo sem superajuste. Este nível de
precisão é promissor, sugerindo que a abordagem pode ser uma ferramenta
valiosa para previsões no mercado de ações.

Um artigo publicado no Science Journal da AAAS, apresenta uma proposta
para estimar a pobreza em países em desenvolvimento (JEAN et al., 2016).
O estudo utiliza dados de pesquisa e imagens de satélite de cinco países
africanos - Nigéria, Tanzânia, Uganda, Malawi e Ruanda. Os autores
treinaram uma rede neural convolucional para identificar características
de imagem que explicam até 75\% da variação nos resultados econômicos
locais.

O artigo também demonstra como técnicas avançadas de aprendizado de
máquina podem ser aplicadas em um contexto com dados de treinamento
limitados, sugerindo uma ampla aplicação potencial em muitos domínios
científicos.

O artigo aborda a escassez de dados confiáveis sobre meios de
subsistência econômicos no mundo em desenvolvimento, o que dificulta o
estudo desses resultados e a elaboração de políticas para melhorá-los.
Os autores destacam que, embora a quantidade e a qualidade dos dados
econômicos em países em desenvolvimento tenham melhorado nos últimos
anos, ainda há uma lacuna significativa de dados, especialmente na
África.

Na literatura internacional o Machine Learning já fora explorado na
previsão de preços de produtos, ações, criptomoedas e a avaliação do
capital humano na probabilidade de sucesso de uma startup, neste quadro
pesquisadores Alemão, Italianos e russos já desenvolveram pesquisas com
Machine Learning no campo econômico.

Jan Vogt utiliza modelos específicos de aprendizado de máquina,
incluindo Gradient Boosted Regression Tree, Multilayer Perceptron e Long
Short-Term Memory, para prever o preço de fechamento das ações no dia
seguinte (VOGT, J. 2021). Esses modelos são escolhidos por sua
capacidade de capturar complexidades nos dados e potencial para lidar
com a natureza não linear dos mercados financeiros.

No entanto, os resultados do estudo indicam que, embora seja possível
alcançar desvios na previsão de preços de ações de cerca de 1,05\%, a
análise de sentimentos das mídias sociais não apresentou um efeito
significativo na redução do erro de previsão. Isso sugere que, apesar do
potencial das mídias sociais em refletir o sentimento do mercado, sua
aplicabilidade direta na previsão de preços de ações, especialmente no
setor de energia, pode ser limitada.

A eficácia dos modelos tratados no estudo de Vogt é, portanto, uma
questão complexa. Embora os modelos de aprendizado de máquina
demonstrassem competência em prever movimentos de preços com uma margem
de erro relativamente baixa, a integração da análise de sentimentos não
melhorou significativamente a precisão das previsões.

O estudo de Francesco Brigo começou com a criação de um dataset robusto,
selecionando aleatoriamente 8731 indivíduos e suas 4753 startups de um
banco de dados inicial de 18.000 nomes e 10.203 empresas (BRIGO, F.
2019). A fase de `feature engineering' fora meticulosa, analisando
parâmetros como experiência de trabalho anterior, formação acadêmica e
histórico de fundação de empresas. Essa abordagem multidimensional
permitiu uma análise aprofundada do impacto do capital humano no sucesso
das startups.

Os modelos de Machine Learning utilizados no estudo incluem regressão
logística, árvores de decisão, Random Forest, k-Nearest Neighbors e
redes neurais. O estudo revelou que o modelo Random Forest se destaca em
termos de eficácia, fornecendo insights significativos sobre as
características mais influentes para o sucesso de uma startup.

A eficiência dos modelos foi avaliada com base em sua capacidade de
prever corretamente o sucesso das startups. O modelo Random Forest, em
particular, demonstrou capacidade robusta de identificar as
características mais críticas que contribuem para o sucesso de uma
startup. Isso não apenas valida a importância do capital humano como um
fator crítico, mas também fornece uma ferramenta valiosa para
investidores, mentores e fundadores de startups para avaliar e melhorar
suas equipes.

Paolo Pesci desenvolveu três tipos de redes neurais em sua tese: LSTM,
GRU e uma rede híbrida combinando LSTM e GRU, para avaliar qual delas
seria mais eficaz na redução do erro de previsão em relação aos valores
reais do preço de fechamento diário do índice e na previsão de preços
futuros (PESCI, P. 2021). Os modelos foram testados com diferentes
configurações de parâmetros de treinamento, como número de épocas e
tamanho do lote (batch size), para otimizar seu desempenho.

Os resultados mostraram que o modelo GRU necessitou de um menor número
de épocas para ser treinado e apresentou um erro de aproximação muito
pequeno. No entanto, foi o modelo híbrido LSTM-GRU que obteve o menor
erro médio de previsão, sugerindo que a combinação desses dois modelos
pode levar a previsões futuras mais precisas. Especificamente, o modelo
híbrido apresentou um erro médio de previsão de 72,43 dólares, em
comparação com 144,70 dólares para o modelo GRU e 176,22 dólares para o
modelo LSTM.

Pesci concluiu que, embora o modelo híbrido tenha se mostrado mais
eficaz, variações bruscas no preço das ações (como as causadas por
guerras, pandemias, etc.) podem diminuir a precisão das previsões. Essa
conclusão destaca a complexidade inerente à previsão de preços de ações
e a necessidade de considerar uma variedade de fatores externos ao
desenvolver modelos de previsão.

Semyannikov discute o conceito de ``ensemble learning'', uma técnica que
combina as forças de múltiplos algoritmos para aumentar a precisão das
previsões (СЕМЯННИКОВ, 2020). Esta abordagem é benéfica porque permite
que o treinamento e o algoritmo de aprendizado ocorram de forma
independente, com as previsões sendo combinadas para uma análise mais
robusta.

Os resultados numéricos apresentados no artigo são detalhados, em
relação à eficácia dos algoritmos de previsão. Por exemplo, o autor
menciona que a precisão dos algoritmos de aprendizado de máquina pode
chegar a 77,6\%. Além disso, o artigo destaca a eficácia do uso de
mineração de texto para analisar fluxos de notícias e avaliar o
sentimento do mercado, o que pode levar a previsões eficazes de
movimentos de preços de ações.

O artigo também apresenta uma tabela detalhada com as estimativas de
desempenho para os top 50 sinais previstos. Estes incluem: Top 30
sinais: Variações de 0,21\% a 19,25\% ao longo de diferentes períodos
(de três dias a um ano); Top 20 sinais: Variações de 0,25\% a 19,63\%;
Top 10 sinais: Variações de 0,22\% a 21,95\%; Top 5 sinais: Variações de
0,24\% a 22,78\%.

O modelo FB Prophet, desenvolvido pelo Facebook, é um exemplo projetado
para lidar com séries temporais e reconhecido por sua facilidade de uso
e precisão em previsões (АГНОН Х.О.. 2021). Este modelo é capaz de lidar
com tendências e padrões sazonais. No estudo em questão, os dados das
ações da Amazon foram analisados usando o FB Prophet. Os dados incluíram
variáveis como preço de abertura, fechamento, mínimo, máximo e volume,
obtidos do Yahoo Finance.

O modelo foi treinado com dados dos últimos três anos, permitindo que
identificasse tendências e padrões relevantes. A precisão do modelo foi
notavelmente alta, atingindo 95\%, o que indica uma confiabilidade
significativa nas previsões geradas.

Além disso, o estudo realizou previsões para os próximos 50 dias com um
intervalo de confiança de 80\%. Isso sugere que o modelo tem uma forte
capacidade de prever com precisão os movimentos futuros dos preços das
ações dentro de um intervalo razoável. As visualizações fornecidas,
incluindo gráficos históricos e modelados, bem como gráficos de
previsão, oferecem uma representação clara das tendências e previsões.

A literatura indica um crescente reconhecimento da capacidade dos
métodos de ML de capturar relações complexas e de proporcionar uma
análise mais profunda de variáveis estudadas por economistas como: os
índices econômicos e sociais no escopo do desenvolvimento e economia
regional, preços de ações de empresas, commodities e criptomoedas.
Contudo, a escolha do método apropriado depende muito das
características dos dados e do contexto do problema a ser analisado pelo
pesquisador.

Em conclusão, a aplicação de ML nos campos da economia e do
desenvolvimento humano é uma área promissora e em rápido crescimento. A
combinação de uma maior disponibilidade de dados e o avanço das técnicas
de ML permitirá um maior entendimento e previsão de importantes
indicadores de desenvolvimento, como o IDH. No entanto, mais pesquisas
são necessárias para aprimorar a eficácia desses modelos e explorar sua
aplicabilidade em outras áreas do desenvolvimento econômico e social.

\begin{enumerate}
\def\labelenumi{\arabic{enumi}.}
\setcounter{enumi}{2}
\tightlist
\item
  \textbf{METODOLOGIA}
\end{enumerate}

O Índice de Desenvolvimento Humano (IDH), estabelecido em 1990 pelo
Programa das Nações Unidas para o Desenvolvimento (PNUD), criado por
Mahbub ul Haq com a colaboração do economista indiano Amartya Sen,
publicado no relatório do desenvolvimento humano de 1990 (``HUMAN
DEVELOPMENT REPORT 1990'', 1990), marcou a compreensão e mensuração do
desenvolvimento humano.

Ele se destaca não por abrangência, uma vez que ele não abrange todos os
aspectos do desenvolvimento, mas três dimensões críticas: saúde,
educação e renda. Outro aspecto que o destaca é sua habilidade de
apresentar uma medida simples de fácil compreensão acerca do
desenvolvimento humano.

No contexto brasileiro, o IDHM tornou-se uma adaptação direta desse
índice global, servindo como uma ferramenta para monitorar avanços a
nível municipal, e estadual no desenvolvimento brasileiro.

Neste panorama, a metodologia adotada neste estudo denota a capacidade
do Machine Learning (ML) na predição de indicadores, sublinhando sua
relevância no campo da economia. A abordagem concentra-se no emprego de
modelos baseados em árvores de decisão, especificamente florestas
aleatórias, utilizando um conjunto de dados originário do Atlas do
Desenvolvimento Humano no Brasil.

O processo de seleção dos dados está alinhado à metodologia delineada no
próprio Atlas e é enriquecido por indicadores macroeconômicos globais,
como população, PIB, proporção de indivíduos abaixo da linha de pobreza
e taxas de mortalidade infantil. A introdução dos índices GINI e Theil e
a população total foi optado por não os efetivar no modelo final, devido
ao uso do IDHM não ajustado, que leva a baixa relação entre os índices e
o IDHM. O núcleo da análise visa ilustrar a precisão e robustez das
técnicas de ML na estimação do IDHM.

O Índice de Desenvolvimento Humano (IDH) é uma métrica reconhecida no
cenário acadêmico, e procura traduzir o progresso humano em simples
indicadores econômicos, para procurar um entendimento de suas dimensões.
Em sua composição, o IDH é constituído por três indicadores
fundamentais, que juntos capturam a amplitude e a profundidade das
oportunidades e capacidades humanas. Seu cálculo pode ser encontrado nas
notas técnicas do IDH publicado no site da PNUD (``Calculating the human
development indices-graphical presentation Inequality-adjusted Human
Development Index (IHDI) Knowledge Human Development Index (HDI) Long
and healthy life A decent standard of living Human Development Index
(HDI) Knowledge Long and healthy life A decent standard of living
Inequality-adjusted Human Development Index (IHDI) Health Education'',
{[}s.d.{]}).

O primeiro desses indicadores é a longevidade, representada pela
dimensão ``Vida Longa e Saudável''. No contexto do desenvolvimento
humano, não se trata apenas de prolongar a vida, mas de garantir sua
qualidade. A essência desta dimensão reside no entendimento de que uma
vida plena é a que as pessoas têm as oportunidades necessárias para
evitar mortes prematuras e onde se garante um acesso equitativo a
serviços de saúde de qualidade.

A dimensão ``Acesso ao Conhecimento'' notabiliza a relevância da
educação no desenvolvimento humano. Este indicador concentra a
transmissão de informações na formação dos indivíduos, possibilitando
que eles exerçam suas liberdades, autonomia e autoestima. A educação é
traduzida não apenas como direito, mas como uma ferramenta capacitadora
das pessoas na tomada de decisões sobre seu futuro.

Por fim, a dimensão ``Padrão de Vida'' centra-se na renda como um meio
para atingir a realização plena das capacidades humanas. Embora a renda
seja fundamental para satisfazer necessidades básicas como alimentação,
água e moradia, principalmente o potencial para permitir que os
indivíduos transcendem as necessidades básicas. No entanto, é crucial
reconhecer que a ausência de renda adequada pode limitar drasticamente
as oportunidades de vida, restringindo o acesso a bens e serviços
essenciais e, por extensão, as próprias liberdades humanas.

\textbf{3.1.ADAPTAÇÃO DO IDHM: UMA VISÃO HISTÓRICA E METODOLÓGICA}

O Brasil, ao longo das últimas décadas, tem demonstrado um compromisso
contínuo com a avaliação e melhoria do desenvolvimento humano em sua
dimensão territorial mais ampla. Este compromisso é evidente na maneira
como o país adaptou o Índice de Desenvolvimento Humano (IDH) global ao
seu contexto municipal/estadual detalhado no livro ``O Índice de
Desenvolvimento Humano Municipal Brasileiro (UNITED NATIONS DEVELOPMENT
PROGRAMME; FUNDAÇÃO JOÃO PINHEIRO; INSTITUTO DE PESQUISA ECONÔMICA
APLICADA, 2013).

Em 1998, o Brasil foi além da utilização do IDH e adaptou para avaliar o
desenvolvimento humano em todos os seus municípios. Esta adaptação gerou
o Índice de Desenvolvimento Humano Municipal (IDHM), fundamentado em
dados do Censo Demográfico.

O compromisso com esta avaliação detalhada não foi um feito isolado,
como evidenciado pela edição subsequente de 2003, que forneceu dados
históricos para os anos de 1991 e 2000, proporcionando uma análise
comparativa ao longo do tempo.

Em 2013 marcou outro momento, em colaboração, o PNUD Brasil, o IPEA e a
Fundação João Pinheiro se propuseram a tarefa de adaptar a metodologia
do IDH global para calcular o IDHM para os 5.565 municípios brasileiros
com base nos dados do Censo Demográfico de 2010.

Esta iniciativa também envolveu a reavaliação do IDHM para os anos de
1991 e 2000, utilizando a nova metodologia. Foi dada uma atenção à
compatibilização das áreas municipais ao longo destes anos, considerando
as alterações administrativas e assegurando a comparabilidade tanto
temporal quanto espacial entre os municípios.

Em essência, o IDHM mantém a tríade fundamental do IDH global,
abrangendo as dimensões de saúde, educação e renda. No entanto,
evidencia uma adaptação ao cenário brasileiro. A metodologia foi
ajustada para refletir não apenas o contexto nacional, mas também a
disponibilidade e relevância de indicadores específicos. Esta
particularização garante que, embora os indicadores do IDHM e IDH possam
medir fenômenos semelhantes, aqueles incorporados ao IDHM são mais aptos
a avaliar, de maneira precisa, o desenvolvimento dos municípios
brasileiros.

\textbf{3.1.1. ADAPTAÇÃO E METODOLOGIA DO IDHM}

O Índice de Desenvolvimento Humano Municipal (IDHM), embora enraizado no
conceito global do Índice de Desenvolvimento Humano (IDH), foi
meticulosamente adaptado para refletir as singularidades do contexto
brasileiro.

Principalmente, essa adaptação buscou dois objetivos centrais: (1)
incorporar indicadores que capturassem de forma mais precisa as
condições dos municípios, entidades menores no tecido social brasileiro,
e (2) garantir uma uniformidade e comparabilidade de dados, optando por
utilizar exclusivamente os indicadores disponíveis nos Censos
Demográficos do país.

A metodologia para o IDHM pode ser conferida tanto no livro ``O Índice
de Desenvolvimento Humano Municipal brasileiro, como na metodologia
publicada pelo Atlas Brasil (''Atlas Brasil'', {[}s.d.{]}).

Dentro dessa estrutura, o IDHM se desdobra em três dimensões principais:
Longevidade, Educação e Renda. A longevidade é calculada com base na
expectativa de vida ao nascer, enquanto a dimensão educacional combina
indicadores de escolaridade adulta e fluxo escolar juvenil. Por fim, a
renda é representada pela renda per capita do município, oferecendo uma
panorâmica do poder aquisitivo médio dos residentes. Essas métricas, em
conjunto, oferecem uma visão holística e contextualizada do
desenvolvimento humano a nível municipal, proporcionando insights
valiosos para políticas públicas e planejamento estratégico.

O IDHM (Índice de Desenvolvimento Humano Municipal) é uma ferramenta que
busca avaliar o nível de desenvolvimento humano dos municípios do
Brasil, adaptando-se à realidade local a partir dos conceitos usados
globalmente pelo IDH. Este índice é estruturado em três dimensões
fundamentais: Longevidade, Educação e Renda. Na dimensão da Longevidade,
o foco é a expectativa de vida ao nascer. Esta métrica é considerada um
indicador primordial de longevidade, pois representa o número médio de
anos que se espera que um indivíduo viva no momento do nascimento,
considerando os padrões de mortalidade atuais.

Para calcular o indicador de Longevidade, são necessários métodos
indiretos, pois os dados demográficos diretos não estão imediatamente
disponíveis. O método adotado para tal é o desenvolvido por William
Brass (1968), que foi adaptado pelo Professor José Alberto Magno de
Carvalho, da Universidade Federal de Minas Gerais (UFMG), para se
adequar a regiões com baixo volume populacional, como é o caso de muitos
municípios brasileiros. Para a dimensão Longevidade, o IDHM e o IDH
Global se alinham na adoção da esperança de vida ao nascer como
indicador chave. Esta métrica é derivada de informações dos Censos
Demográficos e da PNAD Contínua do IBGE. Para transformar o indicador em
um Índice de Longevidade, são adotados valores máximos e mínimos (85 e
25 anos, respectivamente) para normalizar os resultados. Assim, o IDHM
fornece uma visão abrangente e contextualizada da saúde e longevidade em
municípios específicos, o que pode informar a tomada de decisões e o
planejamento de políticas públicas.

Dentro da metodologia do Índice de Desenvolvimento Humano Municipal
(IDHM), a dimensão da Longevidade ocupa um espaço de extrema
importância. Esta dimensão busca avaliar a longevidade da população,
considerando-a uma das três componentes essenciais para aferir o
desenvolvimento humano, juntamente com a Renda e a Educação. O principal
indicador utilizado para essa dimensão é a esperança de vida ao nascer.

A esperança de vida ao nascer, que compõem a dimensão Longevidade do
IDHM, indica o número médio de anos que uma pessoa nascida em
determinado município ou unidade de análise viveria, se prevalecesse os
mesmos padrões de mortalidade observados no período em que o dado foi
coletado. Esse indicador é uma síntese das condições de saúde, sociais e
de salubridade de uma população, visto que incorpora taxas de
mortalidade para todas as faixas etárias, contemplando tanto as doenças
quanto causas externas, como violência e acidentes.

O cálculo desse indicador no IDHM não é direto. Não se obtém
simplesmente a partir das informações dos Censos Demográficos. Em vez
disso, recorre-se a técnicas indiretas, como os métodos desenvolvidos
por William Brass. Tais técnicas, para serem precisas, necessitam de
ajustes, sobretudo quando aplicadas em regiões com populações menores.
No contexto brasileiro, a adaptação dessas técnicas foi executada pelo
Professor José Alberto Magno de Carvalho, do CEDEPLAR/UFMG.

Para que o indicador de esperança de vida ao nascer seja transformado em
Índice de Longevidade, é preciso utilizar uma normalização, adotando-se
parâmetros de valores máximos e mínimos. O IDHM estabelece que o valor
máximo é 85 anos e o mínimo é 25 anos. Portanto, para calcular o Índice
de Longevidade de uma dada região, usa-se a fórmula

I = valor observado-valor mínimo / valor máximo-mínimo

Em resumo, a dimensão Longevidade do IDHM é fundamental para entender o
nível de desenvolvimento humano de uma região, oferecendo uma visão
compreensiva sobre a saúde e as condições de vida da população. Os
métodos e técnicas de cálculo utilizados para este fim garantem uma
avaliação robusta e confiável dessa importante dimensão.

A Dimensão Educação do IDHM é dividida em dois indicadores principais:
escolaridade da população adulta e fluxo escolar da população jovem. A
escolaridade da população adulta foca no percentual de indivíduos com 18
anos ou mais que concluíram o ensino fundamental. Esta métrica reflete o
funcionamento do sistema educacional em períodos anteriores, supondo que
a população adulta, ao passar pelo sistema educacional, deveria ter
completado pelo menos o ensino fundamental.

Por outro lado, o indicador do fluxo escolar da população jovem monitora
a progressão da população em idade escolar em quatro fases cruciais: a
admissão ao sistema educacional, a conclusão do primeiro ciclo do ensino
fundamental (focando apenas no ensino regular), a conclusão do ensino
fundamental e a conclusão do ensino médio. O design destes indicadores
permite avaliar se os estudantes estão progredindo de acordo com sua
faixa etária, alinhando-se às expectativas padrão para sua idade.

Entretanto, é fundamental notar que existem limitações nestes
indicadores. O foco em certas faixas etárias e estágios da educação pode
não capturar toda a imagem da trajetória educacional de um indivíduo.
Por exemplo, a métrica da educação da população jovem não inclui toda a
população em idade escolar e enfatiza apenas certos pontos de sua
jornada educacional.

Além dos indicadores principais, o Atlas Brasil apresenta outros
indicadores complementares para proporcionar uma visão mais holística e
detalhada da situação educacional em determinada área. Tais indicadores
abordam questões como analfabetismo, graus variados de escolaridade e
frequência escolar.

A dimensão Educação do IDHM é calculada com base nas respostas
fornecidas pelo Censo Demográfico e pela Pesquisa Nacional por Amostra
de Domicílio (PNAD Contínua) conduzidos pelo IBGE. Esses dados são
convertidos em índices que variam de 0 a 1, permitindo que diferentes
localidades sejam comparadas em relação ao seu desempenho educacional.
Por exemplo, se um município apresenta 65\% de sua população adulta com
ensino fundamental completo, seu índice de escolaridade será 0,650.

Em resumo, a dimensão Educação do IDHM oferece uma perspectiva crucial
sobre o estado da educação em diferentes partes do Brasil, permitindo
que políticas públicas sejam formuladas para abordar áreas específicas
de preocupação e melhorar o acesso e a qualidade da educação em todo o
país.

A dimensão ``Renda'' no Índice de Desenvolvimento Humano Municipal
(IDHM) avalia o padrão de vida da população, sendo representada pela
renda mensal per capita. Conforme a metodologia, a renda média mensal de
indivíduos em específicas regiões - como municípios, estados, regiões
metropolitanas ou Unidades de Desenvolvimento Humano (UDH) - é expressa
em reais e tem como referência valores de 1º de agosto de 2010. A
principal métrica deste indicador, a renda per capita, busca mensurar a
média de capacidade dos indivíduos em adquirir bens e serviços no local
de referência, sendo um reflexo direto da habilidade dos habitantes em
assegurar suas necessidades vitais, tais como água potável, alimento e
moradia.

A obtenção de tais valores se dá através dos resultados advindos dos
questionários de amostra do Censo Demográfico e da Pesquisa Nacional por
Amostra de Domicílio (PNAD) Contínua, para os períodos intercensitários.
Esse indicador é calculado pela divisão do total de rendimentos de todos
os residentes no local de referência pelo número total desses
indivíduos. Vale destacar que os valores obtidos nos Censos Demográficos
de 1991 e 2000 foram ajustados para reais constantes de 1º de agosto de
2010. Para isso, empregou-se a série do Índice Nacional de Preços ao
Consumidor (INPC) do IBGE, considerando, ainda, um ajuste relacionado à
inflação em julho de 1994.

Contudo, uma ressalva importante é que este indicador não contempla a
desigualdade de renda entre os residentes. Ou seja, mesmo que uma
localidade apresente alta renda per capita, pode existir uma porção
significativa de sua população em condição de pobreza. O cálculo
específico da dimensão Renda no IDHM é efetuado através da fórmula que
envolve o logaritmo natural da renda per capita do local em relação a
valores referenciais pré-estabelecidos. A aplicação do logaritmo tem o
objetivo de amenizar as disparidades entre as rendas mais altas e mais
baixas, levando em conta que aumentos na renda per capita proporcionam
menores retornos em termos de desenvolvimento humano à medida que a
renda cresce.

O desenvolvimento da metodologia da pesquisa é fundamental para qualquer
estudo que envolva dados e suas respectivas análises. A maneira como
coletado, tratado, analisado e interpretado os dados têm impacto
significativo nas conclusões, a referência das técnicas de ciência de
dados usados no estudo é Introduction to Data Science (IRIZARRY, 2019).

O professor Ph.D. Rafael Irizarry professor de bioestatística e biologia
computacional no Dana-Farber Cancer Institute e Harvard University em
seu livro ``Introduction to Data Science'', o professor Rafael Irizarry,
estabelece um manual estruturado da abordagem de projetos de ciência de
dados, que alinham a metodologia adotada neste estudo.

No livro o professor Irizarry descreve os fundamentos da ciência de
dados, desde a escolha do software, no caso a linguagem de programação
R, os processos de visualização passando pela análise exploratória de
dados, processos estatísticos de inferência, e a construção de modelos
de Machine learning.

\textbf{3.2. COLETA E PRÉ-PROCESSAMENTO DE DADOS}

A coleta e o pré-processamento dos dados são etapas fundamentais no
início de qualquer investigação científica. No contexto do presente
estudo, que se centra na análise do Indicador de Desenvolvimento Humano
Municipal (IDHM), esses passos ocorrem de maneira metódica e detalhada.

Inicialmente, os dados relativos ao IDHM e demais indicadores foram
coletados do portal do Atlas Brasil, que é um repositório confiável e
robusto de dados socioeconômicos abrangendo todos os municípios do
Brasil e demais Unidades Federativas e UDHs. O portal disponibiliza
esses dados no formato de arquivos Excel, XLSX, facilitando a importação
para as ferramentas de análise de dados a serem utilizadas
posteriormente.

Para a demonstração foram coletados os do IDHM do Brasil e todos os
estados contendo o Distrito Federal também, organizados em suas
territorialidades, entre os anos de 2012 a 2012. Para uso da plataforma
Rstudio, os anos foram renomeados com um X no início do ano, passando a
ser X2012 à X2021.

\includegraphics{images/Captura de Tela (121)-01.png}

Também foram coletados dados referentes a taxas de analfabetismo
referentes a 15 anos ou mais de idade, 18 anos ou mais e 25 anos ou
mais, o IDHM educação junto com os subíndices de escolaridade e
frequência de escolaridade da população utilizados na construção do IDHM
educação.

O IDHM longevidade foi coletado junto com os dados de esperança de vida
e mortalidade infantil. O IDHM renda foi coletado como também os dados
de Renda per capita e porcentagem de pobres. Para finalizar a coleta de
dados indicadores como população total, índice GINI e o índice Theil L
foram adicionados também para ampliar a abrangência de indicadores do
processo.

\begin{enumerate}
\def\labelenumi{\arabic{enumi}.}
\setcounter{enumi}{3}
\tightlist
\item
  \textbf{DESENVOLVIMENTO}
\end{enumerate}

Após a coleta dos dados, realizou-se o processo de engenharia dos dados,
que garantiu a limpeza e tratamento para auferir a qualidade e a
confiabilidade dos resultados subsequentes. Nesta fase, identifica-se e
corrige eventuais inconsistências nos dados, como valores ausentes,
erros e outliers. A estratégia de tratamento dos valores ausentes pode
variar dependendo do contexto e da natureza do dado, podendo incluir a
substituição pelo valor médio, mediano e modal do respectivo atributo.
Os outliers são identificados por meio de técnicas estatísticas, como o
método IQR (Intervalo entre quartis) ou Z-Score, e tratados de acordo
com as necessidades do estudo.

Dependendo da técnica de análise a ser empregada, os dados do IDHM podem
passar por um processo de transformação. Essa transformação pode
envolver a normalização ou padronização dos dados, de modo a garantir
que todos os atributos estejam na mesma escala, o que torna a comparação
entre diferentes variáveis mais consistente e significativa.

Antes de começar a engenharia de dados os pacotes: ``tidyr'', ``plyr'' e
``readxl'' foram instalados com suas dependências e carregados com a
função library.

O primeiro passo da engenharia de dados fora tratar os dados para o
formato a ser utilizado na confecção do modelo, primeiramente importou
os dados para o Rstudio, e aplicou a função ``gather'' para converter os
anos em uma coluna, tendo três colunas com a Territorialidade, o Estado,
o ano indo de 2012 a 2021, e a variável da tabela, como a renda per
capita no código. Em seguida a função ``gsub'' limpou o X à frente dos
anos, e por fim converteu a variável ano em numérico (Quadro 1). O
processo repetiu todas as variáveis utilizadas no modelo.

carregar pacotes para a Engenharia de dados.

\begin{Shaded}
\begin{Highlighting}[]
\NormalTok{renda\_per\_capita }\OtherTok{\textless{}{-}} \FunctionTok{read\_excel}\NormalTok{(}\StringTok{"renda.per.capita.xlsx"}\NormalTok{)}
\CommentTok{\#usar gather para converter anos à coluna}
\NormalTok{renda\_per\_capita }\OtherTok{\textless{}{-}} \FunctionTok{gather}\NormalTok{(renda\_per\_capita, ano, renda\_per\_capita, X2012}\SpecialCharTok{:}\NormalTok{X2021, }\AttributeTok{convert =} \ConstantTok{TRUE}\NormalTok{)}
\CommentTok{\#Limpar o X a frente do ano}
\NormalTok{renda\_per\_capita}\SpecialCharTok{$}\NormalTok{ano }\OtherTok{\textless{}{-}} \FunctionTok{gsub}\NormalTok{(}\StringTok{\textquotesingle{}X\textquotesingle{}}\NormalTok{, }\StringTok{\textquotesingle{}\textquotesingle{}}\NormalTok{, renda\_per\_capita}\SpecialCharTok{$}\NormalTok{ano)}
\CommentTok{\#converter ano de character para numeric}
\NormalTok{renda\_per\_capita }\OtherTok{\textless{}{-}} \FunctionTok{transform}\NormalTok{(renda\_per\_capita, }\AttributeTok{ano =} \FunctionTok{as.numeric}\NormalTok{(ano))}
\NormalTok{renda\_per\_capita }\OtherTok{\textless{}{-}} \FunctionTok{transform}\NormalTok{(renda\_per\_capita, }\AttributeTok{renda\_per\_capita =} \FunctionTok{as.numeric}\NormalTok{(renda\_per\_capita))}


\NormalTok{sub\_esco\_pop }\OtherTok{\textless{}{-}} \FunctionTok{read\_excel}\NormalTok{(}\StringTok{"sub.esco.pop.xlsx"}\NormalTok{)}
\NormalTok{sub\_esco\_pop }\OtherTok{\textless{}{-}} \FunctionTok{gather}\NormalTok{(sub\_esco\_pop, ano, sub\_esco\_pop, X2012}\SpecialCharTok{:}\NormalTok{X2021, }\AttributeTok{convert =} \ConstantTok{TRUE}\NormalTok{)}
\NormalTok{sub\_esco\_pop}\SpecialCharTok{$}\NormalTok{ano }\OtherTok{\textless{}{-}} \FunctionTok{gsub}\NormalTok{(}\StringTok{\textquotesingle{}X\textquotesingle{}}\NormalTok{, }\StringTok{\textquotesingle{}\textquotesingle{}}\NormalTok{, sub\_esco\_pop}\SpecialCharTok{$}\NormalTok{ano)}
\NormalTok{sub\_esco\_pop }\OtherTok{\textless{}{-}} \FunctionTok{transform}\NormalTok{(sub\_esco\_pop, }\AttributeTok{ano =} \FunctionTok{as.numeric}\NormalTok{(ano))}
\NormalTok{sub\_esco\_pop }\OtherTok{\textless{}{-}} \FunctionTok{transform}\NormalTok{(sub\_esco\_pop, }\AttributeTok{sub\_esco\_pop =} \FunctionTok{as.numeric}\NormalTok{(sub\_esco\_pop))}



\NormalTok{sub\_freq\_esco }\OtherTok{\textless{}{-}} \FunctionTok{read\_excel}\NormalTok{(}\StringTok{"sub.freq.esco.xlsx"}\NormalTok{)}
\NormalTok{sub\_freq\_esco }\OtherTok{\textless{}{-}} \FunctionTok{gather}\NormalTok{(sub\_freq\_esco, ano, sub\_freq\_esco, X2012}\SpecialCharTok{:}\NormalTok{X2021, }\AttributeTok{convert =} \ConstantTok{TRUE}\NormalTok{)}
\NormalTok{sub\_freq\_esco}\SpecialCharTok{$}\NormalTok{ano }\OtherTok{\textless{}{-}} \FunctionTok{gsub}\NormalTok{(}\StringTok{\textquotesingle{}X\textquotesingle{}}\NormalTok{, }\StringTok{\textquotesingle{}\textquotesingle{}}\NormalTok{, sub\_freq\_esco}\SpecialCharTok{$}\NormalTok{ano)}
\NormalTok{sub\_freq\_esco }\OtherTok{\textless{}{-}} \FunctionTok{transform}\NormalTok{(sub\_freq\_esco, }\AttributeTok{ano =} \FunctionTok{as.numeric}\NormalTok{(ano))}
\NormalTok{sub\_freq\_esco }\OtherTok{\textless{}{-}} \FunctionTok{transform}\NormalTok{(sub\_freq\_esco, }\AttributeTok{sub\_freq\_esco =} \FunctionTok{as.numeric}\NormalTok{(sub\_freq\_esco))}




\NormalTok{esperança\_de\_vida }\OtherTok{\textless{}{-}} \FunctionTok{read\_excel}\NormalTok{(}\StringTok{"esperança.de.vida.xlsx"}\NormalTok{)}
\NormalTok{esperança\_de\_vida }\OtherTok{\textless{}{-}} \FunctionTok{gather}\NormalTok{(esperança\_de\_vida, ano, esperança\_de\_vida, X2012}\SpecialCharTok{:}\NormalTok{X2021, }\AttributeTok{convert =} \ConstantTok{TRUE}\NormalTok{)}
\NormalTok{esperança\_de\_vida}\SpecialCharTok{$}\NormalTok{ano }\OtherTok{\textless{}{-}} \FunctionTok{gsub}\NormalTok{(}\StringTok{\textquotesingle{}X\textquotesingle{}}\NormalTok{, }\StringTok{\textquotesingle{}\textquotesingle{}}\NormalTok{, esperança\_de\_vida}\SpecialCharTok{$}\NormalTok{ano)}
\NormalTok{esperança\_de\_vida }\OtherTok{\textless{}{-}} \FunctionTok{transform}\NormalTok{(esperança\_de\_vida, }\AttributeTok{ano =} \FunctionTok{as.numeric}\NormalTok{(ano))}
\NormalTok{esperança\_de\_vida }\OtherTok{\textless{}{-}} \FunctionTok{transform}\NormalTok{(esperança\_de\_vida, esperança}\AttributeTok{\_de\_vida =} \FunctionTok{as.numeric}\NormalTok{(esperança\_de\_vida))}



\NormalTok{porcent\_pobres }\OtherTok{\textless{}{-}} \FunctionTok{read\_excel}\NormalTok{(}\StringTok{"porcent\_pobres.xlsx"}\NormalTok{)}
\NormalTok{porcent\_pobres }\OtherTok{\textless{}{-}} \FunctionTok{gather}\NormalTok{(porcent\_pobres, ano, porcent\_pobres, X2012}\SpecialCharTok{:}\NormalTok{X2021, }\AttributeTok{convert =} \ConstantTok{TRUE}\NormalTok{)}
\NormalTok{porcent\_pobres}\SpecialCharTok{$}\NormalTok{ano }\OtherTok{\textless{}{-}} \FunctionTok{gsub}\NormalTok{(}\StringTok{\textquotesingle{}X\textquotesingle{}}\NormalTok{, }\StringTok{\textquotesingle{}\textquotesingle{}}\NormalTok{, porcent\_pobres}\SpecialCharTok{$}\NormalTok{ano)}
\NormalTok{porcent\_pobres }\OtherTok{\textless{}{-}} \FunctionTok{transform}\NormalTok{(porcent\_pobres, }\AttributeTok{ano =} \FunctionTok{as.numeric}\NormalTok{(ano))}
\NormalTok{porcent\_pobres }\OtherTok{\textless{}{-}} \FunctionTok{transform}\NormalTok{(porcent\_pobres, }\AttributeTok{porcent\_pobres =} \FunctionTok{as.numeric}\NormalTok{(porcent\_pobres))}



\NormalTok{população\_total }\OtherTok{\textless{}{-}} \FunctionTok{read\_excel}\NormalTok{(}\StringTok{"população\_total.xlsx"}\NormalTok{)}
\NormalTok{população\_total }\OtherTok{\textless{}{-}} \FunctionTok{gather}\NormalTok{(população\_total, ano, população\_total, X2012}\SpecialCharTok{:}\NormalTok{X2021, }\AttributeTok{convert =} \ConstantTok{TRUE}\NormalTok{)}
\NormalTok{população\_total}\SpecialCharTok{$}\NormalTok{ano }\OtherTok{\textless{}{-}} \FunctionTok{gsub}\NormalTok{(}\StringTok{\textquotesingle{}X\textquotesingle{}}\NormalTok{, }\StringTok{\textquotesingle{}\textquotesingle{}}\NormalTok{, população\_total}\SpecialCharTok{$}\NormalTok{ano)}
\NormalTok{população\_total }\OtherTok{\textless{}{-}} \FunctionTok{transform}\NormalTok{(população\_total, }\AttributeTok{ano =} \FunctionTok{as.numeric}\NormalTok{(ano))}
\NormalTok{população\_total }\OtherTok{\textless{}{-}} \FunctionTok{transform}\NormalTok{(população\_total, população}\AttributeTok{\_total =} \FunctionTok{as.numeric}\NormalTok{(população\_total))}



\NormalTok{mortalidade\_infantil }\OtherTok{\textless{}{-}} \FunctionTok{read\_excel}\NormalTok{(}\StringTok{"mortalidade\_infantil.xlsx"}\NormalTok{)}
\NormalTok{mortalidade\_infantil }\OtherTok{\textless{}{-}} \FunctionTok{gather}\NormalTok{(mortalidade\_infantil, ano, mortalidade\_infantil, X2012}\SpecialCharTok{:}\NormalTok{X2021, }\AttributeTok{convert =} \ConstantTok{TRUE}\NormalTok{)}
\NormalTok{mortalidade\_infantil}\SpecialCharTok{$}\NormalTok{ano }\OtherTok{\textless{}{-}} \FunctionTok{gsub}\NormalTok{(}\StringTok{\textquotesingle{}X\textquotesingle{}}\NormalTok{, }\StringTok{\textquotesingle{}\textquotesingle{}}\NormalTok{, mortalidade\_infantil}\SpecialCharTok{$}\NormalTok{ano)}
\NormalTok{mortalidade\_infantil }\OtherTok{\textless{}{-}} \FunctionTok{transform}\NormalTok{(mortalidade\_infantil, }\AttributeTok{ano =} \FunctionTok{as.numeric}\NormalTok{(ano))}
\NormalTok{mortalidade\_infantil }\OtherTok{\textless{}{-}} \FunctionTok{transform}\NormalTok{(mortalidade\_infantil, }\AttributeTok{mortalidade\_infantil =} \FunctionTok{as.numeric}\NormalTok{(mortalidade\_infantil))}



\NormalTok{media\_anos\_de\_estudo }\OtherTok{\textless{}{-}} \FunctionTok{read\_excel}\NormalTok{(}\StringTok{"media\_anos\_de\_estudo.xlsx"}\NormalTok{)}
\NormalTok{media\_anos\_de\_estudo }\OtherTok{\textless{}{-}} \FunctionTok{gather}\NormalTok{(media\_anos\_de\_estudo, ano, media\_anos\_de\_estudo, X2012}\SpecialCharTok{:}\NormalTok{X2021, }\AttributeTok{convert =} \ConstantTok{TRUE}\NormalTok{)}
\NormalTok{media\_anos\_de\_estudo}\SpecialCharTok{$}\NormalTok{ano }\OtherTok{\textless{}{-}} \FunctionTok{gsub}\NormalTok{(}\StringTok{\textquotesingle{}X\textquotesingle{}}\NormalTok{, }\StringTok{\textquotesingle{}\textquotesingle{}}\NormalTok{, media\_anos\_de\_estudo}\SpecialCharTok{$}\NormalTok{ano)}
\NormalTok{media\_anos\_de\_estudo }\OtherTok{\textless{}{-}} \FunctionTok{transform}\NormalTok{(media\_anos\_de\_estudo, }\AttributeTok{ano =} \FunctionTok{as.numeric}\NormalTok{(ano))}
\NormalTok{media\_anos\_de\_estudo }\OtherTok{\textless{}{-}} \FunctionTok{transform}\NormalTok{(media\_anos\_de\_estudo, }\AttributeTok{media\_anos\_de\_estudo =} \FunctionTok{as.numeric}\NormalTok{(media\_anos\_de\_estudo))}



\NormalTok{indice\_gini }\OtherTok{\textless{}{-}} \FunctionTok{read\_excel}\NormalTok{(}\StringTok{"indice\_gini.xlsx"}\NormalTok{)}
\NormalTok{indice\_gini }\OtherTok{\textless{}{-}} \FunctionTok{gather}\NormalTok{(indice\_gini, ano, indice\_gini, X2012}\SpecialCharTok{:}\NormalTok{X2021, }\AttributeTok{convert =} \ConstantTok{TRUE}\NormalTok{)}
\NormalTok{indice\_gini}\SpecialCharTok{$}\NormalTok{ano }\OtherTok{\textless{}{-}} \FunctionTok{gsub}\NormalTok{(}\StringTok{\textquotesingle{}X\textquotesingle{}}\NormalTok{, }\StringTok{\textquotesingle{}\textquotesingle{}}\NormalTok{, indice\_gini}\SpecialCharTok{$}\NormalTok{ano)}
\NormalTok{indice\_gini }\OtherTok{\textless{}{-}} \FunctionTok{transform}\NormalTok{(indice\_gini, }\AttributeTok{ano =} \FunctionTok{as.numeric}\NormalTok{(ano))}
\NormalTok{indice\_gini }\OtherTok{\textless{}{-}} \FunctionTok{transform}\NormalTok{(indice\_gini, }\AttributeTok{indice\_gini =} \FunctionTok{as.numeric}\NormalTok{(indice\_gini))}



\NormalTok{ind\_theil\_L }\OtherTok{\textless{}{-}} \FunctionTok{read\_excel}\NormalTok{(}\StringTok{"ind\_theil\_L.xlsx"}\NormalTok{)}
\NormalTok{ind\_theil\_L }\OtherTok{\textless{}{-}} \FunctionTok{gather}\NormalTok{(ind\_theil\_L, ano, ind\_theil\_L, X2012}\SpecialCharTok{:}\NormalTok{X2021, }\AttributeTok{convert =} \ConstantTok{TRUE}\NormalTok{)}
\NormalTok{ind\_theil\_L}\SpecialCharTok{$}\NormalTok{ano }\OtherTok{\textless{}{-}} \FunctionTok{gsub}\NormalTok{(}\StringTok{\textquotesingle{}X\textquotesingle{}}\NormalTok{, }\StringTok{\textquotesingle{}\textquotesingle{}}\NormalTok{, ind\_theil\_L}\SpecialCharTok{$}\NormalTok{ano)}
\NormalTok{ind\_theil\_L }\OtherTok{\textless{}{-}} \FunctionTok{transform}\NormalTok{(ind\_theil\_L, }\AttributeTok{ano =} \FunctionTok{as.numeric}\NormalTok{(ano))}
\NormalTok{ind\_theil\_L }\OtherTok{\textless{}{-}} \FunctionTok{transform}\NormalTok{(ind\_theil\_L, }\AttributeTok{ind\_theil\_L =} \FunctionTok{as.numeric}\NormalTok{(ind\_theil\_L))}



\NormalTok{analfabetismo\_25\_anos }\OtherTok{\textless{}{-}} \FunctionTok{read\_excel}\NormalTok{(}\StringTok{"analfabetismo\_25\_anos.xlsx"}\NormalTok{)}
\NormalTok{analfabetismo\_25\_anos }\OtherTok{\textless{}{-}} \FunctionTok{gather}\NormalTok{(analfabetismo\_25\_anos, ano, analfabetismo\_25\_anos, X2012}\SpecialCharTok{:}\NormalTok{X2021, }\AttributeTok{convert =} \ConstantTok{TRUE}\NormalTok{)}
\NormalTok{analfabetismo\_25\_anos}\SpecialCharTok{$}\NormalTok{ano }\OtherTok{\textless{}{-}} \FunctionTok{gsub}\NormalTok{(}\StringTok{\textquotesingle{}X\textquotesingle{}}\NormalTok{, }\StringTok{\textquotesingle{}\textquotesingle{}}\NormalTok{, analfabetismo\_25\_anos}\SpecialCharTok{$}\NormalTok{ano)}
\NormalTok{analfabetismo\_25\_anos }\OtherTok{\textless{}{-}} \FunctionTok{transform}\NormalTok{(analfabetismo\_25\_anos, }\AttributeTok{ano =} \FunctionTok{as.numeric}\NormalTok{(ano))}
\NormalTok{analfabetismo\_25\_anos }\OtherTok{\textless{}{-}} \FunctionTok{transform}\NormalTok{(analfabetismo\_25\_anos, }\AttributeTok{analfabetismo\_25\_anos =} \FunctionTok{as.numeric}\NormalTok{(analfabetismo\_25\_anos))}



\NormalTok{analfabetismo\_18\_anos }\OtherTok{\textless{}{-}} \FunctionTok{read\_excel}\NormalTok{(}\StringTok{"analfabetismo\_18\_anos.xlsx"}\NormalTok{)}
\NormalTok{analfabetismo\_18\_anos }\OtherTok{\textless{}{-}} \FunctionTok{gather}\NormalTok{(analfabetismo\_18\_anos, ano, analfabetismo\_18\_anos, X2012}\SpecialCharTok{:}\NormalTok{X2021, }\AttributeTok{convert =} \ConstantTok{TRUE}\NormalTok{)}
\NormalTok{analfabetismo\_18\_anos}\SpecialCharTok{$}\NormalTok{ano }\OtherTok{\textless{}{-}} \FunctionTok{gsub}\NormalTok{(}\StringTok{\textquotesingle{}X\textquotesingle{}}\NormalTok{, }\StringTok{\textquotesingle{}\textquotesingle{}}\NormalTok{, analfabetismo\_18\_anos}\SpecialCharTok{$}\NormalTok{ano)}
\NormalTok{analfabetismo\_18\_anos }\OtherTok{\textless{}{-}} \FunctionTok{transform}\NormalTok{(analfabetismo\_18\_anos, }\AttributeTok{ano =} \FunctionTok{as.numeric}\NormalTok{(ano))}
\NormalTok{analfabetismo\_18\_anos }\OtherTok{\textless{}{-}} \FunctionTok{transform}\NormalTok{(analfabetismo\_18\_anos, }\AttributeTok{analfabetismo\_18\_anos =} \FunctionTok{as.numeric}\NormalTok{(analfabetismo\_18\_anos))}



\NormalTok{analfabetismo\_15\_anos }\OtherTok{\textless{}{-}} \FunctionTok{read\_excel}\NormalTok{(}\StringTok{"analfabetismo\_15\_anos.xlsx"}\NormalTok{)}
\NormalTok{analfabetismo\_15\_anos }\OtherTok{\textless{}{-}} \FunctionTok{gather}\NormalTok{(analfabetismo\_15\_anos, ano, analfabetismo\_15\_anos, X2012}\SpecialCharTok{:}\NormalTok{X2021, }\AttributeTok{convert =} \ConstantTok{TRUE}\NormalTok{)}
\NormalTok{analfabetismo\_15\_anos}\SpecialCharTok{$}\NormalTok{ano }\OtherTok{\textless{}{-}} \FunctionTok{gsub}\NormalTok{(}\StringTok{\textquotesingle{}X\textquotesingle{}}\NormalTok{, }\StringTok{\textquotesingle{}\textquotesingle{}}\NormalTok{, analfabetismo\_15\_anos}\SpecialCharTok{$}\NormalTok{ano)}
\NormalTok{analfabetismo\_15\_anos }\OtherTok{\textless{}{-}} \FunctionTok{transform}\NormalTok{(analfabetismo\_15\_anos, }\AttributeTok{ano =} \FunctionTok{as.numeric}\NormalTok{(ano))}
\NormalTok{analfabetismo\_15\_anos }\OtherTok{\textless{}{-}} \FunctionTok{transform}\NormalTok{(analfabetismo\_15\_anos, }\AttributeTok{analfabetismo\_15\_anos =} \FunctionTok{as.numeric}\NormalTok{(analfabetismo\_15\_anos))}



\NormalTok{IDHM }\OtherTok{\textless{}{-}} \FunctionTok{read\_excel}\NormalTok{(}\StringTok{"IDHM.xlsx"}\NormalTok{)}
\NormalTok{IDHM }\OtherTok{\textless{}{-}} \FunctionTok{gather}\NormalTok{(IDHM, ano, IDHM, X2012}\SpecialCharTok{:}\NormalTok{X2021, }\AttributeTok{convert =} \ConstantTok{TRUE}\NormalTok{)}
\NormalTok{IDHM}\SpecialCharTok{$}\NormalTok{ano }\OtherTok{\textless{}{-}} \FunctionTok{gsub}\NormalTok{(}\StringTok{\textquotesingle{}X\textquotesingle{}}\NormalTok{, }\StringTok{\textquotesingle{}\textquotesingle{}}\NormalTok{, IDHM}\SpecialCharTok{$}\NormalTok{ano)}
\NormalTok{IDHM }\OtherTok{\textless{}{-}} \FunctionTok{transform}\NormalTok{(IDHM, }\AttributeTok{ano =} \FunctionTok{as.numeric}\NormalTok{(ano))}
\NormalTok{IDHM }\OtherTok{\textless{}{-}} \FunctionTok{transform}\NormalTok{(IDHM, }\AttributeTok{IDHM =} \FunctionTok{as.numeric}\NormalTok{(IDHM))}

\CommentTok{\#adaptação do código desenvolvido por Anne (2020).}
\end{Highlighting}
\end{Shaded}

O próximo passo consistiu na contagem da ausência de valor nas
observações, nulos, e a proporção de nulos em relação às linhas da
tabela, esse processo foi repetido para todas as variáveis (Quadro 3), e
continuado para demais variáveis.

\begin{Shaded}
\begin{Highlighting}[]
\FunctionTok{print}\NormalTok{(}\FunctionTok{paste0}\NormalTok{(}\StringTok{"renda\_per\_capita"}\NormalTok{))}
\end{Highlighting}
\end{Shaded}

\begin{verbatim}
## [1] "renda_per_capita"
\end{verbatim}

\begin{Shaded}
\begin{Highlighting}[]
\NormalTok{renda\_per\_capita.na }\OtherTok{\textless{}{-}} \FunctionTok{as.data.frame}\NormalTok{(}\FunctionTok{sum}\NormalTok{(}\FunctionTok{is.na}\NormalTok{(renda\_per\_capita}\SpecialCharTok{$}\NormalTok{renda\_per\_capita)))}
\NormalTok{renda\_per\_capita.n }\OtherTok{\textless{}{-}} \FunctionTok{as.data.frame}\NormalTok{(}\FunctionTok{nrow}\NormalTok{(renda\_per\_capita))}
\NormalTok{renda\_per\_capita.na}\SpecialCharTok{$}\StringTok{\textasciigrave{}}\AttributeTok{sum(is.na(renda\_per\_capita$renda\_per\_capita))}\StringTok{\textasciigrave{}}\SpecialCharTok{/}\NormalTok{renda\_per\_capita.n}\SpecialCharTok{$}\StringTok{\textasciigrave{}}\AttributeTok{nrow(renda\_per\_capita)}\StringTok{\textasciigrave{}}\SpecialCharTok{*}\DecValTok{100}
\end{Highlighting}
\end{Shaded}

\begin{verbatim}
## [1] 0
\end{verbatim}

\begin{Shaded}
\begin{Highlighting}[]
\FunctionTok{print}\NormalTok{(}\FunctionTok{paste0}\NormalTok{(}\StringTok{"sub\_esco\_pop"}\NormalTok{))}
\end{Highlighting}
\end{Shaded}

\begin{verbatim}
## [1] "sub_esco_pop"
\end{verbatim}

\begin{Shaded}
\begin{Highlighting}[]
\NormalTok{sub\_esco\_pop.na }\OtherTok{\textless{}{-}} \FunctionTok{as.data.frame}\NormalTok{(}\FunctionTok{sum}\NormalTok{(}\FunctionTok{is.na}\NormalTok{(sub\_esco\_pop}\SpecialCharTok{$}\NormalTok{sub\_esco\_pop)))}
\NormalTok{sub\_esco\_pop.n }\OtherTok{\textless{}{-}} \FunctionTok{as.data.frame}\NormalTok{(}\FunctionTok{nrow}\NormalTok{(sub\_esco\_pop))}
\NormalTok{sub\_esco\_pop.na}\SpecialCharTok{$}\StringTok{\textasciigrave{}}\AttributeTok{sum(is.na(sub\_esco\_pop$sub\_esco\_pop))}\StringTok{\textasciigrave{}}\SpecialCharTok{/}\NormalTok{sub\_esco\_pop.n}\SpecialCharTok{$}\StringTok{\textasciigrave{}}\AttributeTok{nrow(sub\_esco\_pop)}\StringTok{\textasciigrave{}}\SpecialCharTok{*}\DecValTok{100}
\end{Highlighting}
\end{Shaded}

\begin{verbatim}
## [1] 0
\end{verbatim}

\begin{Shaded}
\begin{Highlighting}[]
\FunctionTok{print}\NormalTok{(}\FunctionTok{paste0}\NormalTok{(}\StringTok{"sub\_freq\_esco"}\NormalTok{))}
\end{Highlighting}
\end{Shaded}

\begin{verbatim}
## [1] "sub_freq_esco"
\end{verbatim}

\begin{Shaded}
\begin{Highlighting}[]
\NormalTok{sub\_freq\_esco.na }\OtherTok{\textless{}{-}} \FunctionTok{as.data.frame}\NormalTok{(}\FunctionTok{sum}\NormalTok{(}\FunctionTok{is.na}\NormalTok{(sub\_freq\_esco}\SpecialCharTok{$}\NormalTok{sub\_freq\_esco)))}
\NormalTok{sub\_freq\_esco.n }\OtherTok{\textless{}{-}} \FunctionTok{as.data.frame}\NormalTok{(}\FunctionTok{nrow}\NormalTok{(sub\_freq\_esco))}
\NormalTok{sub\_freq\_esco.na}\SpecialCharTok{$}\StringTok{\textasciigrave{}}\AttributeTok{sum(is.na(sub\_freq\_esco$sub\_freq\_esco))}\StringTok{\textasciigrave{}}\SpecialCharTok{/}\NormalTok{sub\_freq\_esco.n}\SpecialCharTok{$}\StringTok{\textasciigrave{}}\AttributeTok{nrow(sub\_freq\_esco)}\StringTok{\textasciigrave{}}\SpecialCharTok{*}\DecValTok{100}
\end{Highlighting}
\end{Shaded}

\begin{verbatim}
## [1] 0
\end{verbatim}

\begin{Shaded}
\begin{Highlighting}[]
\FunctionTok{print}\NormalTok{(}\FunctionTok{paste0}\NormalTok{(}\StringTok{"esperança\_de\_vida"}\NormalTok{))}
\end{Highlighting}
\end{Shaded}

\begin{verbatim}
## [1] "esperança_de_vida"
\end{verbatim}

\begin{Shaded}
\begin{Highlighting}[]
\NormalTok{esperança\_de\_vida.na }\OtherTok{\textless{}{-}} \FunctionTok{as.data.frame}\NormalTok{(}\FunctionTok{sum}\NormalTok{(}\FunctionTok{is.na}\NormalTok{(esperança\_de\_vida}\SpecialCharTok{$}\NormalTok{esperança\_de\_vida)))}
\NormalTok{esperança\_de\_vida.n }\OtherTok{\textless{}{-}} \FunctionTok{as.data.frame}\NormalTok{(}\FunctionTok{nrow}\NormalTok{(esperança\_de\_vida))}
\NormalTok{esperança\_de\_vida.na}\SpecialCharTok{$}\StringTok{\textasciigrave{}}\AttributeTok{sum(is.na(esperança\_de\_vida$esperança\_de\_vida))}\StringTok{\textasciigrave{}}\SpecialCharTok{/}\NormalTok{esperança\_de\_vida.n}\SpecialCharTok{$}\StringTok{\textasciigrave{}}\AttributeTok{nrow(esperança\_de\_vida)}\StringTok{\textasciigrave{}}\SpecialCharTok{*}\DecValTok{100}
\end{Highlighting}
\end{Shaded}

\begin{verbatim}
## [1] 0
\end{verbatim}

\begin{Shaded}
\begin{Highlighting}[]
\FunctionTok{print}\NormalTok{(}\FunctionTok{paste0}\NormalTok{(}\StringTok{"porcent\_pobres"}\NormalTok{))}
\end{Highlighting}
\end{Shaded}

\begin{verbatim}
## [1] "porcent_pobres"
\end{verbatim}

\begin{Shaded}
\begin{Highlighting}[]
\NormalTok{porcent\_pobres.na }\OtherTok{\textless{}{-}} \FunctionTok{as.data.frame}\NormalTok{(}\FunctionTok{sum}\NormalTok{(}\FunctionTok{is.na}\NormalTok{(porcent\_pobres}\SpecialCharTok{$}\NormalTok{porcent\_pobres)))}
\NormalTok{porcent\_pobres.n }\OtherTok{\textless{}{-}} \FunctionTok{as.data.frame}\NormalTok{(}\FunctionTok{nrow}\NormalTok{(porcent\_pobres))}
\NormalTok{porcent\_pobres.na}\SpecialCharTok{$}\StringTok{\textasciigrave{}}\AttributeTok{sum(is.na(rporcent\_pobres$porcent\_pobres))}\StringTok{\textasciigrave{}}\SpecialCharTok{/}\NormalTok{porcent\_pobres.n}\SpecialCharTok{$}\StringTok{\textasciigrave{}}\AttributeTok{nrow(porcent\_pobres)}\StringTok{\textasciigrave{}}\SpecialCharTok{*}\DecValTok{100}
\end{Highlighting}
\end{Shaded}

\begin{verbatim}
## numeric(0)
\end{verbatim}

\begin{Shaded}
\begin{Highlighting}[]
\FunctionTok{print}\NormalTok{(}\FunctionTok{paste0}\NormalTok{(}\StringTok{"população\_total"}\NormalTok{))}
\end{Highlighting}
\end{Shaded}

\begin{verbatim}
## [1] "população_total"
\end{verbatim}

\begin{Shaded}
\begin{Highlighting}[]
\NormalTok{população\_total.na }\OtherTok{\textless{}{-}} \FunctionTok{as.data.frame}\NormalTok{(}\FunctionTok{sum}\NormalTok{(}\FunctionTok{is.na}\NormalTok{(população\_total}\SpecialCharTok{$}\NormalTok{população\_total)))}
\NormalTok{população\_total.n }\OtherTok{\textless{}{-}} \FunctionTok{as.data.frame}\NormalTok{(}\FunctionTok{nrow}\NormalTok{(população\_total))}
\NormalTok{população\_total.na}\SpecialCharTok{$}\StringTok{\textasciigrave{}}\AttributeTok{sum(is.na(população\_total$população\_total))}\StringTok{\textasciigrave{}}\SpecialCharTok{/}\NormalTok{população\_total.n}\SpecialCharTok{$}\StringTok{\textasciigrave{}}\AttributeTok{nrow(população\_total)}\StringTok{\textasciigrave{}}\SpecialCharTok{*}\DecValTok{100}
\end{Highlighting}
\end{Shaded}

\begin{verbatim}
## [1] 0
\end{verbatim}

\begin{Shaded}
\begin{Highlighting}[]
\FunctionTok{print}\NormalTok{(}\FunctionTok{paste0}\NormalTok{(}\StringTok{"mortalidade\_infantil"}\NormalTok{))}
\end{Highlighting}
\end{Shaded}

\begin{verbatim}
## [1] "mortalidade_infantil"
\end{verbatim}

\begin{Shaded}
\begin{Highlighting}[]
\NormalTok{mortalidade\_infantil.na }\OtherTok{\textless{}{-}} \FunctionTok{as.data.frame}\NormalTok{(}\FunctionTok{sum}\NormalTok{(}\FunctionTok{is.na}\NormalTok{(mortalidade\_infantil}\SpecialCharTok{$}\NormalTok{mortalidade\_infantil)))}
\NormalTok{mortalidade\_infantil.n }\OtherTok{\textless{}{-}} \FunctionTok{as.data.frame}\NormalTok{(}\FunctionTok{nrow}\NormalTok{(mortalidade\_infantil))}
\NormalTok{mortalidade\_infantil.na}\SpecialCharTok{$}\StringTok{\textasciigrave{}}\AttributeTok{sum(is.na(mortalidade\_infantil$mortalidade\_infantil))}\StringTok{\textasciigrave{}}\SpecialCharTok{/}\NormalTok{mortalidade\_infantil.n}\SpecialCharTok{$}\StringTok{\textasciigrave{}}\AttributeTok{nrow(mortalidade\_infantil)}\StringTok{\textasciigrave{}}\SpecialCharTok{*}\DecValTok{100}
\end{Highlighting}
\end{Shaded}

\begin{verbatim}
## [1] 0
\end{verbatim}

\begin{Shaded}
\begin{Highlighting}[]
\FunctionTok{print}\NormalTok{(}\FunctionTok{paste0}\NormalTok{(}\StringTok{"media\_anos\_de\_estudo"}\NormalTok{))}
\end{Highlighting}
\end{Shaded}

\begin{verbatim}
## [1] "media_anos_de_estudo"
\end{verbatim}

\begin{Shaded}
\begin{Highlighting}[]
\NormalTok{media\_anos\_de\_estudo.na }\OtherTok{\textless{}{-}} \FunctionTok{as.data.frame}\NormalTok{(}\FunctionTok{sum}\NormalTok{(}\FunctionTok{is.na}\NormalTok{(media\_anos\_de\_estudo}\SpecialCharTok{$}\NormalTok{media\_anos\_de\_estudo)))}
\NormalTok{media\_anos\_de\_estudo.n }\OtherTok{\textless{}{-}} \FunctionTok{as.data.frame}\NormalTok{(}\FunctionTok{nrow}\NormalTok{(media\_anos\_de\_estudo))}
\NormalTok{media\_anos\_de\_estudo.na}\SpecialCharTok{$}\StringTok{\textasciigrave{}}\AttributeTok{sum(is.na(media\_anos\_de\_estudo$media\_anos\_de\_estudo))}\StringTok{\textasciigrave{}}\SpecialCharTok{/}\NormalTok{media\_anos\_de\_estudo.n}\SpecialCharTok{$}\StringTok{\textasciigrave{}}\AttributeTok{nrow(media\_anos\_de\_estudo)}\StringTok{\textasciigrave{}}\SpecialCharTok{*}\DecValTok{100}
\end{Highlighting}
\end{Shaded}

\begin{verbatim}
## [1] 0
\end{verbatim}

\begin{Shaded}
\begin{Highlighting}[]
\FunctionTok{print}\NormalTok{(}\FunctionTok{paste0}\NormalTok{(}\StringTok{"indice\_gini"}\NormalTok{))}
\end{Highlighting}
\end{Shaded}

\begin{verbatim}
## [1] "indice_gini"
\end{verbatim}

\begin{Shaded}
\begin{Highlighting}[]
\NormalTok{indice\_gini.na }\OtherTok{\textless{}{-}} \FunctionTok{as.data.frame}\NormalTok{(}\FunctionTok{sum}\NormalTok{(}\FunctionTok{is.na}\NormalTok{(indice\_gini}\SpecialCharTok{$}\NormalTok{indice\_gini)))}
\NormalTok{indice\_gini.n }\OtherTok{\textless{}{-}} \FunctionTok{as.data.frame}\NormalTok{(}\FunctionTok{nrow}\NormalTok{(indice\_gini))}
\NormalTok{indice\_gini.na}\SpecialCharTok{$}\StringTok{\textasciigrave{}}\AttributeTok{sum(is.na(indice\_gini$indice\_gini))}\StringTok{\textasciigrave{}}\SpecialCharTok{/}\NormalTok{indice\_gini.n}\SpecialCharTok{$}\StringTok{\textasciigrave{}}\AttributeTok{nrow(indice\_gini)}\StringTok{\textasciigrave{}}\SpecialCharTok{*}\DecValTok{100}
\end{Highlighting}
\end{Shaded}

\begin{verbatim}
## [1] 0
\end{verbatim}

\begin{Shaded}
\begin{Highlighting}[]
\FunctionTok{print}\NormalTok{(}\FunctionTok{paste0}\NormalTok{(}\StringTok{"ind\_theil\_L"}\NormalTok{))}
\end{Highlighting}
\end{Shaded}

\begin{verbatim}
## [1] "ind_theil_L"
\end{verbatim}

\begin{Shaded}
\begin{Highlighting}[]
\NormalTok{ind\_theil\_L.na }\OtherTok{\textless{}{-}} \FunctionTok{as.data.frame}\NormalTok{(}\FunctionTok{sum}\NormalTok{(}\FunctionTok{is.na}\NormalTok{(ind\_theil\_L}\SpecialCharTok{$}\NormalTok{ind\_theil\_L)))}
\NormalTok{ind\_theil\_L.n }\OtherTok{\textless{}{-}} \FunctionTok{as.data.frame}\NormalTok{(}\FunctionTok{nrow}\NormalTok{(ind\_theil\_L))}
\NormalTok{ind\_theil\_L.na}\SpecialCharTok{$}\StringTok{\textasciigrave{}}\AttributeTok{sum(is.na(ind\_theil\_L$ind\_theil\_L))}\StringTok{\textasciigrave{}}\SpecialCharTok{/}\NormalTok{ind\_theil\_L.n}\SpecialCharTok{$}\StringTok{\textasciigrave{}}\AttributeTok{nrow(ind\_theil\_L)}\StringTok{\textasciigrave{}}\SpecialCharTok{*}\DecValTok{100}
\end{Highlighting}
\end{Shaded}

\begin{verbatim}
## [1] 0
\end{verbatim}

\begin{Shaded}
\begin{Highlighting}[]
\FunctionTok{print}\NormalTok{(}\FunctionTok{paste0}\NormalTok{(}\StringTok{"analfabetismo\_25\_anos"}\NormalTok{))}
\end{Highlighting}
\end{Shaded}

\begin{verbatim}
## [1] "analfabetismo_25_anos"
\end{verbatim}

\begin{Shaded}
\begin{Highlighting}[]
\NormalTok{analfabetismo\_25\_anos.na }\OtherTok{\textless{}{-}} \FunctionTok{as.data.frame}\NormalTok{(}\FunctionTok{sum}\NormalTok{(}\FunctionTok{is.na}\NormalTok{(analfabetismo\_25\_anos}\SpecialCharTok{$}\NormalTok{analfabetismo\_25\_anos)))}
\NormalTok{analfabetismo\_25\_anos.n }\OtherTok{\textless{}{-}} \FunctionTok{as.data.frame}\NormalTok{(}\FunctionTok{nrow}\NormalTok{(analfabetismo\_25\_anos))}
\NormalTok{analfabetismo\_25\_anos.na}\SpecialCharTok{$}\StringTok{\textasciigrave{}}\AttributeTok{sum(is.na(analfabetismo\_25\_anos$analfabetismo\_25\_anos))}\StringTok{\textasciigrave{}}\SpecialCharTok{/}\NormalTok{analfabetismo\_25\_anos.n}\SpecialCharTok{$}\StringTok{\textasciigrave{}}\AttributeTok{nrow(analfabetismo\_25\_anos)}\StringTok{\textasciigrave{}}\SpecialCharTok{*}\DecValTok{100}
\end{Highlighting}
\end{Shaded}

\begin{verbatim}
## [1] 0
\end{verbatim}

\begin{Shaded}
\begin{Highlighting}[]
\FunctionTok{print}\NormalTok{(}\FunctionTok{paste0}\NormalTok{(}\StringTok{"analfabetismo\_18\_anos"}\NormalTok{))}
\end{Highlighting}
\end{Shaded}

\begin{verbatim}
## [1] "analfabetismo_18_anos"
\end{verbatim}

\begin{Shaded}
\begin{Highlighting}[]
\NormalTok{analfabetismo\_18\_anos.na }\OtherTok{\textless{}{-}} \FunctionTok{as.data.frame}\NormalTok{(}\FunctionTok{sum}\NormalTok{(}\FunctionTok{is.na}\NormalTok{(analfabetismo\_18\_anos}\SpecialCharTok{$}\NormalTok{analfabetismo\_18\_anos)))}
\NormalTok{analfabetismo\_18\_anos.n }\OtherTok{\textless{}{-}} \FunctionTok{as.data.frame}\NormalTok{(}\FunctionTok{nrow}\NormalTok{(analfabetismo\_18\_anos))}
\NormalTok{analfabetismo\_18\_anos.na}\SpecialCharTok{$}\StringTok{\textasciigrave{}}\AttributeTok{sum(is.na(analfabetismo\_18\_anos$analfabetismo\_18\_anos))}\StringTok{\textasciigrave{}}\SpecialCharTok{/}\NormalTok{analfabetismo\_18\_anos.n}\SpecialCharTok{$}\StringTok{\textasciigrave{}}\AttributeTok{nrow(analfabetismo\_18\_anos)}\StringTok{\textasciigrave{}}\SpecialCharTok{*}\DecValTok{100}
\end{Highlighting}
\end{Shaded}

\begin{verbatim}
## [1] 0
\end{verbatim}

\begin{Shaded}
\begin{Highlighting}[]
\FunctionTok{print}\NormalTok{(}\FunctionTok{paste0}\NormalTok{(}\StringTok{"analfabetismo\_15\_anos"}\NormalTok{))}
\end{Highlighting}
\end{Shaded}

\begin{verbatim}
## [1] "analfabetismo_15_anos"
\end{verbatim}

\begin{Shaded}
\begin{Highlighting}[]
\NormalTok{analfabetismo\_15\_anos.na }\OtherTok{\textless{}{-}} \FunctionTok{as.data.frame}\NormalTok{(}\FunctionTok{sum}\NormalTok{(}\FunctionTok{is.na}\NormalTok{(analfabetismo\_15\_anos}\SpecialCharTok{$}\NormalTok{analfabetismo\_15\_anos)))}
\NormalTok{analfabetismo\_15\_anos.n }\OtherTok{\textless{}{-}} \FunctionTok{as.data.frame}\NormalTok{(}\FunctionTok{nrow}\NormalTok{(analfabetismo\_15\_anos))}
\NormalTok{analfabetismo\_15\_anos.na}\SpecialCharTok{$}\StringTok{\textasciigrave{}}\AttributeTok{sum(is.na(analfabetismo\_15\_anos$analfabetismo\_15\_anos))}\StringTok{\textasciigrave{}}\SpecialCharTok{/}\NormalTok{analfabetismo\_15\_anos.n}\SpecialCharTok{$}\StringTok{\textasciigrave{}}\AttributeTok{nrow(analfabetismo\_15\_anos)}\StringTok{\textasciigrave{}}\SpecialCharTok{*}\DecValTok{100}
\end{Highlighting}
\end{Shaded}

\begin{verbatim}
## [1] 0
\end{verbatim}

\begin{Shaded}
\begin{Highlighting}[]
\FunctionTok{print}\NormalTok{(}\FunctionTok{paste0}\NormalTok{(}\StringTok{"IDHM"}\NormalTok{))}
\end{Highlighting}
\end{Shaded}

\begin{verbatim}
## [1] "IDHM"
\end{verbatim}

\begin{Shaded}
\begin{Highlighting}[]
\NormalTok{IDHM.na }\OtherTok{\textless{}{-}} \FunctionTok{as.data.frame}\NormalTok{(}\FunctionTok{sum}\NormalTok{(}\FunctionTok{is.na}\NormalTok{(IDHM}\SpecialCharTok{$}\NormalTok{IDHM)))}
\NormalTok{IDHM.n }\OtherTok{\textless{}{-}} \FunctionTok{as.data.frame}\NormalTok{(}\FunctionTok{nrow}\NormalTok{(IDHM))}
\NormalTok{IDHM.na}\SpecialCharTok{$}\StringTok{\textasciigrave{}}\AttributeTok{sum(is.na(IDHM$IDHM))}\StringTok{\textasciigrave{}}\SpecialCharTok{/}\NormalTok{IDHM.n}\SpecialCharTok{$}\StringTok{\textasciigrave{}}\AttributeTok{nrow(IDHM)}\StringTok{\textasciigrave{}}\SpecialCharTok{*}\DecValTok{100}
\end{Highlighting}
\end{Shaded}

\begin{verbatim}
## [1] 0
\end{verbatim}

\begin{Shaded}
\begin{Highlighting}[]
\CommentTok{\#adaptação do código desenvolvido por Anne (2020).}
\end{Highlighting}
\end{Shaded}

Seguindo, foi criado um data frame único, nomeado de IDHM.AED, através
da função ``join'' para unir os data frames com os indicadores indexando
o território e o ano para não desorganizar os dados.

\begin{Shaded}
\begin{Highlighting}[]
\NormalTok{IDHM.AED }\OtherTok{=}\NormalTok{ renda\_per\_capita}
\NormalTok{IDHM.AED }\OtherTok{\textless{}{-}} \FunctionTok{join}\NormalTok{(IDHM.AED, sub\_esco\_pop, }\AttributeTok{by =} \FunctionTok{c}\NormalTok{(}\StringTok{"ano"} \OtherTok{=} \StringTok{"ano"}\NormalTok{, }\StringTok{"Territorialidades"} \OtherTok{=} \StringTok{"Territorialidades"}\NormalTok{))}
\NormalTok{IDHM.AED }\OtherTok{=} \FunctionTok{join}\NormalTok{(IDHM.AED, sub\_freq\_esco, }\AttributeTok{by =} \FunctionTok{c}\NormalTok{(}\StringTok{"ano"} \OtherTok{=} \StringTok{"ano"}\NormalTok{, }\StringTok{"Territorialidades"} \OtherTok{=} \StringTok{"Territorialidades"}\NormalTok{))}
\NormalTok{IDHM.AED }\OtherTok{=} \FunctionTok{join}\NormalTok{(IDHM.AED, esperança\_de\_vida, }\AttributeTok{by =} \FunctionTok{c}\NormalTok{(}\StringTok{"ano"} \OtherTok{=} \StringTok{"ano"}\NormalTok{, }\StringTok{"Territorialidades"} \OtherTok{=} \StringTok{"Territorialidades"}\NormalTok{))}
\NormalTok{IDHM.AED }\OtherTok{=} \FunctionTok{join}\NormalTok{(IDHM.AED, porcent\_pobres, }\AttributeTok{by =} \FunctionTok{c}\NormalTok{(}\StringTok{"ano"} \OtherTok{=} \StringTok{"ano"}\NormalTok{, }\StringTok{"Territorialidades"} \OtherTok{=} \StringTok{"Territorialidades"}\NormalTok{))}
\NormalTok{IDHM.AED }\OtherTok{=} \FunctionTok{join}\NormalTok{(IDHM.AED, população\_total, }\AttributeTok{by =} \FunctionTok{c}\NormalTok{(}\StringTok{"ano"} \OtherTok{=} \StringTok{"ano"}\NormalTok{, }\StringTok{"Territorialidades"} \OtherTok{=} \StringTok{"Territorialidades"}\NormalTok{))}
\NormalTok{IDHM.AED }\OtherTok{=} \FunctionTok{join}\NormalTok{(IDHM.AED, mortalidade\_infantil, }\AttributeTok{by =} \FunctionTok{c}\NormalTok{(}\StringTok{"ano"} \OtherTok{=} \StringTok{"ano"}\NormalTok{, }\StringTok{"Territorialidades"} \OtherTok{=} \StringTok{"Territorialidades"}\NormalTok{))}
\NormalTok{IDHM.AED }\OtherTok{=} \FunctionTok{join}\NormalTok{(IDHM.AED, media\_anos\_de\_estudo, }\AttributeTok{by =} \FunctionTok{c}\NormalTok{(}\StringTok{"ano"} \OtherTok{=} \StringTok{"ano"}\NormalTok{, }\StringTok{"Territorialidades"} \OtherTok{=} \StringTok{"Territorialidades"}\NormalTok{))}
\NormalTok{IDHM.AED }\OtherTok{=} \FunctionTok{join}\NormalTok{(IDHM.AED, indice\_gini, }\AttributeTok{by =} \FunctionTok{c}\NormalTok{(}\StringTok{"ano"} \OtherTok{=} \StringTok{"ano"}\NormalTok{, }\StringTok{"Territorialidades"} \OtherTok{=} \StringTok{"Territorialidades"}\NormalTok{))}
\NormalTok{IDHM.AED }\OtherTok{=} \FunctionTok{join}\NormalTok{(IDHM.AED, ind\_theil\_L, }\AttributeTok{by =} \FunctionTok{c}\NormalTok{(}\StringTok{"ano"} \OtherTok{=} \StringTok{"ano"}\NormalTok{, }\StringTok{"Territorialidades"} \OtherTok{=} \StringTok{"Territorialidades"}\NormalTok{))}
\NormalTok{IDHM.AED }\OtherTok{=} \FunctionTok{join}\NormalTok{(IDHM.AED, analfabetismo\_25\_anos, }\AttributeTok{by =} \FunctionTok{c}\NormalTok{(}\StringTok{"ano"} \OtherTok{=} \StringTok{"ano"}\NormalTok{, }\StringTok{"Territorialidades"} \OtherTok{=} \StringTok{"Territorialidades"}\NormalTok{))}
\NormalTok{IDHM.AED }\OtherTok{=} \FunctionTok{join}\NormalTok{(IDHM.AED, analfabetismo\_18\_anos, }\AttributeTok{by =} \FunctionTok{c}\NormalTok{(}\StringTok{"ano"} \OtherTok{=} \StringTok{"ano"}\NormalTok{, }\StringTok{"Territorialidades"} \OtherTok{=} \StringTok{"Territorialidades"}\NormalTok{))}
\NormalTok{IDHM.AED }\OtherTok{=} \FunctionTok{join}\NormalTok{(IDHM.AED, analfabetismo\_15\_anos, }\AttributeTok{by =} \FunctionTok{c}\NormalTok{(}\StringTok{"ano"} \OtherTok{=} \StringTok{"ano"}\NormalTok{, }\StringTok{"Territorialidades"} \OtherTok{=} \StringTok{"Territorialidades"}\NormalTok{))}
\NormalTok{IDHM.AED }\OtherTok{=} \FunctionTok{join}\NormalTok{(IDHM.AED, IDHM, }\AttributeTok{by =} \FunctionTok{c}\NormalTok{(}\StringTok{"ano"} \OtherTok{=} \StringTok{"ano"}\NormalTok{, }\StringTok{"Territorialidades"} \OtherTok{=} \StringTok{"Territorialidades"}\NormalTok{))}

\CommentTok{\#adaptação do código desenvolvido por Anne (2020).}
\end{Highlighting}
\end{Shaded}

Em todas as etapas do processo da engenharia de dados não foram
encontrados erros no data frame, valores nulos ou NA (Not Available). e
a criação do data frame único não apresentou inconsistência.

Com o uso da função ``sapply'' ela retornou a contagem dos valores
apresentados no data frame, introduzido seu uso a seção 3.5 Vetorização
e Funcionais em ``Introduction to Data Science'', (IRIZARRY, 2019)
marcando em territorialidades o valor 28 com os 26 estados o distrito
federal e o Brasil, ano obteve 10 valores, os anos de 2010 a 2021 e
assim por diante nas demais variáveis contando os valores únicos, sem
contar suas repetições na coluna.

O uso da função ``str'' também no livro do Professor Irizarry no 2.4.2
demonstrando um objeto, demonstra os dados do data frame em ordem, no
entanto, o fato do data frame ser muito grande impossibilita a
demonstração de toda a estrutura do data frame.

\begin{Shaded}
\begin{Highlighting}[]
\NormalTok{AED.df }\OtherTok{=}\NormalTok{ IDHM.AED}



\CommentTok{\# Verificar que o número de estados continua os mesmos e o data frame está correto}


\FunctionTok{sapply}\NormalTok{(AED.df, }\ControlFlowTok{function}\NormalTok{(x) }\FunctionTok{length}\NormalTok{(}\FunctionTok{unique}\NormalTok{(x)))}
\end{Highlighting}
\end{Shaded}

\begin{verbatim}
##     Territorialidades                   ano      renda_per_capita 
##                    28                    10                   280 
##          sub_esco_pop         sub_freq_esco     esperança_de_vida 
##                   179                   156                   238 
##        porcent_pobres       população_total  mortalidade_infantil 
##                   261                   280                   255 
##  media_anos_de_estudo           indice_gini           ind_theil_L 
##                   202                   134                   182 
## analfabetismo_25_anos analfabetismo_18_anos analfabetismo_15_anos 
##                   264                   252                   248 
##                  IDHM 
##                   135
\end{verbatim}

\begin{Shaded}
\begin{Highlighting}[]
\FunctionTok{str}\NormalTok{(AED.df)}
\end{Highlighting}
\end{Shaded}

\begin{verbatim}
## 'data.frame':    280 obs. of  16 variables:
##  $ Territorialidades    : chr  "Brasil" "Acre" "Alagoas" "Amapá" ...
##  $ ano                  : num  2012 2012 2012 2012 2012 ...
##  $ renda_per_capita     : num  759 517 395 528 559 ...
##  $ sub_esco_pop         : num  0.606 0.59 0.487 0.67 0.613 0.51 0.54 0.765 0.613 0.619 ...
##  $ sub_freq_esco        : num  0.731 0.681 0.645 0.653 0.642 0.639 0.742 0.77 0.735 0.741 ...
##  $ esperança_de_vida    : num  74.5 72.5 70 72.8 70.8 ...
##  $ porcent_pobres       : num  11.4 23.8 23.4 18.4 22.2 ...
##  $ população_total      : num  1.98e+08 7.77e+05 3.22e+06 7.21e+05 3.54e+06 ...
##  $ mortalidade_infantil : num  15.8 20.2 26.1 24.3 20.9 ...
##  $ media_anos_de_estudo : num  8.56 7.72 6.8 9.09 8.63 ...
##  $ indice_gini          : num  0.54 0.566 0.503 0.528 0.589 0.563 0.545 0.601 0.489 0.474 ...
##  $ ind_theil_L          : num  0.526 0.585 0.447 0.483 0.619 0.571 0.54 0.664 0.411 0.383 ...
##  $ analfabetismo_25_anos: num  10.22 18.22 24.22 7.93 9.46 ...
##  $ analfabetismo_18_anos: num  8.75 14.72 20.54 6.37 7.89 ...
##  $ analfabetismo_15_anos: num  8.21 13.48 18.97 5.76 7.22 ...
##  $ IDHM                 : num  0.746 0.701 0.651 0.707 0.691 0.678 0.701 0.825 0.758 0.744 ...
\end{verbatim}

\begin{Shaded}
\begin{Highlighting}[]
\CommentTok{\#adaptação do código desenvolvido por Anne (2020) e Irizarry ( 2019).}
\end{Highlighting}
\end{Shaded}

Nessa última etapa encerrou o processo de engenharia de dados terminando
com uma etapa de verificação para garantir que não restassem
inconsistências ou erros. Esse processo de verificação envolveu a
revisão de valores extremos e a confirmação das estatísticas
descritivas, como médias, para garantir a acurácia dos dados.

Em todas essas etapas, é importante que cada decisão tomada seja
documentada de maneira detalhada e transparente. Esse rigor metodológico
permite não apenas a reprodutibilidade da pesquisa, mas também garante
que os resultados derivados desses dados sejam confiáveis e válidos.

\textbf{4.1. ANÁLISE EXPLORATÓRIA DE DADOS}

Subsequentemente a engenharia de dados, implementa-se uma Análise
Exploratória de Dados (AED) detalhada, que serve como o alicerce para a
compreensão e interpretação das complexidades subjacentes dos dados.
Essa metodologia se destina a extrair percepções intrínsecas, investigar
a estrutura, detectar outliers e testar hipóteses subjacentes que
poderiam potencialmente beneficiar a modelagem de Random Forest mais
adiante.

Na etapa inicial da análise exploratória, é realizada uma análise uni
variada rigorosa para cada variável no conjunto de dados. Isso implica
na geração de estatísticas descritivas que fornecem um entendimento das
características centrais de cada variável, como tendência central
(média, mediana), dispersão (variância, desvio padrão), e extremos
(mínimo, máximo). É necessário visualizar a distribuição de cada
variável para entender sua forma, identificar qualquer desvio de
normalidade, e observar a presença de valores extremos.

Utiliza-se as técnicas de visualização, como gráficos de dispersão e
mapas de calor de correlação, para auxiliar na visualização dessas
relações multidimensionais. Irizarry (2019) utiliza em sua maioria os
pacotes ``dplyr'' e ``ggplot2'' que permitem resumir, sintetizar dados e
plotar gráficos e ``boxplots'' respectivamente.

Outliers, ou valores extremos, podem ter um impacto substancial nos
resultados do modelo. Como tal, parte de nossa análise exploratória
envolve a detecção e tratamento apropriado desses outliers. Implementar
métodos robustos para a identificação desses valores e, com base na sua
natureza e impacto nos dados, toma-se uma decisão informada sobre se
devem ser mantidos, transformados ou removidos.

Na análise exploratória do IDHM utilizou-se os pacotes: ``corrplot'',
``RColorBrewer'', ``ggplot2'' e ``ggpubr''. O primeiro passo foi
preparar um dataset para montar uma matriz de correlação, para
compreender as inter-relações entre os indicadores.

O pacote ``corrplot'' - Visualization of a Correlation Matrix (WEI et
al.,s.d.) em R é uma ferramenta para visualização de matrizes de
correlação. Ele oferece meios visuais exploratórios para a matriz de
correlação e suporta a reordenação automática de variáveis, auxiliando
na detecção de padrões ocultos entre as variáveis.

O ``ggplot2'' descrito por Wickham (2016) no livro dedicado ao pacote
``ggplot2 Elegant Graphics for Data Analysis'' é utilizado para criar
gráficos de forma declarativa, baseado em ``The Grammar of Graphics''
(WILKINSON, 2012), Com ele, o usuário fornece os dados, específica como
mapear as variáveis para as estéticas e quais primitivos gráficos
utilizar. O ``ggplot2'' cuida dos detalhes, permitindo a criação de
visualizações de dados elegantes e sofisticadas.

O pacote ``RColorBrewer'' (NEUWIRTH, 2022), oferece paletas de cores
para mapas (e outros gráficos) desenhados por Cynthia Brewer
(``COLORBREWER: COLOR ADVICE FOR MAPS'', {[}s.d.{]}). Estas paletas são
eficazes em representar informações de maneira clara e esteticamente
agradável.

O pacote ``ggpubr'' - `ggplot2' Based Publication Ready Plots
(KASSAMBARA, 2023) em R complementa o ``ggplot2'', facilitando a criação
de gráficos prontos para publicação. Embora o ``ggplot2'' seja flexível
e excelente para visualização de dados, os gráficos gerados por padrão
muitas vezes necessitam de formatação adicional. Personalizar um gráfico
no ``ggplot2'' pode ser complexo para pesquisadores sem habilidades
avançadas em R. Assim, ``ggpubr'' oferece funções simples de usar para
criar e personalizar gráficos ``ggplot2'' de qualidade profissional.

Primeiramente, limpou-se os dados eliminando a coluna não numérica e o
ano, ``Territorialidades'' e ``ano'', focando somente nos indicadores
quantitativos. Com os dados prontos, calculamos a matriz de correlação
dos indicadores restantes. Por fim, para melhor visualização e
interpretação, utilizamos o pacote ``corrplot'' para exibir a matriz.

\begin{Shaded}
\begin{Highlighting}[]
\FunctionTok{library}\NormalTok{(corrplot)}
\end{Highlighting}
\end{Shaded}

\begin{verbatim}
## Warning: package 'corrplot' was built under R version 4.3.1
\end{verbatim}

\begin{verbatim}
## corrplot 0.92 loaded
\end{verbatim}

\begin{Shaded}
\begin{Highlighting}[]
\FunctionTok{library}\NormalTok{(RColorBrewer)}
\end{Highlighting}
\end{Shaded}

\begin{verbatim}
## Warning: package 'RColorBrewer' was built under R version 4.3.1
\end{verbatim}

\begin{Shaded}
\begin{Highlighting}[]
\FunctionTok{library}\NormalTok{(ggplot2)}
\end{Highlighting}
\end{Shaded}

\begin{verbatim}
## Warning: package 'ggplot2' was built under R version 4.3.1
\end{verbatim}

\begin{Shaded}
\begin{Highlighting}[]
\FunctionTok{library}\NormalTok{(ggpubr)}
\end{Highlighting}
\end{Shaded}

\begin{verbatim}
## Warning: package 'ggpubr' was built under R version 4.3.1
\end{verbatim}

\begin{verbatim}
## 
## Attaching package: 'ggpubr'
\end{verbatim}

\begin{verbatim}
## The following object is masked from 'package:plyr':
## 
##     mutate
\end{verbatim}

\begin{Shaded}
\begin{Highlighting}[]
\CommentTok{\# análise exploratória}

\CommentTok{\# Montar uma matriz de correlação básica de cada indicador}
\CommentTok{\#criar data frame que remova as linhas com nulos.}

\NormalTok{cc }\OtherTok{=} \FunctionTok{complete.cases}\NormalTok{(AED.df)}
\NormalTok{AED.corr }\OtherTok{=}\NormalTok{ AED.df[cc,]}

\CommentTok{\#remover não numérico}
\NormalTok{AED.corr}\SpecialCharTok{$}\NormalTok{Territorialidades }\OtherTok{\textless{}{-}} \ConstantTok{NULL}
\NormalTok{AED.corr}\SpecialCharTok{$}\NormalTok{ano }\OtherTok{\textless{}{-}} \ConstantTok{NULL}

\DocumentationTok{\#\#\# Matriz de corelação com Corrplot}

\NormalTok{Matrix }\OtherTok{\textless{}{-}}\FunctionTok{cor}\NormalTok{(AED.corr)}
\FunctionTok{corrplot}\NormalTok{(Matrix, }\AttributeTok{type=}\StringTok{"upper"}\NormalTok{, }\AttributeTok{order=}\StringTok{"hclust"}\NormalTok{, }\AttributeTok{method=}\StringTok{"pie"}\NormalTok{,}
         \AttributeTok{col=}\FunctionTok{brewer.pal}\NormalTok{(}\AttributeTok{n=}\DecValTok{8}\NormalTok{, }\AttributeTok{name=}\StringTok{"RdYlBu"}\NormalTok{),}
         \AttributeTok{title=}\StringTok{"Matriz de Correlação dos Dados"}\NormalTok{)}

\CommentTok{\# Adicionar uma nota de rodapé}
\FunctionTok{mtext}\NormalTok{(}\StringTok{"Elaboração própria com dados do Atlas Brasil"}\NormalTok{, }\AttributeTok{side=}\DecValTok{1}\NormalTok{, }\AttributeTok{line=}\DecValTok{4}\NormalTok{, }\AttributeTok{cex=}\FloatTok{1.2}\NormalTok{)}
\end{Highlighting}
\end{Shaded}

\includegraphics{ML-regressão-IDHM_files/figure-latex/Código para limpar colunas não numéricas e montagem da Matriz de Correlação-1.pdf}

\begin{Shaded}
\begin{Highlighting}[]
\CommentTok{\# gini, theil e população tem relação fraca com IDHM. mas irei manter para análise}

\CommentTok{\#adaptação do código desenvolvido por Anne (2020), Irizarry (2019) capítulo 17.}
\end{Highlighting}
\end{Shaded}

Os indicadores foram organizados através de um agrupamento hierárquico e
a correlação entre eles foi representada por gráficos de pizza, onde as
cores variam desde o vermelho, denotando correlação negativa, até o azul
indicando correlação positiva. Este gráfico permitiu uma rápida
compreensão das relações entre os indicadores em estudo.

Os índices de GINI e o de Theil apresentaram uma fraca correlação com o
IDHM, como também a população total também apresentou fraca correlação.
Os índices de Gini e Theil apresentaram correlação pouco além que - 0,25
para se enquadrar em uma outra paleta de cor de - 0,25 à -0,50 na
visualização.

O mesmo ocorreu com a população total, que teve a menor correlação,
embora positiva dessa vez, sua correlação ficará entre 0 e 0,25,
diferentemente dos índices de Gini e Theil que apresentaram uma maior
correlação na Matriz de correlação.

O seguinte passo foi a criação de um data frame único nomeado de
predict.IDHM unindo todos os indicadores no mesmo data frame com a
função ``join''. Cada indicador foi adicionado com a indexação por ano e
territorialidade. Seguindo a montagem do data frame único realizou a
limpeza do data frame predic.IDHM. Inicialmente, exclui a coluna
``Territorialidades'', e filtra e as linhas sem valores. A estrutura
final do data frame é então observada com a função ``str'' para oferecer
uma visão geral do data frame.

\begin{Shaded}
\begin{Highlighting}[]
\CommentTok{\# Montar um df para (IDHM \textasciitilde{} renda per capita, sub. ind. ecolaridade, sub. nd. frequencia escolar e esperança de vida.)}
\NormalTok{predic.IDHM }\OtherTok{\textless{}{-}}\NormalTok{ renda\_per\_capita}
\NormalTok{predic.IDHM }\OtherTok{\textless{}{-}} \FunctionTok{join}\NormalTok{(predic.IDHM, sub\_esco\_pop, }\AttributeTok{by =} \FunctionTok{c}\NormalTok{(}\StringTok{"ano"} \OtherTok{=} \StringTok{"ano"}\NormalTok{, }\StringTok{"Territorialidades"} \OtherTok{=} \StringTok{"Territorialidades"}\NormalTok{))}
\NormalTok{predic.IDHM }\OtherTok{\textless{}{-}} \FunctionTok{join}\NormalTok{(predic.IDHM, sub\_freq\_esco, }\AttributeTok{by =} \FunctionTok{c}\NormalTok{(}\StringTok{"ano"} \OtherTok{=} \StringTok{"ano"}\NormalTok{, }\StringTok{"Territorialidades"} \OtherTok{=} \StringTok{"Territorialidades"}\NormalTok{))}
\NormalTok{predic.IDHM }\OtherTok{\textless{}{-}} \FunctionTok{join}\NormalTok{(predic.IDHM, esperança\_de\_vida, }\AttributeTok{by =} \FunctionTok{c}\NormalTok{(}\StringTok{"ano"} \OtherTok{=} \StringTok{"ano"}\NormalTok{, }\StringTok{"Territorialidades"} \OtherTok{=} \StringTok{"Territorialidades"}\NormalTok{))}
\NormalTok{predic.IDHM }\OtherTok{\textless{}{-}} \FunctionTok{join}\NormalTok{(predic.IDHM, porcent\_pobres, }\AttributeTok{by =} \FunctionTok{c}\NormalTok{(}\StringTok{"ano"} \OtherTok{=} \StringTok{"ano"}\NormalTok{, }\StringTok{"Territorialidades"} \OtherTok{=} \StringTok{"Territorialidades"}\NormalTok{))}
\NormalTok{predic.IDHM }\OtherTok{\textless{}{-}} \FunctionTok{join}\NormalTok{(predic.IDHM, população\_total, }\AttributeTok{by =} \FunctionTok{c}\NormalTok{(}\StringTok{"ano"} \OtherTok{=} \StringTok{"ano"}\NormalTok{, }\StringTok{"Territorialidades"} \OtherTok{=} \StringTok{"Territorialidades"}\NormalTok{))}
\NormalTok{predic.IDHM }\OtherTok{\textless{}{-}} \FunctionTok{join}\NormalTok{(predic.IDHM, mortalidade\_infantil, }\AttributeTok{by =} \FunctionTok{c}\NormalTok{(}\StringTok{"ano"} \OtherTok{=} \StringTok{"ano"}\NormalTok{, }\StringTok{"Territorialidades"} \OtherTok{=} \StringTok{"Territorialidades"}\NormalTok{))}
\NormalTok{predic.IDHM }\OtherTok{\textless{}{-}} \FunctionTok{join}\NormalTok{(predic.IDHM, media\_anos\_de\_estudo, }\AttributeTok{by =} \FunctionTok{c}\NormalTok{(}\StringTok{"ano"} \OtherTok{=} \StringTok{"ano"}\NormalTok{, }\StringTok{"Territorialidades"} \OtherTok{=} \StringTok{"Territorialidades"}\NormalTok{))}
\NormalTok{predic.IDHM }\OtherTok{\textless{}{-}} \FunctionTok{join}\NormalTok{(predic.IDHM, indice\_gini, }\AttributeTok{by =} \FunctionTok{c}\NormalTok{(}\StringTok{"ano"} \OtherTok{=} \StringTok{"ano"}\NormalTok{, }\StringTok{"Territorialidades"} \OtherTok{=} \StringTok{"Territorialidades"}\NormalTok{))}
\NormalTok{predic.IDHM }\OtherTok{\textless{}{-}} \FunctionTok{join}\NormalTok{(predic.IDHM, ind\_theil\_L, }\AttributeTok{by =} \FunctionTok{c}\NormalTok{(}\StringTok{"ano"} \OtherTok{=} \StringTok{"ano"}\NormalTok{, }\StringTok{"Territorialidades"} \OtherTok{=} \StringTok{"Territorialidades"}\NormalTok{))}
\NormalTok{predic.IDHM }\OtherTok{\textless{}{-}} \FunctionTok{join}\NormalTok{(predic.IDHM, analfabetismo\_25\_anos, }\AttributeTok{by =} \FunctionTok{c}\NormalTok{(}\StringTok{"ano"} \OtherTok{=} \StringTok{"ano"}\NormalTok{, }\StringTok{"Territorialidades"} \OtherTok{=} \StringTok{"Territorialidades"}\NormalTok{))}
\NormalTok{predic.IDHM }\OtherTok{\textless{}{-}} \FunctionTok{join}\NormalTok{(predic.IDHM, analfabetismo\_18\_anos, }\AttributeTok{by =} \FunctionTok{c}\NormalTok{(}\StringTok{"ano"} \OtherTok{=} \StringTok{"ano"}\NormalTok{, }\StringTok{"Territorialidades"} \OtherTok{=} \StringTok{"Territorialidades"}\NormalTok{))}
\NormalTok{predic.IDHM }\OtherTok{\textless{}{-}} \FunctionTok{join}\NormalTok{(predic.IDHM, analfabetismo\_15\_anos, }\AttributeTok{by =} \FunctionTok{c}\NormalTok{(}\StringTok{"ano"} \OtherTok{=} \StringTok{"ano"}\NormalTok{, }\StringTok{"Territorialidades"} \OtherTok{=} \StringTok{"Territorialidades"}\NormalTok{))}
\NormalTok{predic.IDHM }\OtherTok{\textless{}{-}} \FunctionTok{join}\NormalTok{(predic.IDHM, IDHM, }\AttributeTok{by =} \FunctionTok{c}\NormalTok{(}\StringTok{"ano"} \OtherTok{=} \StringTok{"ano"}\NormalTok{, }\StringTok{"Territorialidades"} \OtherTok{=} \StringTok{"Territorialidades"}\NormalTok{))}
\NormalTok{predic.IDHM}\SpecialCharTok{$}\NormalTok{Territorialidades }\OtherTok{\textless{}{-}} \ConstantTok{NULL} 
\NormalTok{cc }\OtherTok{=} \FunctionTok{complete.cases}\NormalTok{(predic.IDHM)}
\NormalTok{predic.IDHM }\OtherTok{=}\NormalTok{ predic.IDHM[cc,]}
\FunctionTok{str}\NormalTok{(predic.IDHM)}
\end{Highlighting}
\end{Shaded}

\begin{verbatim}
## 'data.frame':    280 obs. of  15 variables:
##  $ ano                  : num  2012 2012 2012 2012 2012 ...
##  $ renda_per_capita     : num  759 517 395 528 559 ...
##  $ sub_esco_pop         : num  0.606 0.59 0.487 0.67 0.613 0.51 0.54 0.765 0.613 0.619 ...
##  $ sub_freq_esco        : num  0.731 0.681 0.645 0.653 0.642 0.639 0.742 0.77 0.735 0.741 ...
##  $ esperança_de_vida    : num  74.5 72.5 70 72.8 70.8 ...
##  $ porcent_pobres       : num  11.4 23.8 23.4 18.4 22.2 ...
##  $ população_total      : num  1.98e+08 7.77e+05 3.22e+06 7.21e+05 3.54e+06 ...
##  $ mortalidade_infantil : num  15.8 20.2 26.1 24.3 20.9 ...
##  $ media_anos_de_estudo : num  8.56 7.72 6.8 9.09 8.63 ...
##  $ indice_gini          : num  0.54 0.566 0.503 0.528 0.589 0.563 0.545 0.601 0.489 0.474 ...
##  $ ind_theil_L          : num  0.526 0.585 0.447 0.483 0.619 0.571 0.54 0.664 0.411 0.383 ...
##  $ analfabetismo_25_anos: num  10.22 18.22 24.22 7.93 9.46 ...
##  $ analfabetismo_18_anos: num  8.75 14.72 20.54 6.37 7.89 ...
##  $ analfabetismo_15_anos: num  8.21 13.48 18.97 5.76 7.22 ...
##  $ IDHM                 : num  0.746 0.701 0.651 0.707 0.691 0.678 0.701 0.825 0.758 0.744 ...
\end{verbatim}

\begin{Shaded}
\begin{Highlighting}[]
\CommentTok{\#daptação do código desenvolvido por Anne (2020) e Irizarry ( 2019).}
\end{Highlighting}
\end{Shaded}

Um gráfico de dispersão matriz para visualizar as relações entre todas
as variáveis ajuda a compreender a relação dos dados do data frame
predic.IDHM. Este tipo de visualização é útil para uma rápida avaliação
das inter-relações entre múltiplas variáveis. No entanto, uma
desvantagem notável de um ``scatterplot matrix'' geral é a potencial
sobrecarga visual, especialmente quando há muitas variáveis. Isso pode
tornar difícil discernir padrões claros ou tendências específicas.

Em contraste, ao plotar o IDHM contra cada variável individualmente, é
possível obter uma visão mais clara e focada da relação específica entre
o IDHM e cada variável. Isso facilita a identificação de padrões e
correlações diretamente relevantes para o IDHM, ao invés de se perder no
emaranhado de todas as possíveis relações entre variáveis.

Em Seguida foi feito ambos os modelos para melhorar o detalhamento da
análise exploratória de dados aplicada no exemplo do IDHM.

\begin{Shaded}
\begin{Highlighting}[]
\DocumentationTok{\#\#\# Plotar a correlação como scatterplot matrix.}
\CommentTok{\# Criação do scatterplot sem título principal}
\FunctionTok{plot}\NormalTok{(predic.IDHM, }\AttributeTok{pch=}\DecValTok{1}\NormalTok{, }\AttributeTok{cex=}\NormalTok{.}\DecValTok{2}\NormalTok{, }\AttributeTok{col=}\FunctionTok{rgb}\NormalTok{(}\DecValTok{0}\NormalTok{,}\DecValTok{0}\NormalTok{,}\DecValTok{0}\NormalTok{,}\FloatTok{0.4}\NormalTok{), }\AttributeTok{main=}\StringTok{""}\NormalTok{)}

\CommentTok{\# Adição do título, subtítulo e fonte}
\FunctionTok{mtext}\NormalTok{(}\StringTok{"Gráfico 2 {-} Matrix Scatterplot das variáveis com correlação com o IDHM"}\NormalTok{, }\AttributeTok{side=}\DecValTok{3}\NormalTok{, }\AttributeTok{line=}\DecValTok{3}\NormalTok{, }\AttributeTok{adj=}\DecValTok{0}\NormalTok{) }\CommentTok{\# Título}
\FunctionTok{mtext}\NormalTok{(}\StringTok{"Análise baseada em dados do Atlas Brasil"}\NormalTok{, }\AttributeTok{side=}\DecValTok{3}\NormalTok{, }\AttributeTok{line=}\DecValTok{2}\NormalTok{, }\AttributeTok{adj=}\DecValTok{0}\NormalTok{) }\CommentTok{\# Subtítulo}
\FunctionTok{mtext}\NormalTok{(}\StringTok{"Fonte: Elaboração própria com dados do Atlas Brasil"}\NormalTok{, }\AttributeTok{side=}\DecValTok{1}\NormalTok{, }\AttributeTok{line=}\DecValTok{4}\NormalTok{, }\AttributeTok{adj=}\DecValTok{0}\NormalTok{, }\AttributeTok{cex=}\FloatTok{0.8}\NormalTok{, }\AttributeTok{col=}\StringTok{"black"}\NormalTok{, }\AttributeTok{font=}\DecValTok{3}\NormalTok{) }\CommentTok{\# Fonte}
\end{Highlighting}
\end{Shaded}

\includegraphics{ML-regressão-IDHM_files/figure-latex/Código para plotagem de Matriz Scatterplot-1.pdf}

\begin{Shaded}
\begin{Highlighting}[]
\CommentTok{\#Elaboração própria com os dados do IDHM obtidos no portal Atlas Brasil advindos do IBGE e Registros Administrativos.}
\end{Highlighting}
\end{Shaded}

Seguindo com a utilização dos scatterplots, uma medida para entender as
relações lineares que as variáveis empenham ao IDHM é a utilização do
scatterplot com linha de tendência para as variáveis, e a regressão
linear simples das variáveis com o IDHM. O Apêndice D incluí o código e
resultado para as demais variáveis além da renda per capita.

\begin{Shaded}
\begin{Highlighting}[]
\CommentTok{\# Criar um scatterplot com linha de regressão para IDHM e renda per capita}
\FunctionTok{ggscatter}\NormalTok{(predic.IDHM, }\AttributeTok{x =} \StringTok{"renda\_per\_capita"}\NormalTok{, }\AttributeTok{y =} \StringTok{"IDHM"}\NormalTok{, }
          \AttributeTok{color=}\FunctionTok{rgb}\NormalTok{(}\DecValTok{0}\NormalTok{,.}\DecValTok{2}\NormalTok{,.}\DecValTok{5}\NormalTok{,  }\DecValTok{1}\NormalTok{), }\AttributeTok{pch=}\DecValTok{1}\NormalTok{, }\AttributeTok{add =} \StringTok{"reg.line"}\NormalTok{, }
          \AttributeTok{add.params =} \FunctionTok{list}\NormalTok{(}\AttributeTok{color=}\FunctionTok{rgb}\NormalTok{(}\DecValTok{0}\NormalTok{,.}\DecValTok{4}\NormalTok{,.}\DecValTok{4}\NormalTok{,  }\DecValTok{1}\NormalTok{), }\AttributeTok{fill =} \StringTok{"light gray"}\NormalTok{), }
          \AttributeTok{conf.int =} \ConstantTok{TRUE}\NormalTok{, }\AttributeTok{main =} \StringTok{"Relação entre IDHM e Renda Per Capita"}\NormalTok{) }\SpecialCharTok{+}
  \FunctionTok{labs}\NormalTok{(}\AttributeTok{title =} \StringTok{"Gráfico 3 {-} Relação entre IDHM e Renda Per Capita"}\NormalTok{,}
       \AttributeTok{subtitle =} \StringTok{"Análise baseada em dados do Atlas Brasil"}\NormalTok{,}
       \AttributeTok{caption =} \StringTok{"Fonte: Elaboração própria com dados do Atlas Brasil"}\NormalTok{) }\SpecialCharTok{+}
  \FunctionTok{theme}\NormalTok{(}\AttributeTok{plot.caption =} \FunctionTok{element\_text}\NormalTok{(}\AttributeTok{hjust =} \DecValTok{0}\NormalTok{, }\AttributeTok{face=}\StringTok{"italic"}\NormalTok{))}
\end{Highlighting}
\end{Shaded}

\begin{verbatim}
## Warning: Duplicated aesthetics after name standardisation: shape
\end{verbatim}

\includegraphics{ML-regressão-IDHM_files/figure-latex/scatterplot com linha de regressão e os resultados-1.pdf}

\begin{Shaded}
\begin{Highlighting}[]
\CommentTok{\# Printar estatística F para ver a significacia da regressão}
\FunctionTok{summary}\NormalTok{(}\FunctionTok{lm}\NormalTok{(IDHM }\SpecialCharTok{\textasciitilde{}}\NormalTok{ renda\_per\_capita, }\AttributeTok{data=}\NormalTok{predic.IDHM))}
\end{Highlighting}
\end{Shaded}

\begin{verbatim}
## 
## Call:
## lm(formula = IDHM ~ renda_per_capita, data = predic.IDHM)
## 
## Residuals:
##       Min        1Q    Median        3Q       Max 
## -0.063202 -0.012445  0.001026  0.015348  0.036847 
## 
## Coefficients:
##                   Estimate Std. Error t value Pr(>|t|)    
## (Intercept)      6.256e-01  3.516e-03  177.91   <2e-16 ***
## renda_per_capita 1.699e-04  4.842e-06   35.08   <2e-16 ***
## ---
## Signif. codes:  0 '***' 0.001 '**' 0.01 '*' 0.05 '.' 0.1 ' ' 1
## 
## Residual standard error: 0.01965 on 278 degrees of freedom
## Multiple R-squared:  0.8157, Adjusted R-squared:  0.8151 
## F-statistic:  1231 on 1 and 278 DF,  p-value: < 2.2e-16
\end{verbatim}

\begin{Shaded}
\begin{Highlighting}[]
\CommentTok{\# Criar um scatterplot com linha de regressão para IDHM e Subindice de Escolaridade}
\FunctionTok{ggscatter}\NormalTok{(predic.IDHM, }\AttributeTok{x =} \StringTok{"sub\_esco\_pop"}\NormalTok{, }\AttributeTok{y =} \StringTok{"IDHM"}\NormalTok{, }
          \AttributeTok{color=}\FunctionTok{rgb}\NormalTok{(}\DecValTok{0}\NormalTok{,.}\DecValTok{2}\NormalTok{,.}\DecValTok{5}\NormalTok{,  }\DecValTok{1}\NormalTok{), }\AttributeTok{pch=}\DecValTok{1}\NormalTok{, }\AttributeTok{add =} \StringTok{"reg.line"}\NormalTok{, }
          \AttributeTok{add.params =} \FunctionTok{list}\NormalTok{(}\AttributeTok{color=}\FunctionTok{rgb}\NormalTok{(}\DecValTok{0}\NormalTok{,.}\DecValTok{4}\NormalTok{,.}\DecValTok{4}\NormalTok{,  }\DecValTok{1}\NormalTok{), }\AttributeTok{fill =} \StringTok{"light gray"}\NormalTok{), }
          \AttributeTok{conf.int =} \ConstantTok{TRUE}\NormalTok{, }\AttributeTok{main =} \StringTok{"Relação entre IDHM e Subindice de Escolaridade"}\NormalTok{) }\SpecialCharTok{+}
  \FunctionTok{labs}\NormalTok{(}\AttributeTok{title =} \StringTok{" Gráfico 3.1 {-} Relação entre IDHM e Subindice de Escolaridade"}\NormalTok{,}
          \AttributeTok{subtitle =} \StringTok{"Análise baseada em dados do Atlas Brasil"}\NormalTok{,}
          \AttributeTok{caption =} \StringTok{"Fonte: Elaboração própria com dados do Atlas Brasil"}\NormalTok{) }\SpecialCharTok{+}
  \FunctionTok{theme}\NormalTok{(}\AttributeTok{plot.caption =} \FunctionTok{element\_text}\NormalTok{(}\AttributeTok{hjust =} \DecValTok{0}\NormalTok{, }\AttributeTok{face=}\StringTok{"italic"}\NormalTok{))}
\end{Highlighting}
\end{Shaded}

\begin{verbatim}
## Warning: Duplicated aesthetics after name standardisation: shape
\end{verbatim}

\includegraphics{ML-regressão-IDHM_files/figure-latex/scatterplot com linha de regressão e os resultados-2.pdf}

\begin{Shaded}
\begin{Highlighting}[]
\CommentTok{\# Printar estatística F para ver a significacia da regressão}
\FunctionTok{summary}\NormalTok{(}\FunctionTok{lm}\NormalTok{(IDHM }\SpecialCharTok{\textasciitilde{}}\NormalTok{ sub\_esco\_pop, }\AttributeTok{data=}\NormalTok{predic.IDHM))}
\end{Highlighting}
\end{Shaded}

\begin{verbatim}
## 
## Call:
## lm(formula = IDHM ~ sub_esco_pop, data = predic.IDHM)
## 
## Residuals:
##       Min        1Q    Median        3Q       Max 
## -0.101216 -0.017439 -0.000147  0.020216  0.052755 
## 
## Coefficients:
##              Estimate Std. Error t value Pr(>|t|)    
## (Intercept)   0.45495    0.01448   31.41   <2e-16 ***
## sub_esco_pop  0.45417    0.02276   19.96   <2e-16 ***
## ---
## Signif. codes:  0 '***' 0.001 '**' 0.01 '*' 0.05 '.' 0.1 ' ' 1
## 
## Residual standard error: 0.02935 on 278 degrees of freedom
## Multiple R-squared:  0.5889, Adjusted R-squared:  0.5874 
## F-statistic: 398.2 on 1 and 278 DF,  p-value: < 2.2e-16
\end{verbatim}

\begin{Shaded}
\begin{Highlighting}[]
\CommentTok{\# Criar um scatterplot com linha de regressão para IDHM e Subindice de frequência Escolar}
\FunctionTok{ggscatter}\NormalTok{(predic.IDHM, }\AttributeTok{x =} \StringTok{"sub\_freq\_esco"}\NormalTok{, }\AttributeTok{y =} \StringTok{"IDHM"}\NormalTok{, }
          \AttributeTok{color=}\FunctionTok{rgb}\NormalTok{(}\DecValTok{0}\NormalTok{,.}\DecValTok{2}\NormalTok{,.}\DecValTok{5}\NormalTok{,  }\DecValTok{1}\NormalTok{), }\AttributeTok{pch=}\DecValTok{1}\NormalTok{, }\AttributeTok{add =} \StringTok{"reg.line"}\NormalTok{, }
          \AttributeTok{add.params =} \FunctionTok{list}\NormalTok{(}\AttributeTok{color=}\FunctionTok{rgb}\NormalTok{(}\DecValTok{0}\NormalTok{,.}\DecValTok{4}\NormalTok{,.}\DecValTok{4}\NormalTok{,  }\DecValTok{1}\NormalTok{), }\AttributeTok{fill =} \StringTok{"light gray"}\NormalTok{), }
          \AttributeTok{conf.int =} \ConstantTok{TRUE}\NormalTok{, }\AttributeTok{main =} \StringTok{"Relação entre IDHM e Subindice de Frequência Escolar"}\NormalTok{) }\SpecialCharTok{+}
  \FunctionTok{labs}\NormalTok{(}\AttributeTok{title =} \StringTok{"Gráfico 3.2 {-} Relação entre IDHM e Subindice de Frequência Escolar"}\NormalTok{,}
       \AttributeTok{subtitle =} \StringTok{"Análise baseada em dados do Atlas Brasil"}\NormalTok{,}
       \AttributeTok{caption =} \StringTok{"Fonte: Elaboração própria com dados do Atlas Brasil"}\NormalTok{) }\SpecialCharTok{+}
  \FunctionTok{theme}\NormalTok{(}\AttributeTok{plot.caption =} \FunctionTok{element\_text}\NormalTok{(}\AttributeTok{hjust =} \DecValTok{0}\NormalTok{, }\AttributeTok{face=}\StringTok{"italic"}\NormalTok{))}
\end{Highlighting}
\end{Shaded}

\begin{verbatim}
## Warning: Duplicated aesthetics after name standardisation: shape
\end{verbatim}

\includegraphics{ML-regressão-IDHM_files/figure-latex/scatterplot com linha de regressão e os resultados-3.pdf}

\begin{Shaded}
\begin{Highlighting}[]
\CommentTok{\# Printar estatística F para ver a significacia da regressão}
\FunctionTok{summary}\NormalTok{(}\FunctionTok{lm}\NormalTok{(IDHM }\SpecialCharTok{\textasciitilde{}}\NormalTok{ sub\_freq\_esco, }\AttributeTok{data=}\NormalTok{predic.IDHM))}
\end{Highlighting}
\end{Shaded}

\begin{verbatim}
## 
## Call:
## lm(formula = IDHM ~ sub_freq_esco, data = predic.IDHM)
## 
## Residuals:
##       Min        1Q    Median        3Q       Max 
## -0.076437 -0.016621 -0.001193  0.016278  0.077382 
## 
## Coefficients:
##               Estimate Std. Error t value Pr(>|t|)    
## (Intercept)    0.24977    0.02139   11.68   <2e-16 ***
## sub_freq_esco  0.65281    0.02830   23.07   <2e-16 ***
## ---
## Signif. codes:  0 '***' 0.001 '**' 0.01 '*' 0.05 '.' 0.1 ' ' 1
## 
## Residual standard error: 0.02682 on 278 degrees of freedom
## Multiple R-squared:  0.6569, Adjusted R-squared:  0.6556 
## F-statistic: 532.2 on 1 and 278 DF,  p-value: < 2.2e-16
\end{verbatim}

\begin{Shaded}
\begin{Highlighting}[]
\CommentTok{\# Criar um scatterplot com linha de regressão para IDHM e Esperança de Vida}
\FunctionTok{ggscatter}\NormalTok{(predic.IDHM, }\AttributeTok{x =} \StringTok{"esperança\_de\_vida"}\NormalTok{, }\AttributeTok{y =} \StringTok{"IDHM"}\NormalTok{, }
          \AttributeTok{color=}\FunctionTok{rgb}\NormalTok{(}\DecValTok{0}\NormalTok{,.}\DecValTok{2}\NormalTok{,.}\DecValTok{5}\NormalTok{,  }\DecValTok{1}\NormalTok{), }\AttributeTok{pch=}\DecValTok{1}\NormalTok{, }\AttributeTok{add =} \StringTok{"reg.line"}\NormalTok{, }
          \AttributeTok{add.params =} \FunctionTok{list}\NormalTok{(}\AttributeTok{color=}\FunctionTok{rgb}\NormalTok{(}\DecValTok{0}\NormalTok{,.}\DecValTok{4}\NormalTok{,.}\DecValTok{4}\NormalTok{,  }\DecValTok{1}\NormalTok{), }\AttributeTok{fill =} \StringTok{"light gray"}\NormalTok{), }
          \AttributeTok{conf.int =} \ConstantTok{TRUE}\NormalTok{, }\AttributeTok{main =} \StringTok{"Relação entre IDHM e Esperança de Vida"}\NormalTok{) }\SpecialCharTok{+}
  \FunctionTok{labs}\NormalTok{(}\AttributeTok{title =} \StringTok{"Gráfico 3.3 {-} Relação entre IDHM e Esperança de Vida"}\NormalTok{,}
       \AttributeTok{subtitle =} \StringTok{"Análise baseada em dados do Atlas Brasil"}\NormalTok{,}
       \AttributeTok{caption =} \StringTok{"Fonte: Elaboração própria com dados do Atlas Brasil"}\NormalTok{) }\SpecialCharTok{+}
  \FunctionTok{theme}\NormalTok{(}\AttributeTok{plot.caption =} \FunctionTok{element\_text}\NormalTok{(}\AttributeTok{hjust =} \DecValTok{0}\NormalTok{, }\AttributeTok{face=}\StringTok{"italic"}\NormalTok{))}
\end{Highlighting}
\end{Shaded}

\begin{verbatim}
## Warning: Duplicated aesthetics after name standardisation: shape
\end{verbatim}

\includegraphics{ML-regressão-IDHM_files/figure-latex/scatterplot com linha de regressão e os resultados-4.pdf}

\begin{Shaded}
\begin{Highlighting}[]
\CommentTok{\# Printar estatística F para ver a significacia da regressão}
\FunctionTok{summary}\NormalTok{(}\FunctionTok{lm}\NormalTok{(IDHM }\SpecialCharTok{\textasciitilde{}}\NormalTok{ esperança\_de\_vida, }\AttributeTok{data=}\NormalTok{predic.IDHM))}
\end{Highlighting}
\end{Shaded}

\begin{verbatim}
## 
## Call:
## lm(formula = IDHM ~ esperança_de_vida, data = predic.IDHM)
## 
## Residuals:
##       Min        1Q    Median        3Q       Max 
## -0.056084 -0.019263 -0.005752  0.015072  0.083536 
## 
## Coefficients:
##                    Estimate Std. Error t value Pr(>|t|)    
## (Intercept)       -0.324437   0.044098  -7.357 2.12e-12 ***
## esperança_de_vida  0.014419   0.000596  24.194  < 2e-16 ***
## ---
## Signif. codes:  0 '***' 0.001 '**' 0.01 '*' 0.05 '.' 0.1 ' ' 1
## 
## Residual standard error: 0.02598 on 278 degrees of freedom
## Multiple R-squared:  0.678,  Adjusted R-squared:  0.6768 
## F-statistic: 585.4 on 1 and 278 DF,  p-value: < 2.2e-16
\end{verbatim}

\begin{Shaded}
\begin{Highlighting}[]
\CommentTok{\# Criar um scatterplot com linha de regressão para IDHM e Percentual de Pobres na População}
\FunctionTok{ggscatter}\NormalTok{(predic.IDHM, }\AttributeTok{x =} \StringTok{"porcent\_pobres"}\NormalTok{, }\AttributeTok{y =} \StringTok{"IDHM"}\NormalTok{, }
          \AttributeTok{color=}\FunctionTok{rgb}\NormalTok{(}\DecValTok{0}\NormalTok{,.}\DecValTok{2}\NormalTok{,.}\DecValTok{5}\NormalTok{,  }\DecValTok{1}\NormalTok{), }\AttributeTok{pch=}\DecValTok{1}\NormalTok{, }\AttributeTok{add =} \StringTok{"reg.line"}\NormalTok{, }
          \AttributeTok{add.params =} \FunctionTok{list}\NormalTok{(}\AttributeTok{color=}\FunctionTok{rgb}\NormalTok{(}\DecValTok{0}\NormalTok{,.}\DecValTok{4}\NormalTok{,.}\DecValTok{4}\NormalTok{,  }\DecValTok{1}\NormalTok{), }\AttributeTok{fill =} \StringTok{"light gray"}\NormalTok{), }
          \AttributeTok{conf.int =} \ConstantTok{TRUE}\NormalTok{, }\AttributeTok{main =} \StringTok{"Relação entre IDHM e Percentual de Pobres na População"}\NormalTok{) }\SpecialCharTok{+}
  \FunctionTok{labs}\NormalTok{(}\AttributeTok{title =} \StringTok{"Gráfico 3.3 {-} Relação entre IDHM e Percentual de Pobres na População"}\NormalTok{,}
       \AttributeTok{subtitle =} \StringTok{"Análise baseada em dados do Atlas Brasil"}\NormalTok{,}
       \AttributeTok{caption =} \StringTok{"Fonte: Elaboração própria com dados do Atlas Brasil"}\NormalTok{) }\SpecialCharTok{+}
  \FunctionTok{theme}\NormalTok{(}\AttributeTok{plot.caption =} \FunctionTok{element\_text}\NormalTok{(}\AttributeTok{hjust =} \DecValTok{0}\NormalTok{, }\AttributeTok{face=}\StringTok{"italic"}\NormalTok{))}
\end{Highlighting}
\end{Shaded}

\begin{verbatim}
## Warning: Duplicated aesthetics after name standardisation: shape
\end{verbatim}

\includegraphics{ML-regressão-IDHM_files/figure-latex/scatterplot com linha de regressão e os resultados-5.pdf}

\begin{Shaded}
\begin{Highlighting}[]
\CommentTok{\# Printar estatística F para ver a significacia da regressão}
\FunctionTok{summary}\NormalTok{(}\FunctionTok{lm}\NormalTok{(IDHM }\SpecialCharTok{\textasciitilde{}}\NormalTok{ porcent\_pobres, }\AttributeTok{data=}\NormalTok{predic.IDHM))}
\end{Highlighting}
\end{Shaded}

\begin{verbatim}
## 
## Call:
## lm(formula = IDHM ~ porcent_pobres, data = predic.IDHM)
## 
## Residuals:
##       Min        1Q    Median        3Q       Max 
## -0.074741 -0.016308 -0.000152  0.016863  0.069873 
## 
## Coefficients:
##                  Estimate Std. Error t value Pr(>|t|)    
## (Intercept)     0.8054598  0.0030056  267.99   <2e-16 ***
## porcent_pobres -0.0046933  0.0001906  -24.62   <2e-16 ***
## ---
## Signif. codes:  0 '***' 0.001 '**' 0.01 '*' 0.05 '.' 0.1 ' ' 1
## 
## Residual standard error: 0.02567 on 278 degrees of freedom
## Multiple R-squared:  0.6856, Adjusted R-squared:  0.6844 
## F-statistic: 606.1 on 1 and 278 DF,  p-value: < 2.2e-16
\end{verbatim}

\begin{Shaded}
\begin{Highlighting}[]
\CommentTok{\# Criar um scatterplot com linha de regressão para IDHM e População Total}
\FunctionTok{ggscatter}\NormalTok{(predic.IDHM, }\AttributeTok{x =} \StringTok{"população\_total"}\NormalTok{, }\AttributeTok{y =} \StringTok{"IDHM"}\NormalTok{, }
          \AttributeTok{color=}\FunctionTok{rgb}\NormalTok{(}\DecValTok{0}\NormalTok{,.}\DecValTok{2}\NormalTok{,.}\DecValTok{5}\NormalTok{,  }\DecValTok{1}\NormalTok{), }\AttributeTok{pch=}\DecValTok{1}\NormalTok{, }\AttributeTok{add =} \StringTok{"reg.line"}\NormalTok{, }
          \AttributeTok{add.params =} \FunctionTok{list}\NormalTok{(}\AttributeTok{color=}\FunctionTok{rgb}\NormalTok{(}\DecValTok{0}\NormalTok{,.}\DecValTok{4}\NormalTok{,.}\DecValTok{4}\NormalTok{,  }\DecValTok{1}\NormalTok{), }\AttributeTok{fill =} \StringTok{"light gray"}\NormalTok{), }
          \AttributeTok{conf.int =} \ConstantTok{TRUE}\NormalTok{, }\AttributeTok{main =} \StringTok{"Relação entre IDHM e População Total"}\NormalTok{) }\SpecialCharTok{+}
  \FunctionTok{labs}\NormalTok{(}\AttributeTok{title =} \StringTok{"Gráfico 3.4 {-} Relação entre IDHM e População Total"}\NormalTok{,}
       \AttributeTok{subtitle =} \StringTok{"Análise baseada em dados do Atlas Brasil"}\NormalTok{,}
       \AttributeTok{caption =} \StringTok{"Fonte: Elaboração própria com dados do Atlas Brasil"}\NormalTok{) }\SpecialCharTok{+}
  \FunctionTok{theme}\NormalTok{(}\AttributeTok{plot.caption =} \FunctionTok{element\_text}\NormalTok{(}\AttributeTok{hjust =} \DecValTok{0}\NormalTok{, }\AttributeTok{face=}\StringTok{"italic"}\NormalTok{))}
\end{Highlighting}
\end{Shaded}

\begin{verbatim}
## Warning: Duplicated aesthetics after name standardisation: shape
\end{verbatim}

\includegraphics{ML-regressão-IDHM_files/figure-latex/scatterplot com linha de regressão e os resultados-6.pdf}

\begin{Shaded}
\begin{Highlighting}[]
\CommentTok{\# Printar estatística F para ver a significacia da regressão}
\FunctionTok{summary}\NormalTok{(}\FunctionTok{lm}\NormalTok{(IDHM }\SpecialCharTok{\textasciitilde{}}\NormalTok{ população\_total, }\AttributeTok{data=}\NormalTok{predic.IDHM))}
\end{Highlighting}
\end{Shaded}

\begin{verbatim}
## 
## Call:
## lm(formula = IDHM ~ população_total, data = predic.IDHM)
## 
## Residuals:
##       Min        1Q    Median        3Q       Max 
## -0.091832 -0.035392 -0.005422  0.031361  0.120117 
## 
## Coefficients:
##                  Estimate Std. Error t value Pr(>|t|)    
## (Intercept)     7.381e-01  2.870e-03 257.141  < 2e-16 ***
## população_total 2.528e-10  7.092e-11   3.565 0.000428 ***
## ---
## Signif. codes:  0 '***' 0.001 '**' 0.01 '*' 0.05 '.' 0.1 ' ' 1
## 
## Residual standard error: 0.04477 on 278 degrees of freedom
## Multiple R-squared:  0.04372,    Adjusted R-squared:  0.04028 
## F-statistic: 12.71 on 1 and 278 DF,  p-value: 0.0004279
\end{verbatim}

\begin{Shaded}
\begin{Highlighting}[]
\CommentTok{\# Criar um scatterplot com linha de regressão para IDHM e Mortalidade Infantil}
\FunctionTok{ggscatter}\NormalTok{(predic.IDHM, }\AttributeTok{x =} \StringTok{"mortalidade\_infantil"}\NormalTok{, }\AttributeTok{y =} \StringTok{"IDHM"}\NormalTok{, }
          \AttributeTok{color=}\FunctionTok{rgb}\NormalTok{(}\DecValTok{0}\NormalTok{,.}\DecValTok{2}\NormalTok{,.}\DecValTok{5}\NormalTok{,  }\DecValTok{1}\NormalTok{), }\AttributeTok{pch=}\DecValTok{1}\NormalTok{, }\AttributeTok{add =} \StringTok{"reg.line"}\NormalTok{, }
          \AttributeTok{add.params =} \FunctionTok{list}\NormalTok{(}\AttributeTok{color=}\FunctionTok{rgb}\NormalTok{(}\DecValTok{0}\NormalTok{,.}\DecValTok{4}\NormalTok{,.}\DecValTok{4}\NormalTok{,  }\DecValTok{1}\NormalTok{), }\AttributeTok{fill =} \StringTok{"light gray"}\NormalTok{), }
          \AttributeTok{conf.int =} \ConstantTok{TRUE}\NormalTok{, }\AttributeTok{main =} \StringTok{"Relação entre IDHM e Mortalidade Infantil"}\NormalTok{) }\SpecialCharTok{+}
  \FunctionTok{labs}\NormalTok{(}\AttributeTok{title =} \StringTok{"Gráfico 3.5 {-} Relação entre IDHM e Mortalidade Infantil"}\NormalTok{,}
       \AttributeTok{subtitle =} \StringTok{"Análise baseada em dados do Atlas Brasil"}\NormalTok{,}
       \AttributeTok{caption =} \StringTok{"Fonte: Elaboração própria com dados do Atlas Brasil"}\NormalTok{) }\SpecialCharTok{+}
  \FunctionTok{theme}\NormalTok{(}\AttributeTok{plot.caption =} \FunctionTok{element\_text}\NormalTok{(}\AttributeTok{hjust =} \DecValTok{0}\NormalTok{, }\AttributeTok{face=}\StringTok{"italic"}\NormalTok{))}
\end{Highlighting}
\end{Shaded}

\begin{verbatim}
## Warning: Duplicated aesthetics after name standardisation: shape
\end{verbatim}

\includegraphics{ML-regressão-IDHM_files/figure-latex/scatterplot com linha de regressão e os resultados-7.pdf}

\begin{Shaded}
\begin{Highlighting}[]
\CommentTok{\# Printar estatística F para ver a significacia da regressão}
\FunctionTok{summary}\NormalTok{(}\FunctionTok{lm}\NormalTok{(IDHM }\SpecialCharTok{\textasciitilde{}}\NormalTok{ mortalidade\_infantil, }\AttributeTok{data=}\NormalTok{predic.IDHM))}
\end{Highlighting}
\end{Shaded}

\begin{verbatim}
## 
## Call:
## lm(formula = IDHM ~ mortalidade_infantil, data = predic.IDHM)
## 
## Residuals:
##       Min        1Q    Median        3Q       Max 
## -0.051939 -0.020627 -0.002814  0.015853  0.078314 
## 
## Coefficients:
##                        Estimate Std. Error t value Pr(>|t|)    
## (Intercept)           0.8730255  0.0061886   141.1   <2e-16 ***
## mortalidade_infantil -0.0083336  0.0003788   -22.0   <2e-16 ***
## ---
## Signif. codes:  0 '***' 0.001 '**' 0.01 '*' 0.05 '.' 0.1 ' ' 1
## 
## Residual standard error: 0.02765 on 278 degrees of freedom
## Multiple R-squared:  0.6351, Adjusted R-squared:  0.6338 
## F-statistic: 483.9 on 1 and 278 DF,  p-value: < 2.2e-16
\end{verbatim}

\begin{Shaded}
\begin{Highlighting}[]
\CommentTok{\# Criar um scatterplot com linha de regressão para IDHM e Média de Anos de Estudo}
\FunctionTok{ggscatter}\NormalTok{(predic.IDHM, }\AttributeTok{x =} \StringTok{"media\_anos\_de\_estudo"}\NormalTok{, }\AttributeTok{y =} \StringTok{"IDHM"}\NormalTok{, }
          \AttributeTok{color=}\FunctionTok{rgb}\NormalTok{(}\DecValTok{0}\NormalTok{,.}\DecValTok{2}\NormalTok{,.}\DecValTok{5}\NormalTok{,  }\DecValTok{1}\NormalTok{), }\AttributeTok{pch=}\DecValTok{1}\NormalTok{, }\AttributeTok{add =} \StringTok{"reg.line"}\NormalTok{, }
          \AttributeTok{add.params =} \FunctionTok{list}\NormalTok{(}\AttributeTok{color=}\FunctionTok{rgb}\NormalTok{(}\DecValTok{0}\NormalTok{,.}\DecValTok{4}\NormalTok{,.}\DecValTok{4}\NormalTok{,  }\DecValTok{1}\NormalTok{), }\AttributeTok{fill =} \StringTok{"light gray"}\NormalTok{), }
          \AttributeTok{conf.int =} \ConstantTok{TRUE}\NormalTok{, }\AttributeTok{main =} \StringTok{"Relação entre IDHM e Média de Anos de Estudo"}\NormalTok{) }\SpecialCharTok{+}
  \FunctionTok{labs}\NormalTok{(}\AttributeTok{title =} \StringTok{"Gráfico 3.6 {-} Relação entre IDHM e Renda Média de Anos de Estudo"}\NormalTok{,}
       \AttributeTok{subtitle =} \StringTok{"Análise baseada em dados do Atlas Brasil"}\NormalTok{,}
       \AttributeTok{caption =} \StringTok{"Fonte: Elaboração própria com dados do Atlas Brasil"}\NormalTok{) }\SpecialCharTok{+}
  \FunctionTok{theme}\NormalTok{(}\AttributeTok{plot.caption =} \FunctionTok{element\_text}\NormalTok{(}\AttributeTok{hjust =} \DecValTok{0}\NormalTok{, }\AttributeTok{face=}\StringTok{"italic"}\NormalTok{))}
\end{Highlighting}
\end{Shaded}

\begin{verbatim}
## Warning: Duplicated aesthetics after name standardisation: shape
\end{verbatim}

\includegraphics{ML-regressão-IDHM_files/figure-latex/scatterplot com linha de regressão e os resultados-8.pdf}

\begin{Shaded}
\begin{Highlighting}[]
\CommentTok{\# Printar estatística F para ver a significacia da regressão}
\FunctionTok{summary}\NormalTok{(}\FunctionTok{lm}\NormalTok{(IDHM }\SpecialCharTok{\textasciitilde{}}\NormalTok{ media\_anos\_de\_estudo, }\AttributeTok{data=}\NormalTok{predic.IDHM))}
\end{Highlighting}
\end{Shaded}

\begin{verbatim}
## 
## Call:
## lm(formula = IDHM ~ media_anos_de_estudo, data = predic.IDHM)
## 
## Residuals:
##       Min        1Q    Median        3Q       Max 
## -0.104787 -0.015780  0.001814  0.018576  0.054403 
## 
## Coefficients:
##                      Estimate Std. Error t value Pr(>|t|)    
## (Intercept)          0.437262   0.013057   33.49   <2e-16 ***
## media_anos_de_estudo 0.034350   0.001462   23.50   <2e-16 ***
## ---
## Signif. codes:  0 '***' 0.001 '**' 0.01 '*' 0.05 '.' 0.1 ' ' 1
## 
## Residual standard error: 0.02649 on 278 degrees of freedom
## Multiple R-squared:  0.6652, Adjusted R-squared:  0.664 
## F-statistic: 552.3 on 1 and 278 DF,  p-value: < 2.2e-16
\end{verbatim}

\begin{Shaded}
\begin{Highlighting}[]
\CommentTok{\# Criar um scatterplot com linha de regressão para IDHM e Índice de GINI}
\FunctionTok{ggscatter}\NormalTok{(predic.IDHM, }\AttributeTok{x =} \StringTok{"indice\_gini"}\NormalTok{, }\AttributeTok{y =} \StringTok{"IDHM"}\NormalTok{, }
          \AttributeTok{color=}\FunctionTok{rgb}\NormalTok{(}\DecValTok{0}\NormalTok{,.}\DecValTok{2}\NormalTok{,.}\DecValTok{5}\NormalTok{,  }\DecValTok{1}\NormalTok{), }\AttributeTok{pch=}\DecValTok{1}\NormalTok{, }\AttributeTok{add =} \StringTok{"reg.line"}\NormalTok{, }
          \AttributeTok{add.params =} \FunctionTok{list}\NormalTok{(}\AttributeTok{color=}\FunctionTok{rgb}\NormalTok{(}\DecValTok{0}\NormalTok{,.}\DecValTok{4}\NormalTok{,.}\DecValTok{4}\NormalTok{,  }\DecValTok{1}\NormalTok{), }\AttributeTok{fill =} \StringTok{"light gray"}\NormalTok{), }
          \AttributeTok{conf.int =} \ConstantTok{TRUE}\NormalTok{, }\AttributeTok{main =} \StringTok{"Relação entre IDHM e Índice de GINI"}\NormalTok{) }\SpecialCharTok{+}
  \FunctionTok{labs}\NormalTok{(}\AttributeTok{title =} \StringTok{"Gráfico 3.7 {-} Relação entre IDHM e Índice de GINI"}\NormalTok{,}
       \AttributeTok{subtitle =} \StringTok{"Análise baseada em dados do Atlas Brasil"}\NormalTok{,}
       \AttributeTok{caption =} \StringTok{"Fonte: Elaboração própria com dados do Atlas Brasil"}\NormalTok{) }\SpecialCharTok{+}
  \FunctionTok{theme}\NormalTok{(}\AttributeTok{plot.caption =} \FunctionTok{element\_text}\NormalTok{(}\AttributeTok{hjust =} \DecValTok{0}\NormalTok{, }\AttributeTok{face=}\StringTok{"italic"}\NormalTok{))}
\end{Highlighting}
\end{Shaded}

\begin{verbatim}
## Warning: Duplicated aesthetics after name standardisation: shape
\end{verbatim}

\includegraphics{ML-regressão-IDHM_files/figure-latex/scatterplot com linha de regressão e os resultados-9.pdf}

\begin{Shaded}
\begin{Highlighting}[]
\CommentTok{\# Printar estatística F para ver a significacia da regressão}
\FunctionTok{summary}\NormalTok{(}\FunctionTok{lm}\NormalTok{(IDHM }\SpecialCharTok{\textasciitilde{}}\NormalTok{ indice\_gini, }\AttributeTok{data=}\NormalTok{predic.IDHM))}
\end{Highlighting}
\end{Shaded}

\begin{verbatim}
## 
## Call:
## lm(formula = IDHM ~ indice_gini, data = predic.IDHM)
## 
## Residuals:
##       Min        1Q    Median        3Q       Max 
## -0.098729 -0.030321 -0.002438  0.026708  0.127824 
## 
## Coefficients:
##             Estimate Std. Error t value Pr(>|t|)    
## (Intercept)  0.88755    0.03385  26.217  < 2e-16 ***
## indice_gini -0.28277    0.06549  -4.318  2.2e-05 ***
## ---
## Signif. codes:  0 '***' 0.001 '**' 0.01 '*' 0.05 '.' 0.1 ' ' 1
## 
## Residual standard error: 0.04432 on 278 degrees of freedom
## Multiple R-squared:  0.06284,    Adjusted R-squared:  0.05947 
## F-statistic: 18.64 on 1 and 278 DF,  p-value: 2.196e-05
\end{verbatim}

\begin{Shaded}
\begin{Highlighting}[]
\CommentTok{\# Criar um scatterplot com linha de regressão para IDHM e Índice de Theil}
\FunctionTok{ggscatter}\NormalTok{(predic.IDHM, }\AttributeTok{x =} \StringTok{"ind\_theil\_L"}\NormalTok{, }\AttributeTok{y =} \StringTok{"IDHM"}\NormalTok{, }
          \AttributeTok{color=}\FunctionTok{rgb}\NormalTok{(}\DecValTok{0}\NormalTok{,.}\DecValTok{2}\NormalTok{,.}\DecValTok{5}\NormalTok{,  }\DecValTok{1}\NormalTok{), }\AttributeTok{pch=}\DecValTok{1}\NormalTok{, }\AttributeTok{add =} \StringTok{"reg.line"}\NormalTok{, }
          \AttributeTok{add.params =} \FunctionTok{list}\NormalTok{(}\AttributeTok{color=}\FunctionTok{rgb}\NormalTok{(}\DecValTok{0}\NormalTok{,.}\DecValTok{4}\NormalTok{,.}\DecValTok{4}\NormalTok{,  }\DecValTok{1}\NormalTok{), }\AttributeTok{fill =} \StringTok{"light gray"}\NormalTok{), }
          \AttributeTok{conf.int =} \ConstantTok{TRUE}\NormalTok{, }\AttributeTok{main =} \StringTok{"Relação entre IDHM e Índice de Theil"}\NormalTok{) }\SpecialCharTok{+}
  \FunctionTok{labs}\NormalTok{(}\AttributeTok{title =} \StringTok{"Gráfico 3.8 {-} Relação entre IDHM e Índice de Theil"}\NormalTok{,}
       \AttributeTok{subtitle =} \StringTok{"Análise baseada em dados do Atlas Brasil"}\NormalTok{,}
       \AttributeTok{caption =} \StringTok{"Fonte: Elaboração própria com dados do Atlas Brasil"}\NormalTok{) }\SpecialCharTok{+}
  \FunctionTok{theme}\NormalTok{(}\AttributeTok{plot.caption =} \FunctionTok{element\_text}\NormalTok{(}\AttributeTok{hjust =} \DecValTok{0}\NormalTok{, }\AttributeTok{face=}\StringTok{"italic"}\NormalTok{))}
\end{Highlighting}
\end{Shaded}

\begin{verbatim}
## Warning: Duplicated aesthetics after name standardisation: shape
\end{verbatim}

\includegraphics{ML-regressão-IDHM_files/figure-latex/scatterplot com linha de regressão e os resultados-10.pdf}

\begin{Shaded}
\begin{Highlighting}[]
\CommentTok{\# Printar estatística F para ver a significacia da regressão}
\FunctionTok{summary}\NormalTok{(}\FunctionTok{lm}\NormalTok{(IDHM }\SpecialCharTok{\textasciitilde{}}\NormalTok{ ind\_theil\_L, }\AttributeTok{data=}\NormalTok{predic.IDHM))}
\end{Highlighting}
\end{Shaded}

\begin{verbatim}
## 
## Call:
## lm(formula = IDHM ~ ind_theil_L, data = predic.IDHM)
## 
## Residuals:
##      Min       1Q   Median       3Q      Max 
## -0.09734 -0.02999 -0.00193  0.02572  0.13008 
## 
## Coefficients:
##             Estimate Std. Error t value Pr(>|t|)    
## (Intercept)  0.81220    0.01553  52.306  < 2e-16 ***
## ind_theil_L -0.14792    0.03217  -4.599 6.46e-06 ***
## ---
## Signif. codes:  0 '***' 0.001 '**' 0.01 '*' 0.05 '.' 0.1 ' ' 1
## 
## Residual standard error: 0.04413 on 278 degrees of freedom
## Multiple R-squared:  0.0707, Adjusted R-squared:  0.06735 
## F-statistic: 21.15 on 1 and 278 DF,  p-value: 6.462e-06
\end{verbatim}

\begin{Shaded}
\begin{Highlighting}[]
\CommentTok{\# Criar um scatterplot com linha de regressão para IDHM e Tx. Analfabetismo acima de 25 anos}
\FunctionTok{ggscatter}\NormalTok{(predic.IDHM, }\AttributeTok{x =} \StringTok{"analfabetismo\_25\_anos"}\NormalTok{, }\AttributeTok{y =} \StringTok{"IDHM"}\NormalTok{, }
          \AttributeTok{color=}\FunctionTok{rgb}\NormalTok{(}\DecValTok{0}\NormalTok{,.}\DecValTok{2}\NormalTok{,.}\DecValTok{5}\NormalTok{,  }\DecValTok{1}\NormalTok{), }\AttributeTok{pch=}\DecValTok{1}\NormalTok{, }\AttributeTok{add =} \StringTok{"reg.line"}\NormalTok{, }
          \AttributeTok{add.params =} \FunctionTok{list}\NormalTok{(}\AttributeTok{color=}\FunctionTok{rgb}\NormalTok{(}\DecValTok{0}\NormalTok{,.}\DecValTok{4}\NormalTok{,.}\DecValTok{4}\NormalTok{,  }\DecValTok{1}\NormalTok{), }\AttributeTok{fill =} \StringTok{"light gray"}\NormalTok{), }
          \AttributeTok{conf.int =} \ConstantTok{TRUE}\NormalTok{, }\AttributeTok{main =} \StringTok{"Relação entre IDHM e Tx. Analfabetismo acima de 25 anos"}\NormalTok{) }\SpecialCharTok{+}
  \FunctionTok{labs}\NormalTok{(}\AttributeTok{title =} \StringTok{"Gráfico 3.9 {-}Relação entre IDHM e Tx. Analfabetismo acima de 25 anos"}\NormalTok{,}
       \AttributeTok{subtitle =} \StringTok{"Análise baseada em dados do Atlas Brasil"}\NormalTok{,}
       \AttributeTok{caption =} \StringTok{"Fonte: Elaboração própria com dados do Atlas Brasil"}\NormalTok{) }\SpecialCharTok{+}
  \FunctionTok{theme}\NormalTok{(}\AttributeTok{plot.caption =} \FunctionTok{element\_text}\NormalTok{(}\AttributeTok{hjust =} \DecValTok{0}\NormalTok{, }\AttributeTok{face=}\StringTok{"italic"}\NormalTok{))}
\end{Highlighting}
\end{Shaded}

\begin{verbatim}
## Warning: Duplicated aesthetics after name standardisation: shape
\end{verbatim}

\includegraphics{ML-regressão-IDHM_files/figure-latex/scatterplot com linha de regressão e os resultados-11.pdf}

\begin{Shaded}
\begin{Highlighting}[]
\CommentTok{\# Printar estatística F para ver a significacia da regressão}
\FunctionTok{summary}\NormalTok{(}\FunctionTok{lm}\NormalTok{(IDHM }\SpecialCharTok{\textasciitilde{}}\NormalTok{ analfabetismo\_25\_anos, }\AttributeTok{data=}\NormalTok{predic.IDHM))}
\end{Highlighting}
\end{Shaded}

\begin{verbatim}
## 
## Call:
## lm(formula = IDHM ~ analfabetismo_25_anos, data = predic.IDHM)
## 
## Residuals:
##      Min       1Q   Median       3Q      Max 
## -0.08688 -0.01411  0.00297  0.01686  0.07343 
## 
## Coefficients:
##                         Estimate Std. Error t value Pr(>|t|)    
## (Intercept)            0.8053146  0.0031265  257.57   <2e-16 ***
## analfabetismo_25_anos -0.0058758  0.0002496  -23.54   <2e-16 ***
## ---
## Signif. codes:  0 '***' 0.001 '**' 0.01 '*' 0.05 '.' 0.1 ' ' 1
## 
## Residual standard error: 0.02646 on 278 degrees of freedom
## Multiple R-squared:  0.6659, Adjusted R-squared:  0.6647 
## F-statistic: 554.1 on 1 and 278 DF,  p-value: < 2.2e-16
\end{verbatim}

\begin{Shaded}
\begin{Highlighting}[]
\CommentTok{\# Criar um scatterplot com linha de regressão para IDHM e Tx. Analfabetismo acima de 18 anos}
\FunctionTok{ggscatter}\NormalTok{(predic.IDHM, }\AttributeTok{x =} \StringTok{"analfabetismo\_18\_anos"}\NormalTok{, }\AttributeTok{y =} \StringTok{"IDHM"}\NormalTok{, }
          \AttributeTok{color=}\FunctionTok{rgb}\NormalTok{(}\DecValTok{0}\NormalTok{,.}\DecValTok{2}\NormalTok{,.}\DecValTok{5}\NormalTok{,  }\DecValTok{1}\NormalTok{), }\AttributeTok{pch=}\DecValTok{1}\NormalTok{, }\AttributeTok{add =} \StringTok{"reg.line"}\NormalTok{, }
          \AttributeTok{add.params =} \FunctionTok{list}\NormalTok{(}\AttributeTok{color=}\FunctionTok{rgb}\NormalTok{(}\DecValTok{0}\NormalTok{,.}\DecValTok{4}\NormalTok{,.}\DecValTok{4}\NormalTok{,  }\DecValTok{1}\NormalTok{), }\AttributeTok{fill =} \StringTok{"light gray"}\NormalTok{), }
          \AttributeTok{conf.int =} \ConstantTok{TRUE}\NormalTok{, }\AttributeTok{main =} \StringTok{"Relação entre IDHM e Tx. Analfabetismo acima de 18 anos"}\NormalTok{) }\SpecialCharTok{+}
  \FunctionTok{labs}\NormalTok{(}\AttributeTok{title =} \StringTok{"Gráfico 3.10 {-}Relação entre IDHM e Tx. Analfabetismo acima de 18 anos"}\NormalTok{,}
       \AttributeTok{subtitle =} \StringTok{"Análise baseada em dados do Atlas Brasil"}\NormalTok{,}
       \AttributeTok{caption =} \StringTok{"Fonte: Elaboração própria com dados do Atlas Brasil"}\NormalTok{) }\SpecialCharTok{+}
  \FunctionTok{theme}\NormalTok{(}\AttributeTok{plot.caption =} \FunctionTok{element\_text}\NormalTok{(}\AttributeTok{hjust =} \DecValTok{0}\NormalTok{, }\AttributeTok{face=}\StringTok{"italic"}\NormalTok{))}
\end{Highlighting}
\end{Shaded}

\begin{verbatim}
## Warning: Duplicated aesthetics after name standardisation: shape
\end{verbatim}

\includegraphics{ML-regressão-IDHM_files/figure-latex/scatterplot com linha de regressão e os resultados-12.pdf}

\begin{Shaded}
\begin{Highlighting}[]
\CommentTok{\# Printar estatística F para ver a significacia da regressão}
\FunctionTok{summary}\NormalTok{(}\FunctionTok{lm}\NormalTok{(IDHM }\SpecialCharTok{\textasciitilde{}}\NormalTok{ analfabetismo\_18\_anos, }\AttributeTok{data=}\NormalTok{predic.IDHM))}
\end{Highlighting}
\end{Shaded}

\begin{verbatim}
## 
## Call:
## lm(formula = IDHM ~ analfabetismo_18_anos, data = predic.IDHM)
## 
## Residuals:
##       Min        1Q    Median        3Q       Max 
## -0.087967 -0.014119  0.003096  0.017285  0.073505 
## 
## Coefficients:
##                         Estimate Std. Error t value Pr(>|t|)    
## (Intercept)            0.8052461  0.0031911   252.3   <2e-16 ***
## analfabetismo_18_anos -0.0069547  0.0003024   -23.0   <2e-16 ***
## ---
## Signif. codes:  0 '***' 0.001 '**' 0.01 '*' 0.05 '.' 0.1 ' ' 1
## 
## Residual standard error: 0.02687 on 278 degrees of freedom
## Multiple R-squared:  0.6555, Adjusted R-squared:  0.6542 
## F-statistic: 528.9 on 1 and 278 DF,  p-value: < 2.2e-16
\end{verbatim}

\begin{Shaded}
\begin{Highlighting}[]
\CommentTok{\# Criar um scatterplot com linha de regressão para IDHM e Tx. Analfabetismo acima de 15 anos}
\FunctionTok{ggscatter}\NormalTok{(predic.IDHM, }\AttributeTok{x =} \StringTok{"analfabetismo\_15\_anos"}\NormalTok{, }\AttributeTok{y =} \StringTok{"IDHM"}\NormalTok{, }
          \AttributeTok{color=}\FunctionTok{rgb}\NormalTok{(}\DecValTok{0}\NormalTok{,.}\DecValTok{2}\NormalTok{,.}\DecValTok{5}\NormalTok{,  }\DecValTok{1}\NormalTok{), }\AttributeTok{pch=}\DecValTok{1}\NormalTok{, }\AttributeTok{add =} \StringTok{"reg.line"}\NormalTok{, }
          \AttributeTok{add.params =} \FunctionTok{list}\NormalTok{(}\AttributeTok{color=}\FunctionTok{rgb}\NormalTok{(}\DecValTok{0}\NormalTok{,.}\DecValTok{4}\NormalTok{,.}\DecValTok{4}\NormalTok{,  }\DecValTok{1}\NormalTok{), }\AttributeTok{fill =} \StringTok{"light gray"}\NormalTok{), }
          \AttributeTok{conf.int =} \ConstantTok{TRUE}\NormalTok{, }\AttributeTok{main =} \StringTok{"Relação entre IDHM e Tx. Analfabetismo acima de 15 anos"}\NormalTok{) }\SpecialCharTok{+}
  \FunctionTok{labs}\NormalTok{(}\AttributeTok{title =} \StringTok{"Gráfico 3.11 {-}Relação entre IDHM e Tx. Analfabetismo acima de 15 anos"}\NormalTok{,}
       \AttributeTok{subtitle =} \StringTok{"Análise baseada em dados do Atlas Brasil"}\NormalTok{,}
       \AttributeTok{caption =} \StringTok{"Fonte: Elaboração própria com dados do Atlas Brasil"}\NormalTok{) }\SpecialCharTok{+}
  \FunctionTok{theme}\NormalTok{(}\AttributeTok{plot.caption =} \FunctionTok{element\_text}\NormalTok{(}\AttributeTok{hjust =} \DecValTok{0}\NormalTok{, }\AttributeTok{face=}\StringTok{"italic"}\NormalTok{))}
\end{Highlighting}
\end{Shaded}

\begin{verbatim}
## Warning: Duplicated aesthetics after name standardisation: shape
\end{verbatim}

\includegraphics{ML-regressão-IDHM_files/figure-latex/scatterplot com linha de regressão e os resultados-13.pdf}

\begin{Shaded}
\begin{Highlighting}[]
\CommentTok{\# Printar estatística F para ver a significacia da regressão}
\FunctionTok{summary}\NormalTok{(}\FunctionTok{lm}\NormalTok{(IDHM }\SpecialCharTok{\textasciitilde{}}\NormalTok{ analfabetismo\_15\_anos, }\AttributeTok{data=}\NormalTok{predic.IDHM))}
\end{Highlighting}
\end{Shaded}

\begin{verbatim}
## 
## Call:
## lm(formula = IDHM ~ analfabetismo_15_anos, data = predic.IDHM)
## 
## Residuals:
##       Min        1Q    Median        3Q       Max 
## -0.088380 -0.014143  0.003496  0.017045  0.073506 
## 
## Coefficients:
##                         Estimate Std. Error t value Pr(>|t|)    
## (Intercept)            0.8055144  0.0032362  248.91   <2e-16 ***
## analfabetismo_15_anos -0.0074704  0.0003287  -22.73   <2e-16 ***
## ---
## Signif. codes:  0 '***' 0.001 '**' 0.01 '*' 0.05 '.' 0.1 ' ' 1
## 
## Residual standard error: 0.02708 on 278 degrees of freedom
## Multiple R-squared:  0.6501, Adjusted R-squared:  0.6488 
## F-statistic: 516.4 on 1 and 278 DF,  p-value: < 2.2e-16
\end{verbatim}

\begin{Shaded}
\begin{Highlighting}[]
\CommentTok{\#adaptação do código desenvolvido por Anne (2020)}
\end{Highlighting}
\end{Shaded}

As observações indicaram relação significativa dos indicadores e R2
entre 0,45 e 0,7, renda per capita teve maior R2 = 0,8157, os
indicadores população total, índice Theil L e índice de GINI
apresentaram, apesar de significância menor de 0,001 de p valor, o R2
foi de 0.04372, 0.0707 e 0,06284 respectivamente.

Devido às variáveis índice de GINI, índice de Theil e população total
optou-se por retirá-las do modelo final, por meio do pacote ``dplyr''
com a função ``select'' e o nome das variáveis do modelo precedidos pelo
sinal de subtração.

\begin{Shaded}
\begin{Highlighting}[]
\FunctionTok{library}\NormalTok{(dplyr)}
\end{Highlighting}
\end{Shaded}

\begin{verbatim}
## Warning: package 'dplyr' was built under R version 4.3.1
\end{verbatim}

\begin{verbatim}
## 
## Attaching package: 'dplyr'
\end{verbatim}

\begin{verbatim}
## The following objects are masked from 'package:plyr':
## 
##     arrange, count, desc, failwith, id, mutate, rename, summarise,
##     summarize
\end{verbatim}

\begin{verbatim}
## The following objects are masked from 'package:stats':
## 
##     filter, lag
\end{verbatim}

\begin{verbatim}
## The following objects are masked from 'package:base':
## 
##     intersect, setdiff, setequal, union
\end{verbatim}

\begin{Shaded}
\begin{Highlighting}[]
\NormalTok{predic.IDHM }\OtherTok{\textless{}{-}} \FunctionTok{select}\NormalTok{(predic.IDHM, }\SpecialCharTok{{-}}\NormalTok{ind\_theil\_L, }\SpecialCharTok{{-}}\NormalTok{indice\_gini, }\SpecialCharTok{{-}}\NormalTok{população\_total)}
\CommentTok{\#elaboração própria com a função select do pacote “dplyr” no R}
\end{Highlighting}
\end{Shaded}

Como última análise, o monitoramento de outliers pode prover insights de
possíveis vieses à pesquisa, por isso deve atentar a possibilidade. O
uso da função ``quantile'' auxilia essa análise com a definição do
número de quantis a ser usados, e a função ``sort'' para ajustar os
dados em ordem crescente. Entretanto em ocasiões com um número muito
grande de dados como quando utiliza-se micro dados o uso da função
``sort'' pode vir a ser inviável.

\begin{Shaded}
\begin{Highlighting}[]
\FunctionTok{print}\NormalTok{(}\FunctionTok{paste0}\NormalTok{(}\StringTok{"renda\_per\_capita"}\NormalTok{))}
\end{Highlighting}
\end{Shaded}

\begin{verbatim}
## [1] "renda_per_capita"
\end{verbatim}

\begin{Shaded}
\begin{Highlighting}[]
\FunctionTok{quantile}\NormalTok{(predic.IDHM}\SpecialCharTok{$}\NormalTok{renda\_per\_capita, }\AttributeTok{probs =} \FunctionTok{seq}\NormalTok{(.}\DecValTok{1}\NormalTok{, }\DecValTok{1}\NormalTok{, }\AttributeTok{by =}\NormalTok{ .}\DecValTok{1}\NormalTok{))}
\end{Highlighting}
\end{Shaded}

\begin{verbatim}
##      10%      20%      30%      40%      50%      60%      70%      80% 
##  451.576  489.930  520.626  553.800  593.620  727.718  775.931  883.042 
##      90%     100% 
##  985.232 1568.870
\end{verbatim}

\begin{Shaded}
\begin{Highlighting}[]
\FunctionTok{sort}\NormalTok{(predic.IDHM}\SpecialCharTok{$}\NormalTok{renda\_per\_capita)}
\end{Highlighting}
\end{Shaded}

\begin{verbatim}
##   [1]  338.42  341.32  350.41  356.63  362.29  362.31  366.24  367.42  370.74
##  [10]  382.11  395.06  401.31  404.28  405.80  415.17  418.37  419.89  421.03
##  [19]  422.39  432.99  442.82  443.65  444.05  445.31  447.25  447.97  451.27
##  [28]  451.45  451.59  452.75  453.47  458.68  464.70  465.74  467.19  469.93
##  [37]  470.59  471.12  471.54  471.65  472.85  473.32  473.59  473.62  478.02
##  [46]  478.39  478.58  478.80  478.88  479.55  480.55  481.46  482.64  484.80
##  [55]  485.71  486.65  490.75  492.78  494.32  495.42  497.15  498.33  499.19
##  [64]  501.53  502.70  503.29  503.66  503.95  506.02  506.21  508.14  508.36
##  [73]  509.32  509.57  511.42  512.47  512.75  514.59  516.00  516.75  516.97
##  [82]  517.64  520.00  520.08  520.86  523.09  523.38  524.68  525.07  525.50
##  [91]  526.67  528.23  528.65  529.31  531.35  534.64  536.41  536.77  538.12
## [100]  538.50  540.06  540.61  541.74  543.81  546.78  547.09  547.20  548.93
## [109]  549.06  549.29  549.54  550.02  556.32  557.38  557.71  559.44  560.50
## [118]  561.11  561.40  561.52  564.61  565.81  567.26  569.17  569.89  572.81
## [127]  573.92  574.03  574.89  575.58  575.63  579.19  579.46  580.02  583.80
## [136]  585.99  588.48  588.89  589.03  593.46  593.78  593.84  595.13  596.82
## [145]  598.77  599.80  603.61  607.23  613.07  614.31  617.00  640.90  644.99
## [154]  663.76  673.18  679.62  679.72  684.63  691.06  699.24  703.36  707.48
## [163]  713.76  717.96  718.35  722.67  723.84  727.13  728.60  732.94  734.81
## [172]  738.04  739.68  745.24  746.52  747.30  749.17  749.30  750.40  755.14
## [181]  757.43  758.07  758.68  759.11  764.29  765.94  766.18  767.68  767.86
## [190]  770.94  771.16  772.78  773.16  774.72  775.41  775.46  777.03  777.37
## [199]  779.13  780.32  781.44  784.47  788.18  791.12  791.40  792.81  792.86
## [208]  794.67  798.41  802.94  805.71  806.32  807.78  810.08  814.30  817.79
## [217]  819.61  821.80  833.57  849.52  854.53  860.55  873.96  882.69  884.45
## [226]  887.28  888.32  891.00  896.60  900.00  901.20  901.42  909.43  912.60
## [235]  921.16  925.29  926.61  927.60  929.11  932.80  937.67  944.53  944.64
## [244]  949.54  959.06  959.50  961.32  971.82  974.74  976.02  978.64  984.82
## [253]  988.94  990.06  994.83  998.33 1000.82 1001.71 1007.29 1029.92 1032.89
## [262] 1037.42 1038.98 1044.95 1047.74 1068.69 1072.88 1093.25 1096.85 1099.62
## [271] 1326.87 1377.92 1456.83 1457.06 1498.74 1508.91 1533.05 1546.18 1553.68
## [280] 1568.87
\end{verbatim}

\begin{Shaded}
\begin{Highlighting}[]
\FunctionTok{print}\NormalTok{(}\FunctionTok{paste0}\NormalTok{(}\StringTok{"sub\_esco\_pop"}\NormalTok{))}
\end{Highlighting}
\end{Shaded}

\begin{verbatim}
## [1] "sub_esco_pop"
\end{verbatim}

\begin{Shaded}
\begin{Highlighting}[]
\FunctionTok{quantile}\NormalTok{(predic.IDHM}\SpecialCharTok{$}\NormalTok{sub\_esco\_pop, }\AttributeTok{probs =} \FunctionTok{seq}\NormalTok{(.}\DecValTok{1}\NormalTok{, }\DecValTok{1}\NormalTok{, }\AttributeTok{by =}\NormalTok{ .}\DecValTok{1}\NormalTok{))}
\end{Highlighting}
\end{Shaded}

\begin{verbatim}
##    10%    20%    30%    40%    50%    60%    70%    80%    90%   100% 
## 0.5299 0.5620 0.5857 0.6126 0.6305 0.6490 0.6700 0.6986 0.7362 0.8570
\end{verbatim}

\begin{Shaded}
\begin{Highlighting}[]
\FunctionTok{sort}\NormalTok{(predic.IDHM}\SpecialCharTok{$}\NormalTok{sub\_esco\_pop)}
\end{Highlighting}
\end{Shaded}

\begin{verbatim}
##   [1] 0.464 0.470 0.475 0.477 0.487 0.494 0.496 0.499 0.500 0.507 0.508 0.510
##  [13] 0.513 0.513 0.514 0.519 0.519 0.520 0.521 0.523 0.523 0.523 0.524 0.525
##  [25] 0.526 0.527 0.527 0.529 0.530 0.530 0.533 0.534 0.535 0.536 0.536 0.537
##  [37] 0.540 0.540 0.544 0.544 0.546 0.549 0.549 0.551 0.551 0.552 0.552 0.552
##  [49] 0.553 0.554 0.555 0.556 0.557 0.557 0.562 0.562 0.562 0.563 0.563 0.563
##  [61] 0.564 0.564 0.567 0.568 0.568 0.568 0.572 0.572 0.573 0.573 0.574 0.575
##  [73] 0.575 0.576 0.577 0.577 0.578 0.578 0.580 0.581 0.582 0.584 0.585 0.585
##  [85] 0.586 0.588 0.588 0.589 0.590 0.590 0.590 0.591 0.593 0.593 0.595 0.595
##  [97] 0.595 0.598 0.599 0.599 0.599 0.600 0.601 0.603 0.604 0.605 0.605 0.606
## [109] 0.607 0.607 0.609 0.612 0.613 0.613 0.613 0.614 0.614 0.614 0.614 0.615
## [121] 0.615 0.615 0.617 0.617 0.618 0.618 0.618 0.619 0.622 0.623 0.624 0.624
## [133] 0.625 0.626 0.627 0.628 0.628 0.630 0.630 0.630 0.631 0.631 0.631 0.631
## [145] 0.633 0.633 0.633 0.634 0.635 0.636 0.637 0.637 0.637 0.638 0.639 0.639
## [157] 0.640 0.640 0.641 0.642 0.642 0.642 0.644 0.645 0.645 0.647 0.648 0.649
## [169] 0.649 0.652 0.652 0.652 0.652 0.653 0.654 0.654 0.655 0.655 0.657 0.659
## [181] 0.659 0.659 0.659 0.660 0.660 0.662 0.662 0.664 0.665 0.665 0.667 0.667
## [193] 0.668 0.669 0.670 0.670 0.670 0.671 0.673 0.674 0.675 0.676 0.681 0.681
## [205] 0.681 0.683 0.683 0.684 0.684 0.685 0.686 0.686 0.687 0.687 0.688 0.688
## [217] 0.688 0.689 0.689 0.691 0.692 0.695 0.698 0.698 0.701 0.702 0.702 0.703
## [229] 0.703 0.704 0.704 0.706 0.707 0.708 0.708 0.709 0.712 0.713 0.716 0.718
## [241] 0.719 0.720 0.725 0.725 0.726 0.727 0.728 0.730 0.730 0.731 0.733 0.736
## [253] 0.738 0.739 0.739 0.739 0.740 0.742 0.748 0.754 0.754 0.755 0.755 0.759
## [265] 0.760 0.765 0.766 0.775 0.775 0.776 0.783 0.783 0.787 0.793 0.793 0.806
## [277] 0.812 0.813 0.822 0.857
\end{verbatim}

\begin{Shaded}
\begin{Highlighting}[]
\FunctionTok{print}\NormalTok{(}\FunctionTok{paste0}\NormalTok{(}\StringTok{"sub\_freq\_esco"}\NormalTok{))}
\end{Highlighting}
\end{Shaded}

\begin{verbatim}
## [1] "sub_freq_esco"
\end{verbatim}

\begin{Shaded}
\begin{Highlighting}[]
\FunctionTok{quantile}\NormalTok{(predic.IDHM}\SpecialCharTok{$}\NormalTok{sub\_freq\_esco, }\AttributeTok{probs =} \FunctionTok{seq}\NormalTok{(.}\DecValTok{1}\NormalTok{, }\DecValTok{1}\NormalTok{, }\AttributeTok{by =}\NormalTok{ .}\DecValTok{1}\NormalTok{))}
\end{Highlighting}
\end{Shaded}

\begin{verbatim}
##    10%    20%    30%    40%    50%    60%    70%    80%    90%   100% 
## 0.6758 0.7068 0.7230 0.7420 0.7560 0.7700 0.7843 0.8012 0.8260 0.8980
\end{verbatim}

\begin{Shaded}
\begin{Highlighting}[]
\FunctionTok{sort}\NormalTok{(predic.IDHM}\SpecialCharTok{$}\NormalTok{sub\_freq\_esco)}
\end{Highlighting}
\end{Shaded}

\begin{verbatim}
##   [1] 0.606 0.608 0.624 0.628 0.631 0.635 0.635 0.639 0.642 0.645 0.645 0.653
##  [13] 0.654 0.655 0.655 0.660 0.664 0.666 0.667 0.669 0.670 0.670 0.670 0.672
##  [25] 0.672 0.673 0.674 0.674 0.676 0.677 0.679 0.679 0.681 0.681 0.682 0.687
##  [37] 0.687 0.688 0.689 0.689 0.690 0.690 0.691 0.694 0.695 0.697 0.699 0.699
##  [49] 0.700 0.700 0.703 0.704 0.704 0.704 0.705 0.706 0.707 0.707 0.708 0.709
##  [61] 0.710 0.710 0.710 0.711 0.711 0.714 0.714 0.714 0.714 0.715 0.715 0.715
##  [73] 0.717 0.717 0.717 0.717 0.718 0.718 0.719 0.719 0.720 0.720 0.722 0.723
##  [85] 0.723 0.723 0.723 0.725 0.726 0.726 0.727 0.728 0.729 0.729 0.731 0.732
##  [97] 0.732 0.732 0.733 0.735 0.735 0.735 0.735 0.736 0.738 0.739 0.739 0.739
## [109] 0.740 0.740 0.741 0.742 0.742 0.742 0.742 0.742 0.742 0.744 0.744 0.744
## [121] 0.745 0.745 0.746 0.747 0.747 0.748 0.748 0.749 0.750 0.750 0.751 0.752
## [133] 0.752 0.752 0.753 0.753 0.756 0.756 0.756 0.756 0.756 0.757 0.757 0.757
## [145] 0.758 0.758 0.758 0.759 0.760 0.760 0.761 0.762 0.763 0.765 0.766 0.766
## [157] 0.767 0.767 0.768 0.768 0.768 0.769 0.769 0.770 0.770 0.770 0.770 0.770
## [169] 0.770 0.771 0.772 0.772 0.774 0.774 0.774 0.774 0.774 0.775 0.776 0.777
## [181] 0.777 0.778 0.778 0.779 0.779 0.779 0.780 0.781 0.781 0.782 0.782 0.782
## [193] 0.783 0.784 0.784 0.784 0.785 0.785 0.785 0.786 0.787 0.788 0.788 0.789
## [205] 0.789 0.790 0.790 0.791 0.793 0.793 0.793 0.794 0.795 0.795 0.797 0.797
## [217] 0.798 0.799 0.799 0.799 0.799 0.799 0.800 0.801 0.802 0.804 0.805 0.807
## [229] 0.807 0.809 0.811 0.811 0.813 0.815 0.815 0.815 0.816 0.816 0.818 0.818
## [241] 0.819 0.820 0.820 0.821 0.821 0.821 0.821 0.822 0.823 0.824 0.826 0.826
## [253] 0.826 0.826 0.827 0.829 0.829 0.829 0.830 0.831 0.831 0.836 0.838 0.838
## [265] 0.838 0.839 0.840 0.840 0.844 0.847 0.848 0.857 0.857 0.865 0.869 0.872
## [277] 0.873 0.882 0.886 0.898
\end{verbatim}

\begin{Shaded}
\begin{Highlighting}[]
\FunctionTok{print}\NormalTok{(}\FunctionTok{paste0}\NormalTok{(}\StringTok{"esperança\_de\_vida"}\NormalTok{))}
\end{Highlighting}
\end{Shaded}

\begin{verbatim}
## [1] "esperança_de_vida"
\end{verbatim}

\begin{Shaded}
\begin{Highlighting}[]
\FunctionTok{quantile}\NormalTok{(predic.IDHM}\SpecialCharTok{$}\NormalTok{esperança\_de\_vida, }\AttributeTok{probs =} \FunctionTok{seq}\NormalTok{(.}\DecValTok{1}\NormalTok{, }\DecValTok{1}\NormalTok{, }\AttributeTok{by =}\NormalTok{ .}\DecValTok{1}\NormalTok{))}
\end{Highlighting}
\end{Shaded}

\begin{verbatim}
##    10%    20%    30%    40%    50%    60%    70%    80%    90%   100% 
## 70.770 71.596 72.257 73.042 73.705 74.358 75.275 76.500 77.752 79.980
\end{verbatim}

\begin{Shaded}
\begin{Highlighting}[]
\FunctionTok{sort}\NormalTok{(predic.IDHM}\SpecialCharTok{$}\NormalTok{esperança\_de\_vida)}
\end{Highlighting}
\end{Shaded}

\begin{verbatim}
##   [1] 67.90 68.28 68.57 68.77 68.86 69.34 69.61 69.62 69.65 69.71 69.87 69.96
##  [13] 69.96 70.03 70.10 70.19 70.25 70.37 70.38 70.44 70.45 70.50 70.54 70.57
##  [25] 70.71 70.76 70.76 70.77 70.77 70.83 70.85 70.85 70.91 70.94 71.06 71.10
##  [37] 71.11 71.11 71.11 71.12 71.12 71.16 71.16 71.20 71.23 71.30 71.30 71.34
##  [49] 71.36 71.39 71.41 71.44 71.47 71.47 71.50 71.54 71.61 71.64 71.69 71.70
##  [61] 71.71 71.72 71.74 71.76 71.80 71.84 71.85 71.88 71.90 71.99 72.00 72.01
##  [73] 72.02 72.02 72.03 72.03 72.03 72.08 72.11 72.11 72.15 72.15 72.22 72.25
##  [85] 72.26 72.29 72.30 72.31 72.34 72.39 72.41 72.42 72.47 72.48 72.50 72.53
##  [97] 72.55 72.59 72.60 72.66 72.71 72.77 72.78 72.79 72.80 72.83 72.84 72.84
## [109] 72.91 72.94 73.01 73.03 73.05 73.07 73.09 73.16 73.16 73.16 73.18 73.21
## [121] 73.24 73.26 73.29 73.29 73.37 73.37 73.39 73.40 73.41 73.43 73.49 73.56
## [133] 73.56 73.58 73.61 73.64 73.65 73.66 73.69 73.69 73.72 73.75 73.78 73.81
## [145] 73.87 73.89 73.91 73.92 73.92 73.93 73.94 73.95 73.96 74.02 74.05 74.09
## [157] 74.09 74.15 74.16 74.16 74.16 74.18 74.20 74.20 74.20 74.26 74.29 74.35
## [169] 74.37 74.42 74.42 74.44 74.47 74.48 74.51 74.53 74.59 74.59 74.60 74.65
## [181] 74.66 74.68 74.78 74.80 74.86 74.88 74.88 74.88 74.90 74.91 74.94 75.11
## [193] 75.13 75.17 75.22 75.23 75.38 75.40 75.49 75.56 75.62 75.63 75.68 75.74
## [205] 75.75 75.76 75.87 75.88 75.96 76.00 76.08 76.09 76.12 76.18 76.21 76.22
## [217] 76.24 76.28 76.29 76.40 76.40 76.47 76.47 76.47 76.62 76.69 76.72 76.76
## [229] 76.78 76.80 76.86 76.87 76.91 76.96 77.02 77.04 77.15 77.17 77.17 77.21
## [241] 77.23 77.32 77.47 77.47 77.49 77.49 77.54 77.61 77.70 77.70 77.73 77.75
## [253] 77.77 77.78 77.89 77.89 77.97 78.02 78.04 78.07 78.07 78.22 78.27 78.30
## [265] 78.35 78.42 78.52 78.53 78.55 78.63 78.68 78.76 78.76 78.83 78.87 79.09
## [277] 79.12 79.40 79.70 79.98
\end{verbatim}

\begin{Shaded}
\begin{Highlighting}[]
\FunctionTok{print}\NormalTok{(}\FunctionTok{paste0}\NormalTok{(}\StringTok{"porcent\_pobres"}\NormalTok{))}
\end{Highlighting}
\end{Shaded}

\begin{verbatim}
## [1] "porcent_pobres"
\end{verbatim}

\begin{Shaded}
\begin{Highlighting}[]
\FunctionTok{quantile}\NormalTok{(predic.IDHM}\SpecialCharTok{$}\NormalTok{porcent\_pobres, }\AttributeTok{probs =} \FunctionTok{seq}\NormalTok{(.}\DecValTok{1}\NormalTok{, }\DecValTok{1}\NormalTok{, }\AttributeTok{by =}\NormalTok{ .}\DecValTok{1}\NormalTok{))}
\end{Highlighting}
\end{Shaded}

\begin{verbatim}
##    10%    20%    30%    40%    50%    60%    70%    80%    90%   100% 
##  4.097  4.864  6.473  8.990 12.605 16.746 20.103 21.684 23.725 32.530
\end{verbatim}

\begin{Shaded}
\begin{Highlighting}[]
\FunctionTok{sort}\NormalTok{(predic.IDHM}\SpecialCharTok{$}\NormalTok{porcent\_pobres)}
\end{Highlighting}
\end{Shaded}

\begin{verbatim}
##   [1]  1.83  1.98  2.25  2.32  2.39  2.65  2.86  2.90  2.97  2.99  3.00  3.07
##  [13]  3.10  3.33  3.34  3.36  3.48  3.57  3.68  3.68  3.86  3.86  3.89  3.93
##  [25]  4.01  4.03  4.06  4.07  4.10  4.14  4.18  4.20  4.20  4.22  4.25  4.26
##  [37]  4.30  4.30  4.33  4.33  4.35  4.37  4.41  4.42  4.44  4.45  4.56  4.62
##  [49]  4.71  4.73  4.77  4.78  4.79  4.81  4.83  4.84  4.87  4.87  4.98  4.98
##  [61]  5.02  5.03  5.19  5.19  5.20  5.21  5.27  5.28  5.33  5.36  5.40  5.48
##  [73]  5.49  5.73  5.74  5.77  5.97  6.00  6.01  6.07  6.10  6.33  6.39  6.41
##  [85]  6.50  6.52  6.55  6.62  6.62  6.64  6.68  6.71  6.73  6.73  6.87  6.91
##  [97]  6.96  6.99  7.07  7.15  7.18  7.69  7.83  8.12  8.13  8.20  8.21  8.25
## [109]  8.64  8.67  8.68  8.96  9.01  9.11  9.25  9.41  9.59 10.07 10.18 10.44
## [121] 10.52 10.62 10.68 11.03 11.04 11.07 11.11 11.30 11.36 11.38 11.40 11.41
## [133] 11.64 11.66 11.85 11.94 12.32 12.32 12.41 12.48 12.73 13.01 13.02 13.15
## [145] 13.17 13.43 13.66 13.70 14.05 14.35 14.64 14.71 14.83 15.19 15.28 15.35
## [157] 15.57 15.76 15.82 15.92 15.92 15.93 16.19 16.38 16.44 16.45 16.65 16.65
## [169] 16.89 17.34 17.49 17.80 18.03 18.18 18.23 18.24 18.29 18.37 18.56 18.75
## [181] 18.80 18.87 19.16 19.27 19.29 19.32 19.38 19.49 19.57 19.68 19.77 19.79
## [193] 19.87 19.93 19.95 20.10 20.11 20.15 20.28 20.30 20.31 20.34 20.42 20.50
## [205] 20.50 20.51 20.54 20.64 20.70 20.72 20.72 20.90 20.90 20.91 20.92 21.13
## [217] 21.25 21.37 21.56 21.61 21.61 21.65 21.66 21.67 21.74 21.90 21.93 22.05
## [229] 22.20 22.23 22.34 22.35 22.41 22.44 22.46 22.56 22.77 22.78 22.82 22.89
## [241] 23.19 23.23 23.29 23.36 23.44 23.44 23.45 23.46 23.49 23.59 23.60 23.72
## [253] 23.77 23.89 23.89 24.22 24.23 24.38 24.45 24.61 24.91 25.19 25.42 25.47
## [265] 25.63 25.79 25.84 26.43 26.60 26.96 27.26 27.88 28.09 28.43 28.75 29.24
## [277] 29.65 29.88 30.26 32.53
\end{verbatim}

\begin{Shaded}
\begin{Highlighting}[]
\FunctionTok{print}\NormalTok{(}\FunctionTok{paste0}\NormalTok{(}\StringTok{"mortalidade\_infantil"}\NormalTok{))}
\end{Highlighting}
\end{Shaded}

\begin{verbatim}
## [1] "mortalidade_infantil"
\end{verbatim}

\begin{Shaded}
\begin{Highlighting}[]
\FunctionTok{quantile}\NormalTok{(predic.IDHM}\SpecialCharTok{$}\NormalTok{mortalidade\_infantil, }\AttributeTok{probs =} \FunctionTok{seq}\NormalTok{(.}\DecValTok{1}\NormalTok{, }\DecValTok{1}\NormalTok{, }\AttributeTok{by =}\NormalTok{ .}\DecValTok{1}\NormalTok{))}
\end{Highlighting}
\end{Shaded}

\begin{verbatim}
##    10%    20%    30%    40%    50%    60%    70%    80%    90%   100% 
##  9.920 11.130 13.021 14.684 15.910 17.028 18.073 19.362 21.453 30.410
\end{verbatim}

\begin{Shaded}
\begin{Highlighting}[]
\FunctionTok{sort}\NormalTok{(predic.IDHM}\SpecialCharTok{$}\NormalTok{mortalidade\_infantil)}
\end{Highlighting}
\end{Shaded}

\begin{verbatim}
##   [1]  7.82  8.12  8.28  8.39  8.45  8.61  8.65  8.81  8.88  8.93  8.97  9.05
##  [13]  9.10  9.13  9.23  9.23  9.32  9.35  9.39  9.52  9.60  9.60  9.64  9.67
##  [25]  9.74  9.84  9.89  9.92  9.92  9.95 10.02 10.15 10.17 10.18 10.18 10.21
##  [37] 10.21 10.37 10.46 10.46 10.48 10.49 10.51 10.53 10.56 10.57 10.62 10.74
##  [49] 10.80 10.80 10.81 10.88 10.95 11.04 11.09 11.13 11.13 11.18 11.25 11.28
##  [61] 11.42 11.42 11.42 11.46 11.53 11.55 11.57 11.78 11.95 12.00 12.03 12.03
##  [73] 12.18 12.21 12.35 12.42 12.63 12.63 12.66 12.72 12.78 12.82 12.88 12.93
##  [85] 13.06 13.14 13.21 13.27 13.30 13.31 13.34 13.37 13.42 13.61 13.70 13.84
##  [97] 13.86 13.90 13.93 14.02 14.07 14.11 14.12 14.13 14.22 14.22 14.22 14.48
## [109] 14.48 14.56 14.57 14.60 14.74 14.75 14.76 14.84 14.94 14.94 15.00 15.02
## [121] 15.04 15.11 15.16 15.27 15.28 15.37 15.39 15.44 15.44 15.44 15.46 15.52
## [133] 15.52 15.53 15.71 15.79 15.85 15.88 15.89 15.91 15.91 15.97 16.03 16.11
## [145] 16.18 16.23 16.24 16.25 16.26 16.30 16.34 16.35 16.42 16.44 16.47 16.49
## [157] 16.59 16.60 16.67 16.76 16.76 16.78 16.81 16.83 16.84 16.96 16.97 17.00
## [169] 17.07 17.12 17.13 17.13 17.14 17.16 17.21 17.29 17.32 17.41 17.42 17.45
## [181] 17.47 17.51 17.55 17.58 17.68 17.68 17.69 17.75 17.78 17.80 17.83 17.85
## [193] 17.89 17.94 18.00 18.07 18.08 18.09 18.09 18.10 18.17 18.19 18.21 18.24
## [205] 18.27 18.34 18.39 18.48 18.51 18.66 18.71 18.72 18.77 18.90 18.91 18.93
## [217] 19.04 19.09 19.10 19.13 19.16 19.25 19.31 19.32 19.53 19.57 19.65 19.67
## [229] 19.67 19.69 19.70 19.76 19.87 20.02 20.04 20.12 20.18 20.20 20.21 20.27
## [241] 20.31 20.34 20.49 20.54 20.55 20.63 20.88 20.99 21.03 21.04 21.24 21.45
## [253] 21.48 21.65 21.66 21.93 22.00 22.30 22.35 22.41 22.55 22.56 22.57 22.81
## [265] 23.01 23.23 23.38 23.44 23.66 23.73 23.88 24.10 24.25 24.33 24.73 24.98
## [277] 26.14 26.33 27.28 30.41
\end{verbatim}

\begin{Shaded}
\begin{Highlighting}[]
\FunctionTok{print}\NormalTok{(}\FunctionTok{paste0}\NormalTok{(}\StringTok{"media\_anos\_de\_estudo"}\NormalTok{))}
\end{Highlighting}
\end{Shaded}

\begin{verbatim}
## [1] "media_anos_de_estudo"
\end{verbatim}

\begin{Shaded}
\begin{Highlighting}[]
\FunctionTok{quantile}\NormalTok{(predic.IDHM}\SpecialCharTok{$}\NormalTok{media\_anos\_de\_estudo, }\AttributeTok{probs =} \FunctionTok{seq}\NormalTok{(.}\DecValTok{1}\NormalTok{, }\DecValTok{1}\NormalTok{, }\AttributeTok{by =}\NormalTok{ .}\DecValTok{1}\NormalTok{))}
\end{Highlighting}
\end{Shaded}

\begin{verbatim}
##    10%    20%    30%    40%    50%    60%    70%    80%    90%   100% 
##  7.459  7.880  8.227  8.600  8.840  9.144  9.490  9.820 10.183 12.200
\end{verbatim}

\begin{Shaded}
\begin{Highlighting}[]
\FunctionTok{sort}\NormalTok{(predic.IDHM}\SpecialCharTok{$}\NormalTok{media\_anos\_de\_estudo)}
\end{Highlighting}
\end{Shaded}

\begin{verbatim}
##   [1]  6.54  6.66  6.70  6.71  6.72  6.80  6.87  6.94  6.99  7.00  7.03  7.08
##  [13]  7.11  7.19  7.21  7.22  7.25  7.25  7.28  7.29  7.31  7.34  7.34  7.37
##  [25]  7.41  7.44  7.45  7.45  7.46  7.48  7.48  7.53  7.55  7.60  7.60  7.60
##  [37]  7.60  7.61  7.63  7.64  7.64  7.65  7.66  7.69  7.70  7.71  7.72  7.72
##  [49]  7.74  7.75  7.76  7.76  7.78  7.80  7.81  7.84  7.89  7.92  7.93  7.93
##  [61]  7.94  7.95  7.95  7.98  7.98  7.99  8.01  8.01  8.04  8.04  8.06  8.08
##  [73]  8.09  8.10  8.10  8.11  8.11  8.12  8.12  8.13  8.16  8.18  8.18  8.22
##  [85]  8.23  8.25  8.26  8.28  8.28  8.29  8.29  8.29  8.31  8.31  8.31  8.32
##  [97]  8.33  8.36  8.38  8.39  8.40  8.42  8.44  8.50  8.53  8.54  8.56  8.57
## [109]  8.58  8.58  8.60  8.60  8.60  8.61  8.61  8.63  8.63  8.64  8.64  8.64
## [121]  8.65  8.66  8.68  8.68  8.69  8.70  8.72  8.72  8.72  8.73  8.74  8.74
## [133]  8.76  8.76  8.77  8.79  8.81  8.82  8.84  8.84  8.84  8.88  8.89  8.92
## [145]  8.92  8.95  8.95  8.97  8.99  8.99  8.99  9.00  9.01  9.02  9.03  9.04
## [157]  9.04  9.07  9.07  9.09  9.09  9.09  9.10  9.11  9.12  9.13  9.13  9.14
## [169]  9.15  9.16  9.16  9.17  9.17  9.18  9.18  9.18  9.19  9.21  9.24  9.24
## [181]  9.25  9.26  9.28  9.30  9.30  9.30  9.32  9.32  9.32  9.36  9.37  9.41
## [193]  9.45  9.45  9.47  9.49  9.49  9.50  9.50  9.53  9.55  9.56  9.57  9.57
## [205]  9.58  9.59  9.59  9.60  9.62  9.62  9.63  9.65  9.66  9.67  9.70  9.72
## [217]  9.75  9.76  9.77  9.78  9.79  9.79  9.80  9.82  9.82  9.83  9.84  9.84
## [229]  9.85  9.85  9.85  9.89  9.89  9.90  9.91  9.91  9.94  9.94  9.95  9.97
## [241]  9.99 10.00 10.02 10.02 10.02 10.08 10.08 10.10 10.11 10.12 10.13 10.18
## [253] 10.21 10.21 10.27 10.31 10.35 10.35 10.37 10.39 10.47 10.47 10.52 10.58
## [265] 10.68 10.76 10.80 10.88 10.92 10.93 10.95 11.01 11.04 11.19 11.19 11.33
## [277] 11.33 11.65 11.82 12.20
\end{verbatim}

\begin{Shaded}
\begin{Highlighting}[]
\FunctionTok{print}\NormalTok{(}\FunctionTok{paste0}\NormalTok{(}\StringTok{"analfabetismo\_25\_anos"}\NormalTok{))}
\end{Highlighting}
\end{Shaded}

\begin{verbatim}
## [1] "analfabetismo_25_anos"
\end{verbatim}

\begin{Shaded}
\begin{Highlighting}[]
\FunctionTok{quantile}\NormalTok{(predic.IDHM}\SpecialCharTok{$}\NormalTok{analfabetismo\_25\_anos, }\AttributeTok{probs =} \FunctionTok{seq}\NormalTok{(.}\DecValTok{1}\NormalTok{, }\DecValTok{1}\NormalTok{, }\AttributeTok{by =}\NormalTok{ .}\DecValTok{1}\NormalTok{))}
\end{Highlighting}
\end{Shaded}

\begin{verbatim}
##    10%    20%    30%    40%    50%    60%    70%    80%    90%   100% 
##  3.407  4.898  6.431  7.756  8.760 11.058 14.960 17.434 20.251 25.310
\end{verbatim}

\begin{Shaded}
\begin{Highlighting}[]
\FunctionTok{sort}\NormalTok{(predic.IDHM}\SpecialCharTok{$}\NormalTok{analfabetismo\_25\_anos)}
\end{Highlighting}
\end{Shaded}

\begin{verbatim}
##   [1]  1.90  2.01  2.17  2.18  2.25  2.26  2.36  2.40  2.56  2.59  2.60  2.71
##  [13]  2.95  2.95  2.96  2.97  3.03  3.10  3.10  3.12  3.17  3.19  3.20  3.30
##  [25]  3.31  3.35  3.36  3.38  3.41  3.41  3.42  3.45  3.46  3.50  3.54  3.57
##  [37]  3.68  3.69  3.70  3.71  3.73  3.79  3.82  3.86  3.87  3.91  4.21  4.22
##  [49]  4.24  4.28  4.36  4.62  4.71  4.74  4.80  4.85  4.91  4.98  5.13  5.18
##  [61]  5.20  5.27  5.36  5.42  5.47  5.49  5.52  5.54  5.58  5.59  5.61  5.82
##  [73]  5.83  5.86  5.96  5.99  6.01  6.18  6.21  6.21  6.26  6.30  6.32  6.41
##  [85]  6.44  6.51  6.51  6.52  6.54  6.65  6.68  6.72  6.83  6.86  6.95  6.96
##  [97]  6.97  7.04  7.05  7.07  7.07  7.28  7.29  7.42  7.42  7.45  7.60  7.63
## [109]  7.64  7.64  7.66  7.69  7.80  7.81  7.82  7.87  7.88  7.88  7.90  7.90
## [121]  7.93  7.98  7.99  8.02  8.03  8.13  8.23  8.24  8.25  8.26  8.38  8.40
## [133]  8.41  8.55  8.55  8.56  8.56  8.63  8.66  8.71  8.81  8.83  8.91  8.91
## [145]  8.97  8.99  8.99  9.00  9.03  9.11  9.30  9.32  9.38  9.43  9.46  9.52
## [157]  9.77  9.87 10.08 10.21 10.22 10.28 10.41 10.41 10.59 10.67 10.99 11.05
## [169] 11.07 11.17 11.60 11.80 11.84 11.92 12.05 12.28 12.41 12.58 12.60 12.73
## [181] 12.77 12.77 13.39 13.44 13.46 13.49 13.92 14.00 14.24 14.37 14.43 14.61
## [193] 14.71 14.76 14.79 14.93 15.03 15.14 15.15 15.34 15.35 15.42 15.57 15.73
## [205] 15.79 15.84 15.90 15.98 16.09 16.15 16.18 16.36 16.37 16.38 16.41 16.44
## [217] 16.45 16.48 16.70 16.75 16.96 17.02 17.13 17.40 17.57 17.76 17.91 17.92
## [229] 17.96 17.98 18.17 18.22 18.50 18.53 18.55 18.76 18.98 19.06 19.08 19.16
## [241] 19.25 19.27 19.33 19.35 19.41 19.42 19.51 19.69 19.77 19.97 20.05 20.25
## [253] 20.26 20.28 20.47 20.52 20.59 20.95 21.13 21.38 21.44 21.47 21.51 21.58
## [265] 21.97 22.37 22.43 22.44 22.55 22.75 23.25 23.72 23.78 24.04 24.22 24.65
## [277] 24.78 24.96 24.97 25.31
\end{verbatim}

\begin{Shaded}
\begin{Highlighting}[]
\FunctionTok{print}\NormalTok{(}\FunctionTok{paste0}\NormalTok{(}\StringTok{"analfabetismo\_18\_anos"}\NormalTok{))}
\end{Highlighting}
\end{Shaded}

\begin{verbatim}
## [1] "analfabetismo_18_anos"
\end{verbatim}

\begin{Shaded}
\begin{Highlighting}[]
\FunctionTok{quantile}\NormalTok{(predic.IDHM}\SpecialCharTok{$}\NormalTok{analfabetismo\_18\_anos, }\AttributeTok{probs =} \FunctionTok{seq}\NormalTok{(.}\DecValTok{1}\NormalTok{, }\DecValTok{1}\NormalTok{, }\AttributeTok{by =}\NormalTok{ .}\DecValTok{1}\NormalTok{))}
\end{Highlighting}
\end{Shaded}

\begin{verbatim}
##    10%    20%    30%    40%    50%    60%    70%    80%    90%   100% 
##  2.930  4.204  5.447  6.502  7.225  9.186 12.629 14.572 16.966 21.200
\end{verbatim}

\begin{Shaded}
\begin{Highlighting}[]
\FunctionTok{sort}\NormalTok{(predic.IDHM}\SpecialCharTok{$}\NormalTok{analfabetismo\_18\_anos)}
\end{Highlighting}
\end{Shaded}

\begin{verbatim}
##   [1]  1.68  1.83  1.90  1.95  1.95  1.97  2.13  2.14  2.25  2.27  2.31  2.41
##  [13]  2.59  2.61  2.61  2.63  2.67  2.69  2.71  2.71  2.75  2.79  2.84  2.84
##  [25]  2.85  2.86  2.90  2.93  2.93  2.94  2.94  2.94  3.01  3.02  3.08  3.10
##  [37]  3.14  3.18  3.26  3.27  3.28  3.30  3.31  3.35  3.36  3.37  3.65  3.69
##  [49]  3.72  3.79  3.84  4.05  4.05  4.12  4.13  4.18  4.21  4.25  4.29  4.50
##  [61]  4.53  4.56  4.63  4.64  4.64  4.72  4.72  4.74  4.76  4.79  4.80  4.95
##  [73]  4.97  5.02  5.05  5.07  5.24  5.24  5.26  5.31  5.32  5.35  5.36  5.44
##  [85]  5.45  5.48  5.53  5.56  5.58  5.60  5.62  5.66  5.67  5.81  5.85  5.91
##  [97]  5.92  5.93  5.95  5.98  5.99  6.01  6.04  6.08  6.19  6.24  6.37  6.40
## [109]  6.43  6.44  6.46  6.49  6.51  6.56  6.57  6.61  6.61  6.62  6.62  6.63
## [121]  6.63  6.70  6.72  6.72  6.78  6.81  6.82  6.87  6.94  7.03  7.03  7.03
## [133]  7.04  7.07  7.09  7.09  7.11  7.11  7.12  7.22  7.23  7.32  7.33  7.42
## [145]  7.45  7.47  7.57  7.65  7.73  7.73  7.87  7.88  7.89  7.96  7.97  8.21
## [157]  8.26  8.29  8.41  8.44  8.47  8.59  8.67  8.75  8.93  9.04  9.04  9.15
## [169]  9.24  9.50  9.60  9.74  9.90  9.98 10.29 10.36 10.44 10.52 10.75 11.04
## [181] 11.05 11.07 11.08 11.09 11.12 11.53 11.70 11.84 12.03 12.06 12.07 12.08
## [193] 12.12 12.18 12.40 12.62 12.65 12.89 12.90 13.03 13.13 13.15 13.19 13.23
## [205] 13.27 13.36 13.46 13.68 13.68 13.73 13.83 13.83 13.84 13.90 13.91 13.92
## [217] 13.95 14.03 14.07 14.07 14.14 14.39 14.51 14.54 14.70 14.72 14.98 15.16
## [229] 15.17 15.18 15.22 15.46 15.46 15.57 15.58 15.61 15.81 15.89 15.89 16.07
## [241] 16.12 16.19 16.29 16.36 16.40 16.40 16.45 16.64 16.77 16.80 16.85 16.96
## [253] 17.02 17.05 17.32 17.37 17.45 17.46 17.47 18.16 18.24 18.28 18.32 18.42
## [265] 18.68 18.72 18.73 18.99 19.25 19.28 19.37 19.86 20.16 20.35 20.35 20.54
## [277] 20.73 20.88 21.06 21.20
\end{verbatim}

\begin{Shaded}
\begin{Highlighting}[]
\FunctionTok{print}\NormalTok{(}\FunctionTok{paste0}\NormalTok{(}\StringTok{"analfabetismo\_15\_anos"}\NormalTok{))}
\end{Highlighting}
\end{Shaded}

\begin{verbatim}
## [1] "analfabetismo_15_anos"
\end{verbatim}

\begin{Shaded}
\begin{Highlighting}[]
\FunctionTok{quantile}\NormalTok{(predic.IDHM}\SpecialCharTok{$}\NormalTok{analfabetismo\_15\_anos, }\AttributeTok{probs =} \FunctionTok{seq}\NormalTok{(.}\DecValTok{1}\NormalTok{, }\DecValTok{1}\NormalTok{, }\AttributeTok{by =}\NormalTok{ .}\DecValTok{1}\NormalTok{))}
\end{Highlighting}
\end{Shaded}

\begin{verbatim}
##    10%    20%    30%    40%    50%    60%    70%    80%    90%   100% 
##  2.769  3.958  5.148  6.134  6.810  8.562 11.726 13.514 15.741 19.600
\end{verbatim}

\begin{Shaded}
\begin{Highlighting}[]
\FunctionTok{sort}\NormalTok{(predic.IDHM}\SpecialCharTok{$}\NormalTok{analfabetismo\_15\_anos)}
\end{Highlighting}
\end{Shaded}

\begin{verbatim}
##   [1]  1.60  1.77  1.84  1.89  1.89  1.94  2.06  2.07  2.14  2.18  2.23  2.32
##  [13]  2.46  2.48  2.50  2.51  2.54  2.57  2.57  2.58  2.62  2.66  2.68  2.69
##  [25]  2.69  2.71  2.76  2.76  2.77  2.77  2.80  2.82  2.84  2.88  2.93  2.97
##  [37]  2.97  3.00  3.10  3.12  3.12  3.13  3.16  3.19  3.20  3.20  3.43  3.49
##  [49]  3.51  3.63  3.67  3.85  3.87  3.90  3.95  3.95  3.96  4.03  4.05  4.24
##  [61]  4.30  4.32  4.36  4.41  4.41  4.46  4.46  4.50  4.53  4.56  4.56  4.61
##  [73]  4.70  4.73  4.78  4.82  4.87  4.94  4.97  4.98  5.00  5.00  5.00  5.05
##  [85]  5.19  5.19  5.21  5.23  5.23  5.31  5.33  5.34  5.39  5.41  5.46  5.48
##  [97]  5.57  5.59  5.59  5.60  5.61  5.64  5.64  5.76  5.77  5.85  5.89  5.90
## [109]  6.00  6.07  6.09  6.11  6.15  6.16  6.17  6.20  6.20  6.20  6.21  6.22
## [121]  6.24  6.27  6.27  6.28  6.37  6.37  6.40  6.41  6.41  6.44  6.46  6.47
## [133]  6.54  6.54  6.57  6.61  6.62  6.68  6.69  6.81  6.81  6.81  6.84  6.85
## [145]  6.96  7.06  7.07  7.18  7.22  7.22  7.27  7.28  7.29  7.31  7.50  7.63
## [157]  7.75  7.77  7.79  7.84  7.93  7.93  8.13  8.21  8.33  8.41  8.53  8.55
## [169]  8.58  8.89  8.90  8.99  9.20  9.30  9.70  9.74  9.80  9.80 10.11 10.20
## [181] 10.21 10.31 10.37 10.46 10.59 10.75 10.95 11.14 11.16 11.17 11.31 11.32
## [193] 11.44 11.48 11.49 11.63 11.95 12.08 12.12 12.15 12.18 12.29 12.32 12.48
## [205] 12.48 12.51 12.54 12.63 12.76 12.93 12.95 12.98 13.00 13.01 13.05 13.06
## [217] 13.16 13.19 13.23 13.23 13.27 13.38 13.45 13.48 13.65 13.72 14.10 14.14
## [229] 14.19 14.23 14.28 14.29 14.34 14.35 14.42 14.61 14.67 14.72 14.89 14.98
## [241] 15.10 15.20 15.25 15.28 15.30 15.40 15.42 15.49 15.56 15.61 15.72 15.73
## [253] 15.84 15.93 16.17 16.20 16.23 16.24 16.30 17.01 17.02 17.03 17.05 17.15
## [265] 17.19 17.32 17.49 17.68 17.70 17.99 17.99 18.38 18.70 18.87 18.97 19.02
## [277] 19.32 19.33 19.40 19.60
\end{verbatim}

\begin{Shaded}
\begin{Highlighting}[]
\CommentTok{\#adaptação do código desenvolvido por Anne (2020).}
\end{Highlighting}
\end{Shaded}

Todos os insights e observações adquiridos durante a AED são
documentados de forma organizada e detalhada, criando um relatório que
se torna parte integrante da metodologia de pesquisa. Esses insights
orientam o trabalho subsequente na seleção, implementação e otimização
do modelo de Random Forest, permitindo abordar o processo de modelagem
com uma compreensão clara e detalhada dos dados que estão sendo
manipulados.

\textbf{4.2. DESENVOLVIMENTO E TREINAMENTO DO MODELO}

A fase de desenvolvimento e treinamento do modelo é um componente
indispensável na sequência metodológica desta investigação. Nesta etapa,
optou-se por adotar o modelo de Aprendizado de Máquina conhecido como
Random Forest devido às suas notáveis vantagens.

O Random Forest é particularmente adequado para lidar com conjuntos de
dados que possuem relações complexas, o modelo demonstra uma robustez
contra o fenômeno de overfitting, ou seja, quando o modelo aprende tão
especificamente a partir dos dados de treinamento que se torna
ineficiente para prever dados desconhecidos.

Para preparar o ambiente de treinamento, divide-se os dados
pré-processados em dois conjuntos: treinamento e teste. Esta prática
comum é essencial para avaliar o desempenho e a capacidade de
generalização do modelo, fornecendo uma estimativa robusta da sua
performance quando aplicado a novos dados. O conjunto de treinamento é
utilizado para ajustar o modelo, enquanto o conjunto de teste é usado
para avaliar sua performance.

Vale ressaltar que o Random Forest não é o único modelo que pode ser
aplicado nesse cenário. Outros modelos de aprendizado de máquina, tais
como regressão linear, máquinas de vetores de suporte (SVM) ou redes
neurais, podem ser considerados dependendo das características
intrínsecas dos dados e do contexto específico do problema a ser
resolvido. A escolha do modelo adequado envolve uma consideração
cuidadosa do equilíbrio entre capacidade do modelo, interpretabilidade e
eficiência computacional.

Um elemento chave neste estágio é a sintonização de hiperparâmetros, um
processo de ajuste fino que envolve a manipulação cuidadosa de
diferentes parâmetros dentro do modelo. Ajustes nesses hiperparâmetros
podem ter impactos significativos na capacidade do modelo de aprender
com eficácia a partir dos dados, equilibrando a complexidade do modelo
para evitar tanto o underfitting (quando o modelo é muito simples e não
consegue capturar a complexidade dos dados) quanto o overfitting (quando
o modelo é muito complexo e se ajusta excessivamente aos dados de
treinamento, perdendo capacidade de generalização).

Além disso, a experimentação com diferentes configurações de modelo é
uma parte integral deste processo. Por exemplo, ao alterar a estrutura
ou a profundidade das árvores na Random Forest, é possível observar
diferentes desempenhos do modelo. Tais experimentações permitem explorar
uma variedade de cenários e garantir que estão à maximizar a capacidade
do modelo de capturar e aprender a partir dos padrões subjacentes nos
dados.

Por fim, é imprescindível manter uma mente aberta para a avaliação de
alternativas de modelos. Embora o Random Forest tenha sido inicialmente
escolhido por suas vantagens, outros modelos, como regressão linear,
máquinas de vetores de suporte (SVM) ou redes neurais, podem vir a ser
mais adequados conforme as características dos dados se revelam ao longo
da investigação. A adoção de um enfoque flexível e adaptativo para o
desenvolvimento do modelo permite que esteja sempre pronto para
responder a novos desafios e descobertas, buscando continuamente
aprimorar a qualidade e robustez dos resultados.

No momento do desenvolvimento do modelo de aprendizado de máquina foi
utilizado quatro pacotes: ``plyr'', ``caret'', ``randomForest'' e o
``caTools''. o ``plyr'' já fora mencionado anteriormente na Engenharia
de Dados.

O pacote ``caret'' (KUHN, 2008), oferece uma abordagem integrada ao
treinamento e avaliação de modelos de Machine Learning. Com ele, é
possível treinar uma variedade de modelos usando uma única interface,
facilitando o processo de modelagem. Além disso, fornece ferramentas
para avaliação rigorosa, pré-processamento de dados e visualização de
resultados, a manutenção do pacote por Max Kuhn, e desenvolvido com
parcerias para o R (KUHN et al.,s.d.).

O pacote ``randomForest'' desenvolvido com a importação do modelo de
Breiman (2001), por Andy Liaw and Matthew Wiener (LIAW; WIENER, s.d.), é
um pacote para classificação e regressão em análises estatísticas.
Utilizando uma combinação de árvores de decisão com entradas aleatórias,
este pacote permite estimativas mais robustas e precisas. O pacote
inicialmente publicado no Rnews em dezembro de 2002 (LIAW; WIENER, 2002)
com os autores detalhando o uso e funcionamento do pacote.

O pacote ``caTools'' (TUSZYNSKI, 2021) é uma ferramenta versátil que
fornece uma série de funções utilitárias, essenciais para a análise de
dados e complementares a métodos como o Random Forest, de autoria de
Jarek Tuszynski. Estes pacotes foram utilizados na construção e
avaliação do modelo.

Primeiramente, é feita uma divisão de variáveis para treino e teste, no
caso do IDHM, uma separação de 80 \% para treino e 20\% para teste.

\begin{Shaded}
\begin{Highlighting}[]
\FunctionTok{library}\NormalTok{(plyr)}
\FunctionTok{library}\NormalTok{(caret)}
\end{Highlighting}
\end{Shaded}

\begin{verbatim}
## Warning: package 'caret' was built under R version 4.3.1
\end{verbatim}

\begin{verbatim}
## Carregando pacotes exigidos: lattice
\end{verbatim}

\begin{Shaded}
\begin{Highlighting}[]
\FunctionTok{library}\NormalTok{(randomForest)}
\end{Highlighting}
\end{Shaded}

\begin{verbatim}
## Warning: package 'randomForest' was built under R version 4.3.1
\end{verbatim}

\begin{verbatim}
## randomForest 4.7-1.1
\end{verbatim}

\begin{verbatim}
## Type rfNews() to see new features/changes/bug fixes.
\end{verbatim}

\begin{verbatim}
## 
## Attaching package: 'randomForest'
\end{verbatim}

\begin{verbatim}
## The following object is masked from 'package:dplyr':
## 
##     combine
\end{verbatim}

\begin{verbatim}
## The following object is masked from 'package:ggplot2':
## 
##     margin
\end{verbatim}

\begin{Shaded}
\begin{Highlighting}[]
\FunctionTok{library}\NormalTok{(caTools)}
\end{Highlighting}
\end{Shaded}

\begin{verbatim}
## Warning: package 'caTools' was built under R version 4.3.1
\end{verbatim}

\begin{Shaded}
\begin{Highlighting}[]
\CommentTok{\#adaptação do código desenvolvido por Anne (2020).}
\end{Highlighting}
\end{Shaded}

\begin{Shaded}
\begin{Highlighting}[]
\CommentTok{\#usar o data frame fina da análise exploratória (predic.IDHM), e selecionar os dados para treino (80\%) e teste (20\%)}

\FunctionTok{set.seed}\NormalTok{(}\DecValTok{123}\NormalTok{)}
\NormalTok{amostra.IDHM }\OtherTok{\textless{}{-}}\NormalTok{ predic.IDHM}\SpecialCharTok{$}\NormalTok{IDHM }\SpecialCharTok{\%\textgreater{}\%}
  \FunctionTok{createDataPartition}\NormalTok{(}\AttributeTok{p =} \FloatTok{0.8}\NormalTok{, }\AttributeTok{list =} \ConstantTok{FALSE}\NormalTok{)}
\NormalTok{treino.IDHM  }\OtherTok{\textless{}{-}}\NormalTok{ predic.IDHM[amostra.IDHM, ]}
\NormalTok{teste.IDHM }\OtherTok{\textless{}{-}}\NormalTok{ predic.IDHM[}\SpecialCharTok{{-}}\NormalTok{amostra.IDHM, ]}
\CommentTok{\#adaptação do código desenvolvido por Anne (2020).}
\end{Highlighting}
\end{Shaded}

No caso da predição do IDHM, o modelo Random Forest operou construindo
múltiplas árvores de decisão durante o treinamento. Cada árvore é gerada
a partir de uma amostra bootstrap do conjunto de dados, (amostra
retirada com reposição). Entretanto, em vez de considerar todas as
características para a divisão em cada nó, o modelo seleciona um
subconjunto aleatório de características, determinado pelo parâmetro
``mtry''. Essas técnicas de aleatoriedade têm dois propósitos
principais: Redução da Variação: Ao construir árvores em diferentes
amostras de dados, a variabilidade inerente a cada árvore é ``média'',
levando a uma estimativa final mais estável e robusta; Descorrelacionado
Árvores: Ao usar subconjuntos aleatórios de características, cada árvore
tem um ``panorama'' diferente dos dados. Isso torna as árvores menos
correlacionadas entre si e, consequentemente, a agregação das árvores
oferece uma melhora na performance preditiva.

O código do modelo foi desenvolvido com um uso arbitrário de 500
árvores, e o uma utilização arbitrária de uma seleção de variáveis mtry
= 3. os resultados obtidos foram de 5,068181 x 10-5 de média dos
resíduos ao quadrado e 97,38 \% de explicação da variação.

\begin{Shaded}
\begin{Highlighting}[]
\CommentTok{\# random forest para regressão, iniciando com 500 arvores e mtry of 3}
\NormalTok{IDHM.model}\FloatTok{.1} \OtherTok{\textless{}{-}} \FunctionTok{randomForest}\NormalTok{(IDHM }\SpecialCharTok{\textasciitilde{}}\NormalTok{ ., }\AttributeTok{data =}\NormalTok{ treino.IDHM, }\AttributeTok{ntree=}\DecValTok{500}\NormalTok{, }\AttributeTok{mtry =} \DecValTok{3}\NormalTok{, }
                         \AttributeTok{importance =} \ConstantTok{TRUE}\NormalTok{, }\AttributeTok{na.action =}\NormalTok{ na.omit) }
\FunctionTok{print}\NormalTok{(IDHM.model}\FloatTok{.1}\NormalTok{) }
\end{Highlighting}
\end{Shaded}

\begin{verbatim}
## 
## Call:
##  randomForest(formula = IDHM ~ ., data = treino.IDHM, ntree = 500,      mtry = 3, importance = TRUE, na.action = na.omit) 
##                Type of random forest: regression
##                      Number of trees: 500
## No. of variables tried at each split: 3
## 
##           Mean of squared residuals: 5.068181e-05
##                     % Var explained: 97.6
\end{verbatim}

\begin{Shaded}
\begin{Highlighting}[]
\CommentTok{\# Plotar erro vs numero de arvores }
\FunctionTok{plot}\NormalTok{(IDHM.model}\FloatTok{.1}\NormalTok{, }\AttributeTok{main =} \StringTok{""}\NormalTok{)}

\CommentTok{\# Adicionar título e fonte no estilo ABNT}
\FunctionTok{title}\NormalTok{(}\AttributeTok{main =} \StringTok{"Gráfico 4 {-} Erro vs Número de Árvores no Modelo 1"}\NormalTok{, }\AttributeTok{adj =} \DecValTok{0}\NormalTok{)}
\FunctionTok{mtext}\NormalTok{(}\StringTok{"Fonte: Elaboração própria com dados do Atlas Brasil"}\NormalTok{, }\AttributeTok{side=}\DecValTok{1}\NormalTok{, }\AttributeTok{line=}\DecValTok{4}\NormalTok{, }\AttributeTok{adj =} \DecValTok{0}\NormalTok{, }\AttributeTok{cex=}\FloatTok{0.8}\NormalTok{)}
\end{Highlighting}
\end{Shaded}

\includegraphics{ML-regressão-IDHM_files/figure-latex/montagem do modelo e performance-1.pdf}

\begin{Shaded}
\begin{Highlighting}[]
\CommentTok{\#adaptação do código desenvolvido por Anne (2020).}
\end{Highlighting}
\end{Shaded}

O Gráfico feito com a função ``plot'', serve para dar indicações sobre a
evolução do erro conforme se adicionam árvores ao modelo.

Para otimizar o valor de mtry, utiliza a função tuneRF do pacote
randomForest no Rstudio. Esta função ajusta sequencialmente modelos de
Floresta Aleatória com diferentes valores de mtry, avaliando cada modelo
usando a taxa de erro fora da bolsa (OOB). Especificamente, tuneRF
inicia com um valor de mtry e ajusta incrementos e decrementos,
reavaliando o desempenho do modelo a cada passo, até que a melhoria no
erro OOB seja menor que um limite pré-definido.

A função permite, assim, identificar o valor de mtry que minimiza o erro
de previsão OOB, fornecendo uma orientação robusta para a configuração
ideal deste hiperparâmetro em nosso conjunto de dados, no caso o valor
de mtry encontrado na função ``tuneRF'' foi 3. O que sugere que o modelo
já está no melhor hiperparâmetro. Mas optou-se por fazer um modelo dois
com hiperparâmetro mtry = 4, para fins didáticos de comparação de
modelos.

\begin{Shaded}
\begin{Highlighting}[]
\CommentTok{\#Usar tuneRF para determinar se há melhor mtry na tentativa de encontrar o valor que produz o menor erro.}
\NormalTok{mtry }\OtherTok{\textless{}{-}} \FunctionTok{tuneRF}\NormalTok{(treino.IDHM[}\SpecialCharTok{{-}}\DecValTok{6}\NormalTok{],treino.IDHM}\SpecialCharTok{$}\NormalTok{IDHM, }\AttributeTok{ntreeTry=}\DecValTok{500}\NormalTok{,}
               \AttributeTok{stepFactor=}\DecValTok{1}\NormalTok{,}\AttributeTok{improve=}\FloatTok{0.01}\NormalTok{, }\AttributeTok{trace=}\ConstantTok{TRUE}\NormalTok{, }\AttributeTok{plot=}\ConstantTok{FALSE}\NormalTok{)}
\end{Highlighting}
\end{Shaded}

\begin{verbatim}
## mtry = 3  OOB error = 1.782552e-05 
## Searching left ...
## Searching right ...
\end{verbatim}

\begin{Shaded}
\begin{Highlighting}[]
\FunctionTok{print}\NormalTok{(mtry)}
\end{Highlighting}
\end{Shaded}

\begin{verbatim}
##   mtry     OOBError
## 3    3 1.782552e-05
\end{verbatim}

\begin{Shaded}
\begin{Highlighting}[]
\CommentTok{\#adaptação do código desenvolvido por Anne (2020).}
\end{Highlighting}
\end{Shaded}

Com a determinação de 4 para o hiperparâmetro mtry. A criação de um
segundo modelo com mtry = 4 para fazer avaliação dos modelos e
determinar o modelo com melhor desempenho.

\begin{Shaded}
\begin{Highlighting}[]
\CommentTok{\#o valor ótimo para mtry é 4, que produz o menor erro.}

\FunctionTok{set.seed}\NormalTok{(}\DecValTok{123}\NormalTok{)}
\CommentTok{\# random forest para regressão com mtry=4}
\NormalTok{IDHM.model}\FloatTok{.2} \OtherTok{\textless{}{-}} \FunctionTok{randomForest}\NormalTok{(IDHM }\SpecialCharTok{\textasciitilde{}}\NormalTok{ ., }\AttributeTok{data =}\NormalTok{ treino.IDHM, }\AttributeTok{ntree=}\DecValTok{500}\NormalTok{, }\AttributeTok{mtry =} \DecValTok{4}\NormalTok{, }
                         \AttributeTok{importance =} \ConstantTok{TRUE}\NormalTok{, }\AttributeTok{na.action =}\NormalTok{ na.omit) }
\FunctionTok{print}\NormalTok{(IDHM.model}\FloatTok{.2}\NormalTok{) }
\end{Highlighting}
\end{Shaded}

\begin{verbatim}
## 
## Call:
##  randomForest(formula = IDHM ~ ., data = treino.IDHM, ntree = 500,      mtry = 4, importance = TRUE, na.action = na.omit) 
##                Type of random forest: regression
##                      Number of trees: 500
## No. of variables tried at each split: 4
## 
##           Mean of squared residuals: 4.829864e-05
##                     % Var explained: 97.71
\end{verbatim}

\begin{Shaded}
\begin{Highlighting}[]
\CommentTok{\# Plot the error vs the number of trees graph }
\FunctionTok{plot}\NormalTok{(IDHM.model}\FloatTok{.2}\NormalTok{, }\AttributeTok{main =} \StringTok{""}\NormalTok{)}

\CommentTok{\# Adicionar título e fonte no estilo ABNT}
\FunctionTok{title}\NormalTok{(}\AttributeTok{main =} \StringTok{"Gráfico 5 {-} Erro vs Número de Árvores no Modelo 2"}\NormalTok{, }\AttributeTok{adj =} \DecValTok{0}\NormalTok{)}
\FunctionTok{mtext}\NormalTok{(}\StringTok{"Fonte: Elaboração própria com dados do Atlas Brasil"}\NormalTok{, }\AttributeTok{side=}\DecValTok{1}\NormalTok{, }\AttributeTok{line=}\DecValTok{4}\NormalTok{, }\AttributeTok{adj =} \DecValTok{0}\NormalTok{, }\AttributeTok{cex=}\FloatTok{0.8}\NormalTok{)}
\end{Highlighting}
\end{Shaded}

\includegraphics{ML-regressão-IDHM_files/figure-latex/criação de um segundo modelo com mtry 4-1.pdf}

\begin{Shaded}
\begin{Highlighting}[]
\CommentTok{\#adaptação do código desenvolvido por Anne (2020).}
\end{Highlighting}
\end{Shaded}

Com isso, observa-se uma melhora na porcentagem de explicação do segundo
modelo saindo de 97,6\% do modelo IDHM.model.1 para o IDHM.model.2 com
97,71\% de explicação da variação. Com isso, é possível ir para a
avaliação do modelo.

\begin{enumerate}
\def\labelenumi{\arabic{enumi}.}
\setcounter{enumi}{4}
\tightlist
\item
  \textbf{AVALIAÇÃO DO MODELO E RESULTADOS}
\end{enumerate}

A avaliação de modelos na ciência de dados é, sem dúvida, uma etapa de
alta importância, que visa à aferição de eficácia do modelo em questão.
Essa fase é frequentemente realizada após o treinamento do modelo e tem
como objetivo fundamental validar o desempenho do modelo frente a dados
ainda não vistos por este durante o processo de treinamento. A avaliação
do desempenho do modelo vai além de uma simples confirmação de sua
funcionalidade, ela auxilia na compreensão de como as predições foram
feitas e ajuda a elucidar a eficiência do algoritmo na prática.

No contexto da pesquisa, o modelo de aprendizado de máquina adotado é o
Random Forest, um algoritmo capaz de gerenciar relações complexas, além
de resistir ao overfitting. A avaliação desse modelo, então, é realizada
testando-o em um conjunto de dados que foi previamente separado,
denominado conjunto de testes. Essa separação do conjunto de dados em
treinamento e teste é essencial para que a avaliação seja autêntica e
sem viés, fornecendo uma estimativa não tendenciosa do desempenho do
modelo.

A escolha da métrica de avaliação a ser usada é crucial e depende
diretamente do tipo de tarefa de previsão proposta. Para tarefas de
regressão, onde o objetivo é a previsão de um valor contínuo, a métrica
comumente utilizada é o Erro Quadrático Médio (MSE). Esta métrica avalia
a média dos quadrados das diferenças entre os valores verdadeiros e os
valores previstos pelo modelo. Ao fazer isso, fornece uma medida
numérica direta do desempenho do modelo.

Um erro quadrático médio alto não indica necessariamente um modelo ruim,
assim como um erro baixo não garante um bom modelo. A interpretação
desse valor deve ser feita em relação à escala dos dados e ao contexto
específico do problema.

Em contraste, para tarefas de classificação, onde a finalidade é
atribuir uma classe a uma observação, várias outras métricas podem ser
relevantes, como a acurácia, que mede a proporção de acertos em relação
ao total, precisão, que considera a proporção de verdadeiros positivos
em relação a todos os resultados positivos, recall, que considera a
proporção de verdadeiros positivos em relação à soma dos verdadeiros
positivos e falsos negativos, e F1-score, uma média harmônica entre
precisão e recall que busca um equilíbrio entre essas duas métricas.

Outro aspecto importante a considerar é o equilíbrio entre o desempenho
no conjunto de treinamento e no conjunto de testes. Embora um modelo
possa ter um desempenho excepcional no conjunto de treinamento, se ele
não se sair bem no conjunto de teste, isso pode ser um sinal de
overfitting, ou seja, que o modelo se adaptou muito aos dados de
treinamento e não conseguiu generalizar bem para novos dados.

Por fim, é importante lembrar que nenhuma métrica ou procedimento de
avaliação pode garantir a eficácia de um modelo em todos os cenários
possíveis. Em última análise, o valor de um modelo depende de sua
capacidade de fazer previsões úteis em situações reais, o que pode
exigir testes adicionais e ajustes contínuos com base em feedback e
novos dados. Portanto, o processo de avaliação do modelo não é apenas
uma etapa no desenvolvimento do modelo, mas uma tarefa contínua que
continua mesmo após o modelo ser colocado em uso.

Ao caso do IDHM, a avaliação começou gerando as previsões do modelo
IDHM.model.1 e armazenando na variável IDHM.predições.1. A fim de obter
uma visão inicial das previsões, visualizou-se as estimativas de IDHM
para os seis primeiros registros do conjunto de teste, indicadas pelos
índices 4, 5, 6, 21, 25 e 26, e seus valores preditos correspondentes.
Em seguida, o mesmo processo foi feito para o modelo IDHM.model.2,
armazenados na variável IDHM.predições.2.

Em seguida foi calculado o Erro Quadrático Médio da Raiz (RMSE), métrico
padrão utilizada em regressão para avaliar a diferença do valor previsto
e o valor observado. O cálculo do RMSE segue a equação do Erro
Quadrático Médio MSE, no entanto ao final é calculada a raiz quadrada do
Erro Quadrático Médio. As variáveis continuam as mesmas no RMSE.

MSE=1/n ∑\emph{(i=1)\textsuperscript{n▒(yi-ŷi)}2}

\emph{n = número total de observações}

\emph{y = valor real da i-ésima observação}

\emph{ŷ = valor previsto da i-ésima observação}

\emph{RMSE= √(1/n ∑}(i =1)\textsuperscript{n▒(yi-ŷi)}2 )

O resultado dos RMSE dos modelos foi obtido com a função RSME, e os
valores foram de 0.008417118 para o modelo 1 e de 0.00817621 para o
modelo 2.

\begin{Shaded}
\begin{Highlighting}[]
\CommentTok{\# Fazer predições com dados de teste usando modelo 1 (mtry = 3)}
\NormalTok{IDHM.predições}\FloatTok{.1} \OtherTok{\textless{}{-}}\NormalTok{ IDHM.model}\FloatTok{.1} \SpecialCharTok{\%\textgreater{}\%} \FunctionTok{predict}\NormalTok{(teste.IDHM)}
\FunctionTok{head}\NormalTok{(IDHM.predições}\FloatTok{.1}\NormalTok{)}
\end{Highlighting}
\end{Shaded}

\begin{verbatim}
##         4         5         6        21        25        26 
## 0.7146746 0.7019420 0.6878327 0.7019142 0.7868317 0.7983661
\end{verbatim}

\begin{Shaded}
\begin{Highlighting}[]
\CommentTok{\# Fazer predições com dados de teste usando modelo 2 (mtry = 4)}
\NormalTok{IDHM.predições}\FloatTok{.2} \OtherTok{\textless{}{-}}\NormalTok{ IDHM.model}\FloatTok{.2} \SpecialCharTok{\%\textgreater{}\%} \FunctionTok{predict}\NormalTok{(teste.IDHM)}
\FunctionTok{head}\NormalTok{(IDHM.predições}\FloatTok{.2}\NormalTok{)}
\end{Highlighting}
\end{Shaded}

\begin{verbatim}
##         4         5         6        21        25        26 
## 0.7141945 0.7021202 0.6884029 0.7004927 0.7869232 0.8016628
\end{verbatim}

\begin{Shaded}
\begin{Highlighting}[]
\CommentTok{\# Calcular o erro médio de previsão {-}{-} erro quadrático médio da raiz (RMSE) de ambos os modelos}
\FunctionTok{RMSE}\NormalTok{(IDHM.predições}\FloatTok{.1}\NormalTok{, teste.IDHM}\SpecialCharTok{$}\NormalTok{IDHM)}
\end{Highlighting}
\end{Shaded}

\begin{verbatim}
## [1] 0.008417118
\end{verbatim}

\begin{Shaded}
\begin{Highlighting}[]
\FunctionTok{RMSE}\NormalTok{(IDHM.predições}\FloatTok{.2}\NormalTok{, teste.IDHM}\SpecialCharTok{$}\NormalTok{IDHM)}
\end{Highlighting}
\end{Shaded}

\begin{verbatim}
## [1] 0.00817621
\end{verbatim}

\begin{Shaded}
\begin{Highlighting}[]
\CommentTok{\#adaptação do código desenvolvido por Anne (2020).}
\end{Highlighting}
\end{Shaded}

Durante a modelagem do IDHM com o algoritmo Random Forest, a importância
das variáveis foi mensurada através de métricas específicas. Utilizamos
a métrica ``Percentagem de Incremento do Erro Quadrático Médio''
(\%IncMSE) para entender o quanto o erro do modelo é afetado ao alterar
aleatoriamente os valores de uma variável. Adicionalmente, empregamos a
métrica ``Incremento da Pureza do Nó'' (IncNodePurity) para avaliar o
impacto da variável na homogeneidade dos nós resultantes de divisões.
Através dessa análise, identificamos que as variáveis: renda per capita,
subíndice de frequência de escolaridade e esperança de vida exibiram uma
importância destacada, com elevados valores em ambas as métricas. Em
contraste, a variável ano mostrou ter um peso menor na modelagem,
indicando uma relevância reduzida na predição do IDHM.

\begin{Shaded}
\begin{Highlighting}[]
\CommentTok{\#avaliar a importância das variáveis para o modelo }
\FunctionTok{importance}\NormalTok{(IDHM.model}\FloatTok{.1}\NormalTok{)}
\end{Highlighting}
\end{Shaded}

\begin{verbatim}
##                         %IncMSE IncNodePurity
## ano                    7.868941   0.002079949
## renda_per_capita      23.030204   0.136166918
## sub_esco_pop          20.115061   0.016921115
## sub_freq_esco         21.777708   0.029760092
## esperança_de_vida     19.615980   0.045243865
## porcent_pobres        13.947132   0.063730867
## mortalidade_infantil  16.330721   0.056961273
## media_anos_de_estudo  19.724511   0.025218818
## analfabetismo_25_anos 12.473426   0.040472876
## analfabetismo_18_anos 13.835095   0.030522313
## analfabetismo_15_anos 11.938146   0.024789071
\end{verbatim}

\begin{Shaded}
\begin{Highlighting}[]
\FunctionTok{varImpPlot}\NormalTok{(IDHM.model}\FloatTok{.1}\NormalTok{, }\AttributeTok{main =} \StringTok{""}\NormalTok{)}

\CommentTok{\# Adicionar título e fonte no estilo ABNT}
\FunctionTok{title}\NormalTok{(}\AttributeTok{main =} \StringTok{"Gráfico 6 {-} Importância das Variáveis no Modelo Random Forest"}\NormalTok{, }\AttributeTok{adj =} \DecValTok{0}\NormalTok{)}
\FunctionTok{mtext}\NormalTok{(}\StringTok{"Fonte: Elaboração própria com dados do Atlas Brasil"}\NormalTok{, }\AttributeTok{side=}\DecValTok{1}\NormalTok{, }\AttributeTok{line=}\DecValTok{4}\NormalTok{, }\AttributeTok{adj =} \DecValTok{0}\NormalTok{, }\AttributeTok{cex=}\FloatTok{0.8}\NormalTok{)}
\end{Highlighting}
\end{Shaded}

\includegraphics{ML-regressão-IDHM_files/figure-latex/Resultado dos Incrementos do Erro Quadrático Médio e do Nó de Pureza.-1.pdf}

\begin{Shaded}
\begin{Highlighting}[]
\CommentTok{\#adaptação do código desenvolvido por Anne (2020).}
\end{Highlighting}
\end{Shaded}

Para avaliar a performance do modelo Random Forest na predição do IDHM,
foi necessário confrontar as predições geradas pelo modelo com os
valores reais de IDHM. Primeiramente, as predições, originalmente na
forma de um vetor, foram convertidas em um data frame denominado
IDHM.predições.df Posteriormente, com a finalidade de facilitar a
comparação, o conjunto de testes (teste.IDHM) foi mesclado com o data
frame das predições, utilizando os índices dos registros como chave de
junção.

Com os dados alinhados, uma nova coluna foi calculada, nomeada diff, a
qual representa a diferença entre os valores reais do IDHM e as
respectivas predições. Esta coluna proporciona uma visão imediata do
erro associado a cada predição, permitindo uma análise mais granular da
performance do modelo. Por fim, o data frame resultante
IDHM.predições.df fornece um panorama completo das predições, valores
reais e diferenças associadas, constituindo uma ferramenta valiosa para
avaliação e refinamento do modelo.

\begin{Shaded}
\begin{Highlighting}[]
\CommentTok{\# Converter predições para um data frame }
\NormalTok{IDHM.predições.df }\OtherTok{\textless{}{-}} \FunctionTok{as.data.frame}\NormalTok{(IDHM.predições}\FloatTok{.1}\NormalTok{)}
\CommentTok{\# Mesclar com base no índice}
\NormalTok{IDHM.predições.df }\OtherTok{\textless{}{-}} \FunctionTok{merge}\NormalTok{(teste.IDHM, IDHM.predições.df, }\AttributeTok{by.x =} \DecValTok{0}\NormalTok{, }\AttributeTok{by.y =} \DecValTok{0}\NormalTok{, }\AttributeTok{all.x =} \ConstantTok{TRUE}\NormalTok{, }\AttributeTok{all.y =} \ConstantTok{TRUE}\NormalTok{)}
\CommentTok{\# Criar uma nova coluna calculada com a diferença da predição do IDHM, e o valor Real}
\NormalTok{IDHM.predições.df}\SpecialCharTok{$}\NormalTok{diff }\OtherTok{\textless{}{-}} \FunctionTok{with}\NormalTok{(IDHM.predições.df, IDHM.predições.df}\SpecialCharTok{$}\NormalTok{IDHM }\SpecialCharTok{{-}}\NormalTok{ IDHM.predições.df}\SpecialCharTok{$}\NormalTok{IDHM.predições}\FloatTok{.1}\NormalTok{)}
\CommentTok{\# Obter a média da diferença}
\NormalTok{IDHM.predições.df}
\end{Highlighting}
\end{Shaded}

\begin{verbatim}
##    Row.names  ano renda_per_capita sub_esco_pop sub_freq_esco esperança_de_vida
## 1        117 2016           467.19        0.681         0.717             71.85
## 2        119 2016           473.32        0.585         0.779             73.72
## 3        120 2016          1456.83        0.813         0.826             78.02
## 4        121 2016           738.04        0.654         0.774             78.22
## 5        122 2016           722.67        0.655         0.809             74.09
## 6        131 2016           469.93        0.513         0.719             70.94
## 7        137 2016           921.16        0.686         0.840             79.09
## 8        147 2017           503.66        0.585         0.791             73.94
## 9        152 2017           791.40        0.628         0.752             75.74
## 10       153 2017           766.18        0.652         0.816             74.44
## 11       154 2017           757.43        0.631         0.821             77.49
## 12       157 2017           447.25        0.595         0.691             72.22
## 13       163 2017           596.82        0.598         0.756             71.50
## 14       168 2017           575.58        0.631         0.783             73.56
## 15       174 2018           499.19        0.564         0.714             73.87
## 16       182 2018           767.86        0.623         0.820             77.73
## 17       196 2018           617.00        0.618         0.811             73.81
## 18       207 2019           366.24        0.578         0.777             71.39
## 19        21 2012           478.88        0.519         0.673             74.59
## 20       216 2019          1044.95        0.760         0.790             77.04
## 21       217 2019           595.13        0.599         0.739             76.29
## 22       218 2019          1047.74        0.673         0.768             78.55
## 23       224 2019           598.77        0.630         0.813             74.05
## 24       225 2020           779.13        0.695         0.807             76.21
## 25       229 2020           478.39        0.727         0.759             71.99
## 26       230 2020           547.09        0.617         0.729             73.66
## 27       234 2020           713.76        0.692         0.844             70.76
## 28       239 2020           497.15        0.568         0.732             73.93
## 29       244 2020           961.32        0.775         0.784             73.24
## 30       245 2020           603.61        0.627         0.756             76.28
## 31       247 2020           640.90        0.652         0.793             71.20
## 32        25 2012           927.60        0.626         0.805             77.70
## 33       252 2020           588.48        0.664         0.818             74.02
## 34       257 2021           432.99        0.739         0.710             69.61
## 35        26 2012          1037.42        0.703         0.840             76.80
## 36       263 2021           341.32        0.618         0.770             67.90
## 37       265 2021           707.48        0.706         0.785             68.77
## 38       274 2021           944.53        0.708         0.772             72.84
## 39        34 2013           508.14        0.520         0.635             72.66
## 40        36 2013          1568.87        0.783         0.799             77.17
## 41         4 2012           528.23        0.670         0.653             72.77
## 42        45 2013           484.80        0.534         0.624             71.44
## 43        47 2013           503.95        0.496         0.699             70.38
## 44        48 2013           888.32        0.698         0.723             75.23
## 45         5 2012           559.44        0.613         0.642             70.85
## 46        55 2013           567.26        0.536         0.631             71.76
## 47        56 2013           550.02        0.599         0.742             72.42
## 48         6 2012           503.29        0.510         0.639             72.39
## 49        60 2014           575.63        0.687         0.689             73.37
## 50        65 2014           807.78        0.633         0.751             77.54
## 51        75 2014           502.70        0.507         0.718             70.57
## 52        90 2015           520.00        0.556         0.687             73.16
## 53        91 2015           478.58        0.562         0.763             73.49
## 54        93 2015           758.07        0.645         0.770             77.89
##    porcent_pobres mortalidade_infantil media_anos_de_estudo
## 1           25.19                18.34                 9.11
## 2           21.93                14.48                 7.65
## 3            4.41                10.57                11.19
## 4            8.21                 8.81                 9.16
## 5            5.77                15.02                 9.04
## 6           22.82                19.25                 7.34
## 7            2.86                 9.23                 9.49
## 8           20.72                13.86                 8.04
## 9            5.40                13.70                 9.21
## 10           6.07                16.60                 8.99
## 11           7.18                10.46                 9.04
## 12          20.54                16.18                 8.33
## 13           8.64                19.70                 8.36
## 14          11.66                15.44                 8.70
## 15          21.67                16.11                 8.12
## 16           6.71                10.02                 9.07
## 17          13.15                15.00                 8.74
## 18          30.26                18.71                 7.84
## 19          17.80                18.21                 7.44
## 20           6.64                10.46                10.76
## 21          19.57                13.14                 8.68
## 22           3.68                 8.88                 9.90
## 23          14.05                14.57                 8.89
## 24           9.41                14.02                 9.82
## 25          20.50                19.76                10.02
## 26          16.65                18.09                 8.72
## 27           4.87                17.45                 9.79
## 28          15.92                16.34                 8.29
## 29           8.20                13.06                10.93
## 30          12.32                15.39                 9.10
## 31           6.99                22.41                 9.24
## 32           3.00                10.51                 8.95
## 33          11.40                17.13                 9.24
## 34          25.79                21.66                10.10
## 35           3.86                11.25                 9.77
## 36          32.53                23.38                 8.32
## 37           6.39                22.35                 9.83
## 38           4.35                12.18                10.35
## 39          19.79                20.02                 7.34
## 40           4.20                11.28                10.95
## 41          18.37                24.33                 9.09
## 42          19.49                18.39                 7.64
## 43          19.16                21.24                 6.87
## 44           6.01                12.78                 9.80
## 45          22.23                20.88                 8.63
## 46          18.75                19.09                 7.41
## 47          13.17                17.55                 8.11
## 48          22.56                21.04                 7.25
## 49          13.70                23.88                 9.30
## 50           6.91                 9.67                 9.02
## 51          20.31                20.54                 7.11
## 52          18.03                18.24                 7.93
## 53          20.34                15.16                 7.48
## 54           8.13                 9.23                 9.07
##    analfabetismo_25_anos analfabetismo_18_anos analfabetismo_15_anos  IDHM
## 1                   8.63                  7.09                  6.57 0.711
## 2                  17.98                 15.16                 14.19 0.722
## 3                   3.17                  2.71                  2.57 0.847
## 4                   6.97                  5.95                  5.59 0.779
## 5                   7.07                  5.98                  5.61 0.764
## 6                  21.58                 18.28                 17.03 0.685
## 7                   3.31                  2.85                  2.69 0.814
## 8                  16.75                 14.07                 13.19 0.730
## 9                   5.86                  5.02                  4.73 0.762
## 10                  7.63                  6.49                  6.11 0.770
## 11                  6.95                  5.92                  5.59 0.784
## 12                 11.07                  9.24                  8.58 0.694
## 13                  8.56                  7.12                  6.62 0.721
## 14                 12.41                 10.44                  9.70 0.740
## 15                 15.34                 13.15                 12.32 0.710
## 16                  6.96                  6.08                  5.77 0.784
## 17                 11.60                  9.98                  9.30 0.749
## 18                 18.50                 15.46                 14.34 0.694
## 19                 19.25                 16.19                 15.25 0.695
## 20                  2.59                  2.31                  2.23 0.809
## 21                 15.14                 12.89                 12.29 0.742
## 22                  3.20                  2.84                  2.71 0.801
## 23                 11.17                  9.50                  8.90 0.751
## 24                  6.52                  5.67                  5.39 0.784
## 25                  5.99                  4.97                  4.61 0.727
## 26                 12.58                 11.04                 10.37 0.724
## 27                  4.85                  4.25                  4.05 0.758
## 28                 16.45                 13.83                 13.05 0.714
## 29                  2.17                  1.95                  1.89 0.785
## 30                 12.77                 11.12                 10.59 0.750
## 31                  5.61                  4.74                  4.46 0.739
## 32                  4.24                  3.65                  3.43 0.792
## 33                 10.59                  8.93                  8.33 0.755
## 34                  5.59                  4.64                  4.36 0.700
## 35                  4.22                  3.69                  3.49 0.812
## 36                 14.93                 12.40                 11.49 0.676
## 37                  5.36                  4.56                  4.30 0.736
## 38                  2.36                  2.14                  2.07 0.771
## 39                 18.76                 15.81                 14.89 0.680
## 40                  3.45                  2.86                  2.69 0.837
## 41                  7.93                  6.37                  5.76 0.707
## 42                 13.44                 11.07                 10.20 0.671
## 43                 24.96                 20.88                 19.32 0.679
## 44                  3.57                  3.14                  2.97 0.768
## 45                  9.46                  7.89                  7.22 0.691
## 46                 19.41                 16.12                 14.98 0.683
## 47                 15.35                 12.62                 11.63 0.719
## 48                 19.27                 16.29                 15.28 0.678
## 49                  8.91                  7.11                  6.54 0.725
## 50                  7.69                  6.57                  6.20 0.772
## 51                 23.78                 20.16                 18.87 0.686
## 52                 15.98                 13.68                 12.76 0.701
## 53                 19.35                 16.40                 15.30 0.715
## 54                  7.66                  6.63                  6.21 0.776
##    IDHM.predições.1          diff
## 1         0.7095903  0.0014097333
## 2         0.7150993  0.0069006667
## 3         0.8385089  0.0084910667
## 4         0.7800819 -0.0010818667
## 5         0.7665971 -0.0025971333
## 6         0.6886255 -0.0036255333
## 7         0.8100509  0.0039491333
## 8         0.7231550  0.0068450000
## 9         0.7685123 -0.0065123333
## 10        0.7674915  0.0025085333
## 11        0.7804120  0.0035880000
## 12        0.7030948 -0.0090948333
## 13        0.7264488 -0.0054487667
## 14        0.7371906  0.0028094000
## 15        0.7119679 -0.0019679000
## 16        0.7811909  0.0028091000
## 17        0.7407610  0.0082389667
## 18        0.6964571 -0.0024571000
## 19        0.7019142 -0.0069142333
## 20        0.8222143 -0.0132143333
## 21        0.7351314  0.0068686000
## 22        0.8130208 -0.0120208000
## 23        0.7404551  0.0105449000
## 24        0.7732565  0.0107435000
## 25        0.7297161 -0.0027161333
## 26        0.7258185 -0.0018185000
## 27        0.7605134 -0.0025134000
## 28        0.7206246 -0.0066246000
## 29        0.7920688 -0.0070688333
## 30        0.7399933  0.0100067333
## 31        0.7413534 -0.0023533667
## 32        0.7868317  0.0051683000
## 33        0.7424384  0.0125615667
## 34        0.7022753 -0.0022753000
## 35        0.7983661  0.0136339333
## 36        0.7008335 -0.0248335333
## 37        0.7493369 -0.0133369000
## 38        0.7891303 -0.0181302667
## 39        0.6903452 -0.0103452000
## 40        0.8314648  0.0055351667
## 41        0.7146746 -0.0076745667
## 42        0.6847285 -0.0137285333
## 43        0.6693097  0.0096902667
## 44        0.7758335 -0.0078334667
## 45        0.7019420 -0.0109420000
## 46        0.6890759 -0.0060759333
## 47        0.7146250  0.0043750000
## 48        0.6878327 -0.0098327000
## 49        0.7261022 -0.0011021667
## 50        0.7740489 -0.0020489333
## 51        0.6744937  0.0115063000
## 52        0.7059610 -0.0049609667
## 53        0.7103681  0.0046319333
## 54        0.7765281 -0.0005280667
\end{verbatim}

\begin{Shaded}
\begin{Highlighting}[]
\FunctionTok{mean}\NormalTok{(IDHM.predições.df[,}\StringTok{"diff"}\NormalTok{])}
\end{Highlighting}
\end{Shaded}

\begin{verbatim}
## [1] -0.00127523
\end{verbatim}

\begin{Shaded}
\begin{Highlighting}[]
\CommentTok{\#adaptação do código desenvolvido por Anne (2020).}
\end{Highlighting}
\end{Shaded}

O data frame com as previsões do modelo 1 foi extraído com a previsão e
o diferencial para cada observação. e a média da diferença foi de
-0.00127523, obtido pela função ``mean''.

Por fim, para finalizar a avaliação do modelo é feito um gráfico
plotando os valores reais e os valores previstos para o IDHM.

\begin{Shaded}
\begin{Highlighting}[]
\DocumentationTok{\#\#\# visualizando as predições do modelo em comparação com os valores reais em plots}
\CommentTok{\# Redefinir o índice de linha(row.names)}
\FunctionTok{rownames}\NormalTok{(IDHM.predições.df ) }\OtherTok{\textless{}{-}} \ConstantTok{NULL}
\CommentTok{\# Ordenar os dados (sort)}
\NormalTok{IDHM.predições.df }\OtherTok{\textless{}{-}}\NormalTok{ IDHM.predições.df[}\FunctionTok{order}\NormalTok{(IDHM.predições.df}\SpecialCharTok{$}\NormalTok{IDHM),]}
\CommentTok{\# Plotar as predições versus o IDHM real}
\FunctionTok{plot}\NormalTok{(IDHM.predições.df}\SpecialCharTok{$}\NormalTok{IDHM.predições}\FloatTok{.1}\NormalTok{, }\AttributeTok{type =} \StringTok{"l"}\NormalTok{, }\AttributeTok{col=}\StringTok{"red"}\NormalTok{, }
     \AttributeTok{xlab=}\StringTok{"Dados Testados"}\NormalTok{, }\AttributeTok{ylab=}\StringTok{"Real vs. Predição"}\NormalTok{, }\AttributeTok{main=}\StringTok{""}\NormalTok{)}

\CommentTok{\# Adicionar linhas do IDHM Real}
\FunctionTok{lines}\NormalTok{(IDHM.predições.df}\SpecialCharTok{$}\NormalTok{IDHM, }\AttributeTok{lwd=}\DecValTok{2}\NormalTok{)}

\CommentTok{\# Adicionar a legenda}
\FunctionTok{legend}\NormalTok{(}\StringTok{"topright"}\NormalTok{, }
       \AttributeTok{legend=}\FunctionTok{c}\NormalTok{(}\StringTok{"Predição IDHM"}\NormalTok{, }\StringTok{"IDHM Real"}\NormalTok{),}
       \AttributeTok{col=}\FunctionTok{c}\NormalTok{(}\StringTok{"red"}\NormalTok{, }\StringTok{"black"}\NormalTok{),}
       \AttributeTok{lty=}\DecValTok{1}\NormalTok{,}
       \AttributeTok{cex=}\FloatTok{0.8}\NormalTok{)}

\CommentTok{\# Adicionar título e fonte no estilo ABNT}
\FunctionTok{title}\NormalTok{(}\AttributeTok{main =} \StringTok{"Gráfico 3 {-} Variação da Predição do IDHM"}\NormalTok{, }\AttributeTok{adj =} \DecValTok{0}\NormalTok{)}
\FunctionTok{mtext}\NormalTok{(}\StringTok{"Fonte: Elaboração própria com dados do Atlas Brasil"}\NormalTok{, }\AttributeTok{side=}\DecValTok{1}\NormalTok{, }\AttributeTok{line=}\DecValTok{4}\NormalTok{, }\AttributeTok{adj =} \DecValTok{0}\NormalTok{, }\AttributeTok{cex=}\FloatTok{0.8}\NormalTok{)}
\end{Highlighting}
\end{Shaded}

\includegraphics{ML-regressão-IDHM_files/figure-latex/Plotagem das linhas de valores reais x previstos no modelo-1.pdf}

\begin{Shaded}
\begin{Highlighting}[]
\CommentTok{\#adaptação do código desenvolvido por Anne (2020).}
\end{Highlighting}
\end{Shaded}

PUm modelo comparativo foi feito com a divisão dos dados de 90\% para
treino e 10\% para teste, entretanto o modelo não teve um desempenho
significativamente diferente, o modelo 1 teve desempenho de 97,6\% de
explicação, enquanto no modelo com divisão 90/10 teve desempenho de
97,52\%.

A raiz do erro quadrático médio do modelo 1 foi 0.008417118, no modelo
com divisão 90/10 o RSME foi de 0.006009533 uma queda de 0,002407585 no
RMSE, o indica um melhor ajuste do modelo aos dados. Entretanto a
mudança na divisão dos dados não leva á mudanças significativas, visto
que embora o RSME tenha diminuído, a explicação da variação dos dados
modelo diminui.

\textbf{6. CONCLUSÃO}

Foi observado ao longo do trabalho que o aprendizado de máquina está
popularizando no campo econômico com economistas renomados, economista
chefe em empresas grandes e laureados do Prémio de Ciências Económicas
em Memória de Alfred Nobel já dedicam atenção a área do Machine Learning
como o professor Ph.D. Guido Imbens, e o Economista Chefe da Google Inc.
Hal Varian.

O processo de interpretação dos resultados gerados por modelos de
aprendizado de máquina e a subsequente comunicação destes para
diferentes públicos-alvo é um desafio que envolve um leque de
habilidades que vão além do domínio técnico do aprendizado de máquina.
Em particular, essas habilidades são de natureza interdisciplinar,
incorporando elementos de estatística, ciência da computação,
visualização de dados e comunicação.

Uma das etapas críticas na interpretação dos resultados é a avaliação da
importância das variáveis no modelo de aprendizado de máquina. Para o
modelo de Florestas Aleatórias (Random Forest), esta tarefa geralmente
envolve a análise da diminuição da precisão do modelo quando os valores
de uma variável são embaralhados de maneira aleatória, mantendo todos os
outros valores constantes. Assim, as variáveis que geram uma queda mais
acentuada na precisão são consideradas mais importantes. Este método,
conhecido como ``importância de permutação'', fornece uma medida
quantitativa da contribuição de cada variável para o poder preditivo do
modelo.

No entanto, é importante notar que a importância das variáveis pode ser
influenciada por uma série de fatores, incluindo a correlação entre
variáveis e a escala em que as variáveis são medidas. Assim, a
interpretação da importância das variáveis deve ser realizada com
cautela e em combinação com outras técnicas de interpretabilidade do
modelo, como análise de componentes principais ou mapas de calor de
correlação.

O modelo obteve um excelente grau de ajuste aos dados no treino, não
chegando à uma explicação de 100\% da variância (overfitting) , se
mantendo a uma explicação de 97,6\%, com isso conclui-se que o modelo
Random Forest teve um erro quadrático médio na casa dos centésimos,
demonstrando uma incidência de erro pequeno para o IDHM, visto que este
varia de 0 à 1.

Ao se construir um data frame para avaliar os valores previstos pelo
modelo em relação ao real, obteve uma média de erro de -0.00127523,
demonstrando que embora haja um ajuste expressivo, é possível observar
uma banda para variar as previsões.

Uma eficiência considerável da previsão com modelos de ML também
observada na comparação dos modelos Random Forest, SVM e AdaBoost na
classificação do IDH feita por Arumnisaa et al, (2023) em que os modelos
SVM e AdaBoost tiveram acurácia de 84,61\% e 80,36\% enquanto o modelo
de RF teve acurácia de 85,23\% na classificação do IDH de
distrito/cidade na Indonésia. Demonstrando que o resultado do desempenho
de 97,6\% do modelo 1 na pesquisa não contrasta com a alta eficiência do
modelo em outros estudos.

Com isso esse trabalho contribuiu para a difusão de ferramental novo na
economia, alinhado à uma demonstração prática detalhada de sua
aplicação, para que futuros entrantes no campo do Machine Learning na
economia possam dispor que um manual passo à passo para auxiliá-los a
desenvolver seus primeiros modelos de Machine Learning.

Junto aos trabalhos apresentados em que pesquisadores aplicaram o ML não
somente a previsão de indicadores como o IDH, mas ao estudo do
crescimento econômico, previsão de preços no setor energético na
Alemanha (VOGT, J.2021), preço de ações (Агнон Х.О.. 2021), junto a
modelos de series temporais como o caso do Prophet, o ML demonstra
capacidade de fazer previsões e modelações de variáveis econômicas, o
que demonstra esta ser uma excelente ferramenta para os economistas
aprender e replicá-las.

\textbf{REFERÊNCIAS}

ATHEY, S. The Impact of Machine Learning on Economics. Disponível em:
\url{https://www.nber.org/books-and-chapters/economics-artificial-intelligence-agenda/impact-machine-learning-economics}.

ATHEY, S.; IMBENS, G. W. Machine Learning Methods That Economists Should
Know About. Annual Review of Economics, v. 11, n. 1, p.~685--725, 2 ago.
2019.

CONCEIÇÃO, P. Human Development Report. LANHAM: BERNAN PRESS, 2022.

SISTEMA FIRJAN • IFDM • 2018 METODOLOGIA. {[}s.l: s.n.{]}. Disponível
em:
\url{https://www.firjan.com.br/data/files/E8/06/F0/D5/58E1B610E6543AA6A8A809C2/Metodologia\%20IFDM\%20-\%20Final.pdf}.
Acesso em: 24 jun. 2023.

IFDM \textbar{} Índice FIRJAN de Desenvolvimento Municipal: Consulta.
Disponível em: \url{https://www.firjan.com.br/ifdm/}.

AJAY AGRAWAL; GANS, J.; AVI GOLDFARB. The economics of artificial
intelligence : an agenda. Chicago: The University Of Chicago Press,
2019.

KAUR, M. et al. Supervised Machine-Learning Predictive Analytics for
National Quality of Life Scoring. Applied Sciences, v. 9, n. 8, p.~1613,
18 abr. 2019.

SHERMAN, L. et al. Global High-Resolution Estimates of the United
Nations Human Development Index Using Satellite Imagery and
Machine-learning. 1 mar. 2023.

TOBAIGY, F.; ALAMOUDI, M.; BAFAIL, O. Human Development Index:
Determining and Ranking the Significant Factors. International Journal
of Engineering Research \& Technology, v. 12, n. 3, 29 mar. 2023.

MULLAINATHAN, S.; SPIESS, J. Machine Learning: An Applied Econometric
Approach. The Journal of Economic Perspectives, v. 31, n. 2, p.~87--106,
2017.

DONALDSON, D.; STOREYGARD, A. The View from Above: Applications of
Satellite Data in Economics. Journal of Economic Perspectives, v. 30, n.
4, p.~171--198, nov. 2016.

BHATTAD, S. et al. EasyChair Preprint Review of Machine Learning
Techniques for Cryptocurrency Price Prediction REVIEW OF MACHINE
LEARNING TECHNIQUES FOR CRYPTOCURRENCY PRICE PREDICTION. {[}s.l:
s.n.{]}. Disponível em:
\url{https://easychair.org/publications/preprint_open/t5fX}. Acesso em:
14 nov. 2023.

SONI, P.; TEWARI, Y.; KRISHNAN, D. Machine Learning Approaches in Stock
Price Prediction: A Systematic Review. Journal of Physics: Conference
Series, v. 2161, n. 1, p.~012065, 1 jan. 2022.

NIKOU, M.; MANSOURFAR, G.; BAGHERZADEH, J. Stock price prediction using
DEEP learning algorithm and its comparison with machine learning
algorithms. Intelligent Systems in Accounting, Finance and Management, 3
dez. 2019.

JEAN, N. et al. Combining satellite imagery and machine learning to
predict poverty. Science, v. 353, n. 6301, p.~790--794, 18 ago. 2016.

ARUMNISAA, R. I.; WIJAYANTO, A. W. Comparison of Ensemble Learning
Method: Random Forest, Support Vector Machine, AdaBoost for
Classification Human Development Index (HDI). Sistemasi: Jurnal Sistem
Informasi, v. 12, n. 1, p.~206--218, 31 jan. 2023.

VOGT, JAN. ``Vorhersage von Aktienkursbewegungen der Energiebranche
mithilfe maschinellen Lernens und Stimmungserkennung von Beitr¨agen aus
sozialen Medien'', 2021

BRIGO, FRANCESCO. ``Applicazione di tecniche di Machine Learning per
l'analisi del ruolo del capitale umano nelle startup'', 2019.

PESCI, P. Previsione del prezzo delle azioni di S\&P con reti neurali
LSTM e GRU. 2021.

СЕМЯННИКОВ, Г. В. Рекомендательный сервис для торговли акциями на
фондовом рынке с применением машинного обучения. Естественные и
Технические Науки, n. №04, p.~131--134, 2020.

АГНОН Х.О.. ``прогнозирование цены акций с использованием машинного
обучения'' Инновационная наука, no. 6, 2021, pp.~21-26.

VARIAN, H. R. Big Data: New Tricks for Econometrics. Journal of Economic
Perspectives, v. 28, n. 2, p.~3--28, maio 2014.

OZDEN, E.; GULERYUZ, D. Optimized Machine Learning Algorithms for
Investigating the Relationship Between Economic Development and Human
Capital. Computational Economics, 24 set. 2021.

Metodologia Índice de Desenvolvimento Humano Municipal - IDHM:
metodologia Disponível em:
\url{https://onedrive.live.com/?authkey=\%21AHWsj\%2DUGXcU7LKE\&id=124653557C0404EC\%2122849\&cid=124653557C0404EC\&parId=root\&parQt=sharedby\&o=OneUp}.

ANNE, J. Predicting the Human Development Index. Disponível em:
\url{https://github.com/julieanneco/predictingHDI}.

AREL-BUNDOCK, V.; BACHER, E. WDI: World Development Indicators and Other
World Bank Data. Disponível em:
\url{https://cran.r-project.org/web/packages/WDI/index.html}.

IRIZARRY, R. A. Introduction to Data Science. {[}s.l.{]} CRC Press,
2019.

Atlas Brasil. Disponível em:
\url{http://www.atlasbrasil.org.br/acervo/atlas}.

UNDP (United Nations Development Programme). 1990. Human Development
Report 1990: Concept and Measurement of Human Development. New York.

UNITED NATIONS DEVELOPMENT PROGRAMME; FUNDAÇÃO JOÃO PINHEIRO; INSTITUTO
DE PESQUISA ECONÔMICA APLICADA. O Índice de Desenvolvimento Humano
Municipal Brasileiro. Brasília, Dstrito Federal, Brazil: Pnud, Dezembro,
2013.

Liaw A, Wiener M (2002). ``Classification and Regression by
randomForest.'' R News, 2(3), 18-22.
\url{https://CRAN.R-project.org/doc/Rnews/}.

LIAW, A.; WIENER, M. Classification and Regression by randomForest. R
News, v. 2, n. 3, 2002.

WICKHAM, H.; HENRY, L.; RSTUDIO. tidyr: Tidy Messy Data. Disponível em:
\url{https://cran.r-project.org/web/packages/tidyr/index.html}.

WICKHAM, H. et al.~readxl: Read Excel Files. Disponível em:
\url{https://cran.r-project.org/web/packages/readxl/index.html}.

WICKHAM, H. The Split-Apply-Combine Strategy for Data Analysis. Journal
of Statistical Software, v. 40, n. 1, 2011.

WICKHAM, H. plyr: Tools for Splitting, Applying and Combining Data.
Disponível em:
\url{https://cran.r-project.org/web/packages/plyr/index.html}.

Wei T, Simko V (2021). R package `corrplot': Visualization of a
Correlation Matrix. (Version 0.92),
\url{https://github.com/taiyun/corrplot}.

WEI, T. et al.~corrplot: Visualization of a Correlation Matrix.
Disponível em:
\url{https://cran.r-project.org/web/packages/corrplot/index.html}.

NEUWIRTH, E. RColorBrewer: ColorBrewer Palettes. Disponível em:
\url{https://cran.r-project.org/web/packages/RColorBrewer/index.html}.

WICKHAM, H. Ggplot2 : Elegant Graphics For Data Analysis. {[}s.l.{]}
Springer-Verlag New York, 2016.

Create Elegant Data Visualisations Using the Grammar of Graphics {[}R
package ggplot2 version 3.2.1{]}. R-project.org, 2019.

WILKINSON, L. Grammar of graphics. {[}s.l.{]} Springer, 2005.

KASSAMBARA, A. ggpubr: ``ggplot2'' Based Publication Ready Plots.
Disponível em:
\url{https://cran.r-project.org/web/packages/ggpubr/index.html}.

KUHN, M. Building Predictive Models inRUsing thecaretPackage. Journal of
Statistical Software, v. 28, n. 5, 2008.

KUHN, M. et al.~caret: Classification and Regression Training.
Disponível em:
\url{https://cran.r-project.org/web/packages/caret/index.html}.

TUSZYNSKI, J. caTools: Tools: Moving Window Statistics, GIF, Base64, ROC
AUC, etc. Disponível em:
\url{https://cran.r-project.org/web/packages/caTools/index.html}.

ANNE, J. Predicting the Human Development Index. Disponível em:
\url{https://github.com/julieanneco/predictingHDI}. Acesso em: 12 out.
2023.

BREIMAN, L. Random Forests. Machine Learning, v. 45, n. 1, p.~5--32,
2001.

SALA-I-MARTIN, X. X. I Just Ran Two Million Regressions. The American
Economic Review, v. 87, n. 2, p.~178--183, 1997.

Calculating the human development indices-graphical presentation
Inequality-adjusted Human Development Index (IHDI) Knowledge Human
Development Index (HDI) Long and healthy life A decent standard of
living Human Development Index (HDI) Knowledge Long and healthy life A
decent standard of living Inequality-adjusted Human Development Index
(IHDI) Health Education. {[}s.l: s.n.{]}. Disponível em:
\url{https://hdr.undp.org/system/files/documents/technical-notes-calculating-human-development-indices.pdf}.

Atlas Brasil. Disponível em:
\url{http://www.atlasbrasil.org.br/acervo/atlas}.

The Demography of Tropical Africa William Brass . {[}et Al{]}. {[}s.l:
s.n.{]}.

\end{document}
